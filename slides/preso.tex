\documentclass[22pt,aspectratio=169]{beamer}

% There are many different themes available for Beamer. A comprehensive
% list with examples is given here:
% http://deic.uab.es/~iblanes/beamer_gallery/index_by_theme.html
% You can uncomment the themes below if you would like to use a different
% one:
%\usetheme{AnnArbor}
%\usetheme{Antibes}
%\usetheme{Bergen}
%\usetheme{Berkeley}
%\usetheme{Berlin}
%\usetheme{Boadilla}
\usetheme{boxes}
%\usetheme{CambridgeUS}
%\usetheme{Copenhagen}
%\usetheme{Darmstadt}
%\usetheme{default}
%\usetheme{Frankfurt}
%\usetheme{Goettingen}
%\usetheme{Hannover}
%\usetheme{Ilmenau}
%\usetheme{JuanLesPins}
%\usetheme{Luebeck}
% \usetheme{Madrid}
%\usetheme{Malmoe}
%\usetheme{Marburg}
%\usetheme{Montpellier}
%\usetheme{PaloAlto}
%\usetheme{Pittsburgh}
%\usetheme{Rochester}
%\usetheme{Singapore}
%\usetheme{Szeged}
% \usetheme{Warsaw}
% \setbeamertemplate{footline}{} % remove footnote line

\usepackage[T2A]{fontenc}           % кодировка
\usepackage[utf8]{inputenc}         % кодировка исходного текста
\usepackage[english,russian]{babel}
\hypersetup{unicode=true}
% \usepackage[T1]{fontenc}
\usepackage{mathtools}
\usepackage{listings}
\usepackage{tikz}
\usetikzlibrary{matrix, patterns, patterns.meta, backgrounds, automata, positioning, tikzmark, calc, arrows, arrows.meta, shapes, shadows.blur, shapes.misc, shapes.callouts}
\usepackage{minted}
\usepackage{advdate}
\usepackage[normalem]{ulem}
\usepackage[export]{adjustbox}
\usepackage{makecell}
\usepackage{environ}
\usepackage{amsmath}
\usepackage{multirow}
\usepackage{pgfplots}
\usepackage{natbib}
\usepackage{bibentry}
\usepackage{nameref}
\usepackage{etoolbox}
\usepackage{hyperref}
\usepackage{pifont} % for ding
\usepackage{subcaption} % for subfigure
\usepackage{appendixnumberbeamer}
\usepackage{diagbox}
\usepackage{stmaryrd}
\setbeamertemplate{blocks}[rounded][shadow=false]

\setbeamertemplate{footline}{}
\beamertemplatenavigationsymbolsempty
\setbeamertemplate{footline}[frame number]
\let\otp\titlepage
\renewcommand{\titlepage}{\otp\addtocounter{framenumber}{-1}}


% \BeforeBeginEnvironment{itemize}{\vskip-2ex} % remove spaces before itemize

\newtoggle{fastCompile}
\toggletrue{fastCompile}
\newcommand{\whenFullCompile}[1]{\iftoggle{fastCompile}{
{%\setbeamercolor{background canvas}{bg=red}\begin{frame}{FAST COMPILATION MOCK}\end{frame}
}}{#1}}

\newcommand{\btVFill}{\vskip0pt plus 1filll}


\newcommand\blfootnote[1]{%
  \begingroup
  \renewcommand\thefootnote{}\footnote{#1}%
  \addtocounter{footnote}{-1}%
  \endgroup
}

\tikzset{
    onslide/.code args={<#1>#2}{\only<#1>{\pgfkeysalso{#2}}}, 
}
\tikzset{/handlers/.provide style/.code={%
    \pgfkeysifdefined{\pgfkeyscurrentpath/.@cmd}{}%
        {\pgfkeys {\pgfkeyscurrentpath /.code=\pgfkeysalso {#1}}}%
}}

\AtBeginDocument{\setlength\abovedisplayskip{0pt}}
% \AtBeginDocument{\setlength\belowdisplayskip{0pt}}

\pgfkeys{%
    /calloutquote/.cd,
    width/.code                   =  {\def\calloutquotewidth{#1}},
    position/.code                =  {\def\calloutquotepos{#1}}, 
    bubblePosition/.code          =  {\def\calloutquoteBubblePos{#1}}, 
    author/.code                  =  {\def\calloutquoteauthor{#1}},
    /calloutquote/.unknown/.code   =  {\let\searchname=\pgfkeyscurrentname
                                 \pgfkeysalso{\searchname/.try=#1,                                
    /tikz/\searchname/.retry=#1},\pgfkeysalso{\searchname/.try=#1,
                                  /pgf/\searchname/.retry=#1}}
                            }  

\newcommand\calloutquote[2][]{%
       \pgfkeys{/calloutquote/.cd,
         width               = 5cm,
         position            = {(0,-1)},
         author              = {}}
  \pgfqkeys{/calloutquote}{#1}                   
  \node [rectangle callout,callout relative pointer={\calloutquotepos},/calloutquote/.cd,
     #1] (tmpcall) at \calloutquoteBubblePos {#2};
  \node at (tmpcall.pointer){\calloutquoteauthor};    
}  

\newenvironment<>{framesection}[1]{\section{#1}\begin{frame}{#1}#2}{\end{frame}}

\let\oldfootnote\footnote
\renewcommand\footnote[1][]{\oldfootnote[frame,#1]}
\renewcommand{\footref}[1]{\textsuperscript{\ref{#1}}}

\newenvironment<>{varblock}[2][0.9\textwidth]{%
  \begin{center}
    \begin{minipage}{#1}
    \setlength{\textwidth}{#1}%
    \setlength{\linewidth}{\textwidth}% required to itemize respect the width of block
  \begin{actionenv}#3%
    \def\insertblocktitle{#2}%
    \par%
    \usebeamertemplate{block begin}}
  {\par%
  \usebeamertemplate{block end}%
  \end{actionenv}
  \end{minipage}
    \end{center}}



% Variable width example block
\newenvironment<>{varexampleblock}[2][0.9\textwidth]{%
  \begin{center}
    \begin{minipage}{#1}
    \setlength{\textwidth}{#1}%
    \setlength{\linewidth}{\textwidth}%
  \begin{actionenv}#3%
    \def\insertblocktitle{#2}%
    \par%
    \setbeamercolor{local structure}{parent=example text}%
    \usebeamertemplate{block example begin}}
  {\par%
  \usebeamertemplate{block example end}%
    \end{actionenv}
  \end{minipage}
    \end{center}}

% Variable width alert block
\newenvironment<>{varalertblock}[2][0.9\textwidth]{%
  \begin{center}
    \begin{minipage}{#1}
    \setlength{\textwidth}{#1}%
    \setlength{\linewidth}{\textwidth}%
  \begin{actionenv}#3%
    \def\insertblocktitle{#2}%
    \par%
    \setbeamercolor{local structure}{parent=alerted text}%
    \usebeamertemplate{block alerted begin}}
  {\par%
  \usebeamertemplate{block alerted end}%
    \end{actionenv}
  \end{minipage}
    \end{center}}


\tikzset{
    invisible/.style={opacity=0,text opacity=0},
    visible on/.style={alt=#1{}{invisible}},
    alt/.code args={<#1>#2#3}{%
      \alt<#1>{\pgfkeysalso{#2}}{\pgfkeysalso{#3}} 
    },
    beameralert/.style={alt=<#1>{fill=red!30,rounded corners}{},anchor=base},
    BeamerAlert/.style={alt=#1{fill=red!30,rounded corners}{},anchor=base}
}
  \newcommand<>{\tikzMe}[1]{% previously: \def\tikzMe<#1>#2{…
    \tikz[baseline]\node[BeamerAlert=#2,anchor=base] {#1};
}

\newcommand{\mattention}[1]{\color{focusColor}#1} % TODO: \bm
\newcommand<>{\mathattention}[1]{\alt#2{\mattention{#1}}{#1}}
% \addtobeamertemplate{block begin}{\setlength\abovedisplayskip{-1cm}}


\newcommand{\currentTableOfContents}{\begin{frame}{Содержание}\tableofcontents[currentsection]\end{frame}}

\newcommand\eqby[1]{\mathrel{\stackrel{\makebox[0pt]{\mbox{\normalfont\tiny #1}}}{=}}}
\newcommand\eqbyref[1]{\mathrel{\stackrel{\makebox[0pt]{\mbox{\normalfont\tiny\eqref{#1}}}}{=}}}
\newcommand\restr[2]{{% we make the whole thing an ordinary symbol
  \left.\kern-\nulldelimiterspace % automatically resize the bar with \right
  #1 % the function
  \vphantom{\big|} % pretend it's a little taller at normal size
  \right|_{#2} % this is the delimiter
  }}

\newcommand{\csharp}{\textsc{C\#}}
\newcommand{\java}{\textsc{Java}}
\newcommand{\scala}{\textsc{Scala}}
\newcommand{\dotnet}{\textsc{.NET}}
\newcommand{\clang}{\textsc{C}}
\newcommand{\cpplang}{\textsc{C++}}
\newcommand{\coq}{\textsc{Coq}}

\lstdefinelanguage{CSharp}
{
 morecomment = [l]{//}, 
 morecomment = [l]{///},
 morecomment = [s]{/*}{*/},
 morestring=[b]", 
 sensitive = true,
 morekeywords = {abstract,  event,  new,  struct,
  as,  explicit,  null,  switch,
  base,  extern,  object,  this,
  bool,  false,  operator,  throw,
  break,  finally,  out,  true,
  byte,  fixed,  override,  try,
  case,  float,  params,  typeof,
  catch,  for,  private,  uint,
  char,  foreach,  protected,  ulong,
  checked,  goto,  public,  unchecked,
  class,  if,  readonly,  unsafe,
  const,  implicit,  ref,  ushort,
  continue,  in,  return,  using,
  decimal,  int,  sbyte,  virtual,
  default,  interface,  sealed,  volatile,
  delegate,  internal,  short,  void,
  do,  is,  sizeof,  while, where,
  double,  lock,  stackalloc,   
  else,  long,  static,   
  enum,  namespace,  string, Span, var,
  ValueListBuilder}
}

\lstdefinestyle{sharpc}{language=CSharp,
    %frame=lr,
    % frame=l,
    rulecolor=\color{blue!80!black},
    basicstyle=\fontsize{9}{13}\selectfont\ttfamily,
    keywordstyle=\color{blue}\ttfamily,
    stringstyle=\color{red}\ttfamily,
    commentstyle=\color{green}\ttfamily,
    morecomment=[l][\color{magenta}]{\#}
}

\lstdefinelanguage{mycpp}{
  keywords={assume, for, assert},
  keywordstyle=\color{blue}\bfseries,
  keywords=[2]{size_t, int},
  keywordstyle=[2]\color{red}\bfseries,
  identifierstyle=\color{black},
  sensitive=false,
  comment=[l]{//},
  morecomment=[s]{/*}{*/},
  commentstyle=\color{purple}\ttfamily,
  stringstyle=\color{red}\ttfamily,
  morestring=[b]',
  morestring=[b]"
}

\lstset{
   language=mycpp,
   extendedchars=true,
   basicstyle=\small\ttfamily,
   showstringspaces=false,
   showspaces=false,
   tabsize=2,
   breaklines=true,
   showtabs=false
}

\newcommand\eqdef{\mathrel{\stackrel{\makebox[0pt]{\mbox{\normalfont\tiny def}}}{=}}}

\newcommand\drawCodeBox[2]{%
  \begin{tikzpicture}[remember picture,overlay]
    \coordinate (start) at ([yshift=1.7ex]pic cs:#1);
    \coordinate (end) at ([yshift=-0.3ex]pic cs:#2);
    \node[line width=2pt, inner sep=2pt,draw=red,fit=(start) (end)] {};
  \end{tikzpicture}%
}


\newcommand\pair[2]{\langle#1, #2\rangle}
\newcommand\mg[2]{#1=#2}
\newcommand\nmg[2]{#1\neq#2}
\newcommand\li[1]{LI(#1)}
\let\emptyheap\epsilon
\newcommand\agrec{Rec}
\newcommand\agmerge{Merge}
\newcommand\agcompose{\bigcirc}
\newcommand\GRec[1]{\agrec\big(#1\big)}
\newcommand\GMerge[1]{\agmerge\big(#1\big)}
\newcommand\GCompose[2]{#1\agcompose#2}
\newcommand\agho{App}
\newcommand\gapp[1]{\agho(#1)}
\newcommand\GApp[1]{\agho\big(#1\big)}
\newcommand\aunion{union}
\newcommand\union[1]{\aunion\big(#1\big)}
\newcommand\Union[1]{\aunion\Big(#1\Big)}
\newcommand\aderef{read}
\newcommand\afind{find}
\newcommand\find[5]{\afind(#1,#2,#3,#4,#5)}
\newcommand\finD[5]{\afind\big(#1,#2,#3,#4,#5\big)}
\newcommand\Find[5]{\afind\Big(#1,#2,#3,#4,#5\Big)}
\newcommand\deref[2]{\aderef(#1,#2)}
\newcommand\Deref[2]{\aderef\big(#1,#2\big)}
\newcommand\compose[2]{#1\circ#2}
\newcommand\lmbd[2]{\lambda #1.#2}
\newcommand\lmbdx[1]{\lambda x.#1}
\newcommand\dom[1]{dom(#1)}
\newcommand\Dom[1]{dom\big(#1\big)}

\newcommand\exprset{Expr}
\newcommand\guardset{Guard}

\newcommand\pro{\item[$+$]}
\newcommand\contra{\item[$-$]}

\newcommand{\ruquote}[1]{«#1»}

\theoremstyle{plain}
\newtheorem{thm}{Теорема}%[section]
\newtheorem{lem}{Лемма}%[section]
% \newtheorem{crlr}{Corollary}%[section]

\theoremstyle{definition}
\newtheorem{defn}{Определение}
\newtheorem{remk}{Замечание}
\newtheorem{prop}{Утверждение}
\newtheorem{exmp}{Пример}

\newcommand\encircle[1]{%
  \tikz[baseline=(X.base)] 
    \node (X) [draw, shape=circle, inner sep=0] {\strut #1};}
    
    

\definecolor{dgreen}{rgb}{0.,0.6,0.}
\definecolor{focusGreen}{RGB}{7,225,19}
\definecolor{focusRed}{RGB}{254,31,31}
\definecolor{focusViolet}{RGB}{143,00,255}
\definecolor{focusBlue}{HTML}{1620A6}

\makeatletter
\pgfdeclarepatternformonly[\LineSpace]{my north east lines}{\pgfqpoint{-1pt}{-1pt}}{\pgfqpoint{\LineSpace}{\LineSpace}}{\pgfqpoint{\LineSpace}{\LineSpace}}%
{
    \pgfsetcolor{\tikz@pattern@color}
    \pgfsetlinewidth{0.4pt}
    \pgfpathmoveto{\pgfqpoint{0pt}{0pt}}
    \pgfpathlineto{\pgfqpoint{\LineSpace + 0.1pt}{\LineSpace + 0.1pt}}
    \pgfusepath{stroke}
}

\pgfdeclarepatternformonly[\LineSpace]{my north west lines}{\pgfqpoint{-1pt}{-1pt}}{\pgfqpoint{\LineSpace}{\LineSpace}}{\pgfqpoint{\LineSpace}{\LineSpace}}%
{
    \pgfsetcolor{\tikz@pattern@color}
    \pgfsetlinewidth{0.4pt}
    \pgfpathmoveto{\pgfqpoint{0pt}{\LineSpace}}
    \pgfpathlineto{\pgfqpoint{\LineSpace + 0.1pt}{-0.1pt}}
    \pgfusepath{stroke}
}
\makeatother

\newdimen\LineSpace
\tikzset{
    line space/.code={\LineSpace=#1},
    line space=7pt
}

\newcommand{\codefontsize}{\footnotesize}

\NewEnviron{onslideenv}{\onslide<2>{\BODY}}{}


\renewcommand{\phi}{\varphi}

\newcommand{\pdr}{IC3/PDR}
\newcommand{\cegar}{CEGAR}
\newcommand{\ice}{ICE}
\newcommand{\fmf}{FMF}
\newcommand{\satur}{Насыщение}

\newcommand{\theregclass}{Reg}
\newcommand{\regclass}{\textsc{\theregclass{}}}
\newcommand{\theelemclass}{Elem}
\newcommand{\elemclass}{\textsc{\theelemclass{}}}
\newcommand{\elemextclass}{\textsc{\theelemclass{}\textsubscript{*}}}
\newcommand{\sizeelemclass}{\textsc{Size\theelemclass{}}}
\newcommand{\sizeelemextclass}{\textsc{Size\theelemclass{}\textsubscript{*}}}
\newcommand{\catelemclass}{\textsc{Cat\theelemclass{}}}
\newcommand{\syncRegFlatClass}{\regclass{}\textsubscript{+}} %TODO: take it from Haudebourg
\newcommand{\syncRegFullClass}{\regclass{}\textsubscript{$\times$}}
\newcommand{\regelemclass}{\textsc{\theelemclass{}\theregclass{}}}

\newcommand{\zprover}{\textsc{Z3}}
\newcommand{\cvc}{\textsc{cvc5}}
\newcommand{\princess}{\textsc{Princess}}
\newcommand{\chaff}{\textsc{Chaff}}
\newcommand{\cvcind}{\textsc{\cvc{}-Ind}}
\newcommand{\mace}{\textsc{Mace4}}
\newcommand{\kodkod}{\textsc{Kodkod}}
\newcommand{\paradox}{\textsc{Paradox}}
\newcommand{\vampire}{\textsc{Vampire}}
\newcommand{\eprover}{\textsc{E}}
\newcommand{\zipperposition}{\textsc{Zipperposition}}

\newcommand{\eldarica}{\textsc{Eldarica}}
\newcommand{\theringen}{\textsc{RInGen}}
\newcommand{\ringenAny}[2]{#1(#2)}
\newcommand{\ringenSyncAny}[1]{\textsc{#1-Sync}}
\newcommand{\theringenCICIAny}[1]{\textsc{#1-CICI}}
\newcommand{\ringenCICIAny}[2]{\theringenCICIAny{#1}(#2)}
\newcommand{\ringen}[1]{\ringenAny{\theringen}{#1}}
\newcommand{\ringenSync}{\ringenSyncAny{\theringen}}
\newcommand{\theringenCICI}{\theringenCICIAny{\theringen}}
\newcommand{\ringenCICI}[1]{\ringenCICIAny{\theringen}{#1}}
\newcommand{\verifier}{\mathcal{V}}
\newcommand{\spacer}{\textsc{Spacer}}
\newcommand{\racer}{\textsc{Racer}}
\newcommand{\hoice}{\textsc{HoIce}}
\newcommand{\rchc}{\textsc{RCHC}}
\newcommand{\vericat}{\textsc{VeriCaT}}
\newcommand{\verimap}{\textsc{VeriMAP-iddt}}

\newcommand{\signature}{\Sigma}
\newcommand{\theory}{\mathcal{T}}
\newcommand{\relations}{\mathcal{R}}
\newcommand{\oracle}{\mathcal{O}}
\newcommand{\ourCEGAR}{\cegar{}($\oracle$)}
\newcommand{\prog}{\mathcal{S}}
\newcommand{\fsymbs}{\signature_F}
\newcommand{\btuple}[1]{\big\langle #1 \big\rangle}
\newcommand{\theautomaton}[4]{\btuple{#1, #2, #3, #4}}
\newcommand{\autInit}{s_0}
\newcommand{\autStates}{S}
\newcommand{\autFinStates}{\autStates_F}
\newcommand{\autTrans}{\Delta}
\newcommand{\autApp}[1]{A\left(#1\right)}
\newcommand{\automaton}[3]{\theautomaton{#1}{\fsymbs}{#2}{#3}}
\newcommand{\automatonDef}{\automaton{\autStates}{\autFinStates}{\autTrans}}
\newcommand{\Automaton}[2]{\automaton{\{#1\}}{\{#2\}}{\autTrans}}
\newcommand{\sautomatonGen}[4]{\theautomaton{#1}{\fsymbs^{\leq #2}}{#3}{#4}}
\newcommand{\sautomaton}[3]{\sautomatonGen{\{#1\}}{#2}{\{#3\}}{\autTrans}}


\newcommand{\structur}{\mathcal{H}}
\newcommand{\mystructur}{\mathcal{M}}
\newcommand{\invariant}{\mathcal{I}}
\newcommand{\thedomain}{\left|\mystructur\right|}
\newcommand{\domain}[1]{\thedomain_{#1}}
\newcommand{\tuple}[1]{\langle #1 \rangle}


\let\oldemptyset\emptyset
\let\emptyset\varnothing

\newcommand{\formallang}{\mathbf{L}}
\newcommand{\automat}{\mathcal{A}}
\newcommand{\langOf}[1]{\mathcal{L}(#1)}
\newcommand{\height}[1]{\mathcal{H}eight(#1)}

\definecolor{badInvariantColor}{RGB}{255,50,50}
\definecolor{goodInvariantColor}{RGB}{5,150,2}
\definecolor{darkyellow}{RGB}{153,153,0}
\definecolor{focusColor}{RGB}{230,5,5}
\definecolor{nicegreen}{RGB}{5,150,2}
% \definecolor{nicegreen}{rgb}{0.0, 0.8, 0.3}
\newcommand{\xmark}{\text{\ding{53}}}%

\newcommand{\attention}[1]{\textcolor{badInvariantColor}{\textbf{#1}}}

\makeatletter
\newcommand*{\currentname}{\@currentlabelname}
\makeatother

\newcommand{\beamermessage}[2]{%
\begin{beamercolorbox}[wd=#1,center,rounded=true]{author in head/foot}
#2
%[wd=#1,ht=3ex,dp=2ex,center,rounded=true]{author in head/foot}
%\hfill\parbox[c][7ex][c]{#1}{\centering\LARGE #2}\hfill\null
\end{beamercolorbox}}

\makeatletter
\newcommand*\Alt{\alt{\Alt@branch0}{\Alt@branch1}}

\newcommand\Alt@branch[3]{%
  \begingroup
  \ifbool{mmode}{%
    \mathchoice{%
      \Alt@math#1{\displaystyle}{#2}{#3}%
    }{%
      \Alt@math#1{\textstyle}{#2}{#3}%
    }{%
      \Alt@math#1{\scriptstyle}{#2}{#3}%
    }{%
      \Alt@math#1{\scriptscriptstyle}{#2}{#3}%
    }%
  }{%
    \sbox0{#2}%
    \sbox1{#3}%
    \Alt@typeset#1%
  }%
  \endgroup
}

\newcommand\Alt@math[4]{%
  \sbox0{$#2{#3}\m@th$}%
  \sbox1{$#2{#4}\m@th$}%
  \Alt@typeset#1%
}

\newcommand\Alt@typeset[1]{%
  \ifnumcomp{\wd0}{>}{\wd1}{%
    \def\setwider   ##1##2{##2##1##2 0}%
    \def\setnarrower##1##2{##2##1##2 1}%
  }{%
    \def\setwider   ##1##2{##2##1##2 1}%
    \def\setnarrower##1##2{##2##1##2 0}%
  }%
  \sbox2{}%
  \sbox3{}%
  \setwider2{\wd}%
  \setwider2{\ht}%
  \setwider2{\dp}%
  \setnarrower3{\ht}%
  \setnarrower3{\dp}%
  \leavevmode
  \rlap{\usebox#1}%
  \usebox2%
  \usebox3%
}
\makeatother

\newcommand{\Focus}[2]{{\begin{center}#1 #2\end{center}}}
\newcommand{\focus}[1]{\Focus{\huge}{#1}}

% \DeclareFieldFormat*{citetitle}{<<#1>>} % removes ``quotes'' from title
% \newcommand{\citee}[1]{\citetitle{#1}}
\newcommand{\citee}[1]{\bibentry{#1}}

\newcommand{\chccomp}{CHC-COMP}
\newcommand{\fsharp}{F\#}

\delimitershortfall=-1pt % вложенные скобки

\newcolumntype{x}[1]{>{\centering\let\newline\\\arraybackslash\hspace{0pt}}p{#1}}
\newcommand{\tnote}[1]{$^{#1}$}
\newcommand{\exRef}[1]{\textit{#1}}
\newcommand{\exEvenLeft}{\exRef{even}}
\newcommand{\exLR}{\exRef{lr}}
\newcommand{\exLt}{\exRef{lt}}
\newcommand{\exNode}{\exRef{node}}
\title{Автоматический вывод индуктивных инвариантов программ с алгебраическими типами данных}
\titlegraphic{\includegraphics[height=.1\textheight]{resources/block_ru.pdf}}

\author{Костюков Юрий Олегович}
% {\footnotesize\textcolor{gray}{группа 571\\научный руководитель: cт. преп. Д.\,А. Мордвинов}}

% - Give the names in the same order as the appear in the paper.
% - Use the \inst{?} command only if the authors have different
%   affiliation.

\institute[] % (optional, but mostly needed)
{
    Научный руководитель:\\д.\:т.\:н., доцент Кознов Дмитрий Владимирович
    % Санкт-Петербургский государственный университет
}
% - Use the \inst command only if there are several affiliations.
% - Keep it simple, no one is interested in your street address.

% \day=15
% \month=12
% \year=2023

% \date{\today}
\date{2024}

\begin{document}

% \toggletrue{fastCompile}
\togglefalse{fastCompile}

\newcommand\exprVsAutomationPlot{
\begin{tikzpicture}[domain=0:4]
    \draw[->] (-0.5,0) -- (7,0) node[below] {Возможность автоматизации};
    \draw[->] (0,-0.5) -- (0,5) node[above] {Выразительность};
    \draw (2,4) node [circle,fill,inner sep=1pt,label=above:\textsc{Coq}-термы] {};
    \draw (5,2.5) node [circle,fill,inner sep=1pt,label=above:\emph{Новое представление}] {};
    \draw (5,1.25) node [circle,fill,inner sep=1pt,label=above:FOL-формулы] {};
    % \draw[color=red] plot[id=x] function{x} 
    %     node[right] {$f(x) =x$};
    % \draw[color=blue] plot[id=sin] function{sin(x)} 
    %     node[right] {$f(x) = \sin x$};
    % \draw[color=orange] plot[id=exp] function{0.05*exp(x)} 
    %     node[right] {$f(x) = \frac{1}{20} \mathrm e^x$};
\end{tikzpicture}
}

\newcommand{\ciciPic}{
\tikzset{empty block/.style={minimum width=70pt,minimum height=30pt}}
\tikzset{block/.provide style={empty block, draw, rounded rectangle}}
\hspace*{-5mm}
\begin{tikzpicture}[every node/.style={align=center,font=\small}]

\begin{scope}[overlay]
\draw[line width=3,onslide=<3->{dash pattern={on 10pt off 10pt}}] (14, -3) circle (240pt);
\node[rotate=-10] at (-2, -4.5) {\Large\textbf{Хорн-решатели}};
\node[rotate=10] at (8, -4.5) {\Large\textbf{Инструменты}\\\textbf{вывода теорем}};
\end{scope}

\only<1>{
\begin{scope}[overlay,node distance=30pt]
\node (z3) at (0,-1) {\textsc{Z3/Spacer}};
\node[below left=of z3] {\textsc{Eldarica}};
\node[below right=of z3] {\textsc{Hoice}};
\node[below=of z3] {\textsc{Golem}};
\node[above left=of z3] {\textsc{FreqHorn}};
\node[above right=of z3] {\textsc{QARMC}};
\node[above=of z3] {\textsc{Duality}};
\end{scope}}

\only<1-2>{
\begin{scope}[overlay,node distance=20pt]
\node (vampire) at (8,-1) {\textsc{Vampire}};
\node[right=of vampire] {\textsc{E}};
\node[below=of vampire] {\textsc{Zipperposition}};
\node[above=of vampire] {\textsc{iProver}};
\node[above right=of vampire] {\textsc{cvc5}};
% \node[above right=of vampire] {\textsc{QARMC}};
% \node[above=of vampire] {\textsc{Duality}};
\end{scope}
}

\uncover<2->{
\begin{scope}[node distance=50pt]
\node[block] (isFeasible) {\textsc{IsFeasible}};
\node[block,below=of isFeasible] (modelCheck) {\textsc{Validate}};
\node[block,right=of modelCheck] (refine) {\textsc{Refine}};
\node[left=of isFeasible] (unsafe) {UNSAT};
\node[left=of modelCheck] (safe) {SAT};
\draw[->] (isFeasible) -- (unsafe) node [midway,above] {трасса};
\draw[->] (modelCheck) -- (safe) node [midway] {инд.\\инвариант};
\draw[->] (refine) -- (modelCheck);
\draw[->] (modelCheck) -- (isFeasible) node [midway,left=-1mm,align=right] {абстрактная\\трасса};

\uncover<3->{
\node[block,above=of refine] (runOracle) {\textsc{Collaborate}};
\path (runOracle) -- (refine) node [midway] (tmp) {};
\node[block,right=108pt of tmp] (atp) {Инструмент\\вывода\\теорем};
\draw[->] (isFeasible) -- (runOracle);
\draw[->] (runOracle) -- (atp) node [pos=.65,sloped] {\textit{остаточная}\\\textit{система}};
\draw[->] (atp) -- (refine) node [pos=.4,sloped] {\textcolor{focusRed}{абстр. трасса}\\\textcolor{focusGreen}{част. инвариант}};
\only<4>{
\calloutquote[width=30mm,textwidth=30mm,position={(0.7,0.1)},bubblePosition={(3,-1.5)},callout pointer width=.4cm,fill=yellow!80,rounded corners]{Её достижимые состояния~--- это\\состояния исходной без состояний\\текущего кандидата в инварианты}
}
}
\only<2>{
\node (runOracle) [empty block,above=of refine] {};
\draw (isFeasible) -- (runOracle.center);
\draw[->] (runOracle.center) -- (refine);
}
\end{scope}}



% \onslide<3>{\filldraw [line width=3,fill opacity=0.4,fill={rgb, 255:red, 7; green, 225; blue, 19 }] (-7.5, 2) circle (300pt);}

% \onslide<4>{
% % print(''.join(["\\textcolor{focus%s}{\\textsc{%s}}" % ("Red" if i % 2 else "Green", x) for i,x in enumerate(a)]))
% \node[draw=focusRed,dash pattern={on 7pt off 9pt},postaction={draw=focusGreen, dash phase=8pt},align=center,inner sep=5mm,line width=2.75pt] at (500,320) {\LARGE \textcolor{focusGreen}{\textsc{C}}\textcolor{focusRed}{\textsc{o}}\textcolor{focusGreen}{\textsc{l}}\textcolor{focusRed}{\textsc{l}}\textcolor{focusGreen}{\textsc{a}}\textcolor{focusRed}{\textsc{b}}\textcolor{focusGreen}{\textsc{o}}\textcolor{focusRed}{\textsc{r}}\textcolor{focusGreen}{\textsc{a}}\textcolor{focusRed}{\textsc{t}}\textcolor{focusGreen}{\textsc{i}}\textcolor{focusRed}{\textsc{o}}\textcolor{focusGreen}{\textsc{n}}};}
\end{tikzpicture}}

\newcommand{\toolplotTwo}[2]{%
% #1 is backend solver name
% #2 is a .csv file name
\pgfplotstableread[col sep = comma]{#2}\toolplotcsv
\begin{subfigure}[t]{0.47\textwidth}
\begin{center}
\begin{tikzpicture}[scale=.55]
\begin{axis}[xmode=log, ymode=log, legend pos= north west, xlabel={\ringenCICI{#1}, мс}, ylabel style = {align=center}, ylabel={{\color{red}\racer{}} и {\color{blue}\ringen{#1}}, мс}]
    \addplot[dashed,no marks,very thin] coordinates {(10,10) (1200000,1200000)};
    \addplot [dashed, no marks, thin] coordinates {(10,10) (1200000,1200000)};
    \addplot [dashed, no marks, thin] coordinates {(10,600000) (600000,600000)};
    \addplot [dashed, no marks, thin] coordinates {(600000, 10) (600000,600000)};
    \addplot [dashed, no marks, thin] coordinates {(10, 1200000) (1200000,1200000)};
    \addplot [dashed, no marks, thin] coordinates {(1200000, 10) (1200000,1200000)};
    
    % one second
    \addplot [no marks, thin] coordinates {(1000, 10) (1000,1000)};
    \addplot [no marks, thin] coordinates {(10, 1000) (1000,1000)};

    \addplot  [only marks,  mark=o, color=red,  mark size=3pt] table [x={Collab}, y={Z3}] {\toolplotcsv};
    \addplot  [only marks,  mark=triangle, color=blue, mark size=3pt] table [x={Collab}, y={RInGen}] {\toolplotcsv};
\end{axis}
\end{tikzpicture}
\end{center}
\end{subfigure}
}

\newcommand{\residualPic}{
\tikzset{
    importantNode/.style={font=\Large},
    test1/.style={diamond, aspect=2, text width=5em, inner sep=5pt}}
\begin{tikzpicture}
\def\Bad{(9, 1.0) ellipse[x radius=15mm, y radius=15mm]}
\def\Abstrk{(-1.5, -2.1) rectangle ++(4.6, 4.2)}
\def\Abstr{(-2.2, -2.7) rectangle ++(10.5, 5.4)}
\def\TAbstrk{(-1.2, 0) -- (-0.8,0.8) -- (1,1.5) -- (4,1) -- (5,0) -- (4,-1) -- (1,-1.5) -- (-0.8,-0.8) -- cycle}
\def\Reach{(2.7,0) ellipse [x radius=40mm, y radius=20mm]}
\def\ReachRes{(2.8,-2.2) rectangle ++(4.2, 4.4)}
\def\OracleInv{(4.9,0) ellipse [x radius=26mm, y radius=40mm]}
\def\AroundOracleInv{(4.9,0) ellipse [x radius=76mm, y radius=41mm]}


\uncover<1-7>{
\draw (0,0) ellipse [x radius=10mm, y radius=5mm] node (init) {$Init=R_0$};
}

\draw[draw=focusRed,ultra thick] \Bad node[importantNode] {\textcolor{focusRed}{$Bad$}};

\uncover<2-7>{
\draw (0.9,0) ellipse [x radius=20mm, y radius=10mm] node {};
\node[right=3mm of init.east] (dots) {$\ldots$};
\node[right=3mm of dots] (rk) {$R_k$};
}


\uncover<3-8, 13->{\draw[thick] \Reach;}
\uncover<3-11, 13->{\node[importantNode,right=31mm of rk] (justr) {$R_{\invisible{k}}$};}
\uncover<3-11>{\node[left=3mm of justr] (dots2) {$\ldots$};}

\uncover<4>{\draw[overlay,ultra thick] (5, 5) -- (9.5, -5);}


\uncover<6-12>{
\uncover<3-8>{\draw \TAbstrk;}
\alt<8->{
    \node [importantNode,left=5mm of dots2] (tak) {$Init'$};
}{
    \node [importantNode,left=2mm of dots2] (tak) {$T(A_k)$};
}
\begin{scope}
\uncover<9-12>{
  \clip \TAbstrk;
  \draw[thick,focusGreen] \Abstrk;
}
\end{scope}
\begin{scope}
\uncover<9-11>{
  \clip \Reach;
  \draw[thick] \Abstrk;
}
\end{scope}
\begin{scope}[even odd rule]
\uncover<9-12>{
  \clip \Abstrk \Abstr;
  \uncover<-11>{\draw[thick] \Reach;}
  \draw[thick,focusGreen] \TAbstrk;
}
\end{scope}
\begin{scope}[even odd rule]
\uncover<9-12>{
  \clip \Abstrk \Abstr;
  \fill[pattern={Lines[angle=45,distance=4pt,line width=0.3mm]},pattern color=focusGreen] \TAbstrk;
}
\end{scope}
}

\uncover<10-12>{
  \draw[thick,focusGreen] \ReachRes;
  \node[importantNode] (Rres) at (5.8, 1.9) {\textcolor{focusGreen}{$R'$}};
  \path[initial/.tip = {latex},-initial] (tak) edge [bend right,thick,focusGreen] node[black,importantNode,left=-1mm] {$T'^*$} (Rres);
}

\uncover<12>{
  \draw[ultra thick,focusBlue] \OracleInv;
}
\uncover<13->{
  \fill[pattern={Lines[angle=45,distance=4pt,line width=0.3mm]},pattern color=focusBlue] \OracleInv;
  \begin{scope}[overlay,even odd rule]
  \clip \Abstrk \AroundOracleInv;
  \draw[ultra thick,focusBlue] \OracleInv;
  \end{scope}
}
\uncover<12->{\node[importantNode] at (2.9, 3.2) {\textcolor{focusBlue}{$B$}};}

\begin{scope}[overlay,even odd rule]
\uncover<13->{
  \clip \OracleInv \AroundOracleInv;
  \draw[draw=focusViolet,ultra thick] \Abstrk;
  % \fill[pattern=my north west lines,pattern color=focusViolet] \Abstrk;
  % \fill[overlay,pattern=my north west lines,pattern color=focusViolet] \Abstrk;
  % \fill[pattern=north west lines,pattern color=focusViolet] \Abstrk;
  \fill[pattern={Lines[angle=-45,distance=4pt,line width=0.3mm]},pattern color=focusViolet] \Abstrk;
}
\end{scope}

\uncover<5-8>{
\draw[draw=focusViolet,ultra thick,onslide=<8>{fill=white}] \Abstrk;
\node (ak) at (3, -1.1) {};
}
\uncover<5-8,13->{\node[importantNode] at (-1, 2.4) {\textcolor{focusViolet}{$A_k$}};}

\uncover<6-7>{\path[initial/.tip = {latex},-initial] (ak) edge [bend right,thick] (tak);}
\uncover<7>{
\draw[draw=focusViolet,ultra thick] \Abstr;
\node[importantNode] (akp1) at (7.8, -1.1) {\textcolor{focusViolet}{$A_{k+1}$}};
\path[initial/.tip = {latex},-initial] (tak) edge [bend right,thick,focusViolet] (akp1);
\begin{scope}
\uncover<7>{
  \clip \Abstr;
  \fill[pattern={Lines[angle=45,distance=4pt,line width=0.3mm]},pattern color=focusViolet] \Bad;
}
\end{scope}}

\begin{scope}[overlay,font=\Large]
\only<-4>{\node[font=\large] at (-1.5,3) {
    \begin{minipage}{40mm}
    \begin{align*}
    Init(x) &\rightarrow I(x)\\
    I(x) \land Tr(x, x') &\rightarrow I(x')\\
    I(x) \land Bad(x) &\rightarrow \bot
    \end{align*}
    \end{minipage}
};}
\node at (4, -3.4) {\begin{minipage}{\textwidth}
\only<2>{$$R_{k}(x) = R_{k-1}(x) \lor \exists x'. \big(R_{k-1}\left(x'\right) \land Tr\left(x', x\right)\big)$$}
\only<3>{$$R(x) = \bigvee_{k=0}^\infty R_{k}(x)$$}
\only<4>{$$\models \forall x. \big(R(x) \land Bad(x) \rightarrow \bot\big)$$}
\only<5>{\centering for each $i$, $\models R_i(x) \rightarrow A_i(x)$}
\only<6>{$$\text{transition }T(A)(x) = \exists x'. \big(A(x') \land Tr(x', x)\big)$$}
\only<7>{$$A_{k+1} = \alpha\big(T(A_k)\big)\text{ for abstraction }\alpha$$}
\only<8-9>{\large $$Init' \text{ of residual system is } T(A_k)\Alt<8>{\land\neg}{\ \setminus} A_k\text{~--- \textbf{one-step non-inductive states}}$$}
\only<10>{$$\text{transition }T'(B)\text{ of residual system is } T(B)\setminus A_k$$}
\only<11>{\large $$R' = \text{reachable in residual system}=\text{reachable from non-inductive subset of }A_k$$}
\end{minipage}};
\only<7>{\calloutquote[position={(2,-1.9)},bubblePosition={(5.5,3.3)},callout pointer width=.5cm,fill=focusViolet!50,rounded corners]{abstract counterexample}}
\only<12>{\calloutquote[position={(1.77,-1.9)},bubblePosition={(-0.75,2)},callout pointer width=.6cm,fill=focusBlue!50,rounded corners,font=\large]{\begin{minipage}{45mm}
% $B$~-- residual system invariant obtained from oracle:
\begin{align*}
B \text{ -- } &\text{residual system invariant}\\
&\text{obtained from oracle:}
\end{align*}\vspace*{-8mm}
% \alt<13>{\begin{align*}
%     A_k(x') \land Tr(x', x) \land \neg A_k(x) &\rightarrow B(x)\\
%     B(x) \land Tr(x, x') \land \neg A_k(x') &\rightarrow B(x')\\
%     B(x) \land Bad(x) &\rightarrow \bot
% \end{align*}}{
\begin{align*}
    Init'(x) &\rightarrow B(x)\\
    B(x) \land Tr'(x, x') &\rightarrow B(x')\\
    B(x) \land Bad(x) &\rightarrow \bot
    \end{align*}
    % }
\end{minipage}}}
\only<14>{\node[draw,fill=focusViolet!40,opacity=0.95,rounded rectangle,minimum height=2cm,rotate=0,align=center] at (4,-2) {\begin{minipage}{.85\textwidth}
\centering $A_k \lor B$~--- \textbf{combined invariant} of the original system:
\begin{align*}
R(x) &\rightarrow A_k(x)\lor B(x)\\
\big(A_k(x)\lor B(x)\big) \land Bad(x) &\rightarrow \bot
\end{align*}
\end{minipage}};}
\end{scope}
\end{tikzpicture}
}

% %Rounded Rect [id:dp9443796803339821] 
% \draw   (170,249) .. controls (170,244.58) and (173.58,241) .. (178,241) -- (272,241) .. controls (276.42,241) and (280,244.58) .. (280,249) -- (280,273) .. controls (280,277.42) and (276.42,281) .. (272,281) -- (178,281) .. controls (173.58,281) and (170,277.42) .. (170,273) -- cycle ;
% %Rounded Rect [id:dp2773818458647491] 
% \draw   (170,148) .. controls (170,143.58) and (173.58,140) .. (178,140) -- (262,140) .. controls (266.42,140) and (270,143.58) .. (270,148) -- (270,172) .. controls (270,176.42) and (266.42,180) .. (262,180) -- (178,180) .. controls (173.58,180) and (170,176.42) .. (170,172) -- cycle ;
% % Refine rect
% \draw   (380,248) .. controls (380,243.58) and (383.58,240) .. (388,240) -- (472,240) .. controls (476.42,240) and (480,243.58) .. (480,248) -- (480,272) .. controls (480,276.42) and (476.42,280) .. (472,280) -- (388,280) .. controls (383.58,280) and (380,276.42) .. (380,272) -- cycle ;
% %Straight Lines [id:da932730372325787] 
% \draw    (380,261) -- (283,261) ;
% \draw [shift={(280,261)}, rotate = 360] [fill={rgb, 255:red, 0; green, 0; blue, 0 }  ][line width=0.08]  [draw opacity=0] (8.93,-4.29) -- (0,0) -- (8.93,4.29) -- cycle    ;
% %Straight Lines [id:da4961993079694478] 
% \draw    (220,240) -- (220,183) ;
% \draw [shift={(220,180)}, rotate = 90] [fill={rgb, 255:red, 0; green, 0; blue, 0 }  ][line width=0.08]  [draw opacity=0] (8.93,-4.29) -- (0,0) -- (8.93,4.29) -- cycle    ;
% %Straight Lines [id:da36870516429389966] 
% \draw    (170,260) -- (93,260) ;
% \draw [shift={(90,260)}, rotate = 360] [fill={rgb, 255:red, 0; green, 0; blue, 0 }  ][line width=0.08]  [draw opacity=0] (8.93,-4.29) -- (0,0) -- (8.93,4.29) -- cycle    ;
% %Straight Lines [id:da24013791401632556] 
% \draw    (170,160) -- (93,160) ;
% \draw [shift={(90,160)}, rotate = 360] [fill={rgb, 255:red, 0; green, 0; blue, 0 }  ][line width=0.08]  [draw opacity=0] (8.93,-4.29) -- (0,0) -- (8.93,4.29) -- cycle    ;



% \onslide<2->{
% %Rounded Rect [id:dp3414133323513948] 
% \draw   (380,148) .. controls (380,143.58) and (383.58,140) .. (388,140) -- (472,140) .. controls (476.42,140) and (480,143.58) .. (480,148) -- (480,172) .. controls (480,176.42) and (476.42,180) .. (472,180) -- (388,180) .. controls (383.58,180) and (380,176.42) .. (380,172) -- cycle ;
% %Straight Lines [id:da3745745784791844] 
% % Right Arrows
% \draw    (270,160) -- (377,160) ;
% \draw    (430,180) -- (430,237) ;
% \draw [shift={(380,160)}, rotate = 180] [fill={rgb, 255:red, 0; green, 0; blue, 0 }  ][line width=0.08]  [draw opacity=0] (8.93,-4.29) -- (0,0) -- (8.93,4.29) -- cycle    ;
% % Text Node
% \draw (370,197) node [anchor=north west][inner sep=0.75pt]   [align=center] {abstract\\cex};
% % Text Node
% \draw (379,150) node [anchor=north west][inner sep=0.75pt]   [align=center] {\textsc{RunOracle}};


% % Text Node
% \draw (588,160) node [anchor=north west][inner sep=0.75pt]  [align=center] {Black-Box Solver};
% %Rounded Rect [id:dp2146350233524532] 
% \draw   (580,152) .. controls (580,145.37) and (585.37,140) .. (592,140) -- (698,140) .. controls (704.63,140) and (710,145.37) .. (710,152) -- (710,188) .. controls (710,194.63) and (704.63,200) .. (698,200) -- (592,200) .. controls (585.37,200) and (580,194.63) .. (580,188) -- cycle ;

% \alt<4->{
% % predicates arrow
% \draw    (480,160) -- (577.13,170) ;
% \draw [shift={(580,170)}, rotate = 186.7] [fill={rgb, 255:red, 0; green, 0; blue, 0 }  ][line width=0.08]  [draw opacity=0] (8.93,-4.29) -- (0,0) -- (8.93,4.29) -- cycle    ;
% % abs cex arrow
% \draw    (600,200) -- (480,259.28) ;
% \draw [shift={(480,259)}, rotate = 336.17] [fill={rgb, 255:red, 0; green, 0; blue, 0 }  ][line width=0.08]  [draw opacity=0] (8.93,-4.29) -- (0,0) -- (8.93,4.29) -- cycle    ;
% % Text Node
% \draw (492,139) node [anchor=north west][inner sep=0.75pt]  [rotate=-6.23] [align=center] {\textcolor{focusGreen}{\textbf{predicates}}};
% % Text Node
% \draw (488,230) node [anchor=north west][inner sep=0.75pt]  [rotate=-334.67] [align=center] {\textcolor{focusRed}{\textbf{abstract cex}}};
% }{
% % predicates arrow
% \draw    (480,160) -- (577.13,170) ;
% \draw [shift={(580,170)}, rotate = 186.7] [fill={rgb, 255:red, 0; green, 0; blue, 0 }  ][line width=0.08]  [draw opacity=0] (8.93,-4.29) -- (0,0) -- (8.93,4.29) -- cycle    ;
% % abs cex arrow
% \draw    (600,200) -- (480,259.28) ;
% \draw [shift={(480,259)}, rotate = 336.17] [fill={rgb, 255:red, 0; green, 0; blue, 0 }  ][line width=0.08]  [draw opacity=0] (8.93,-4.29) -- (0,0) -- (8.93,4.29) -- cycle    ;
% % Text Node
% \draw (495,141.63) node [anchor=north west][inner sep=0.75pt]  [rotate=-6.23] [align=center] {predicates};
% % Text Node
% \draw (488,230) node [anchor=north west][inner sep=0.75pt]  [rotate=-334.67] [align=center] {abstract cex};
% }

% % SAFE arrow
% \draw    (650,200) -- (650,260) ;
% \draw [shift={(650,260)}, rotate = 270] [fill={rgb, 255:red, 0; green, 0; blue, 0 }  ][line width=0.08]  [draw opacity=0] (8.93,-4.29) -- (0,0) -- (8.93,4.29) -- cycle    ;
% % Text Node
% \draw (630,265) node [anchor=north west][inner sep=0.75pt]   [align=center] {$SAFE$};
% % Text Node
% \draw (655,220) node [anchor=north west][inner sep=0.75pt]   [align=center] {combined\\invariant};
% }



% \onslide<1>{\draw    (270,160) -- (430,160) -- (430,237) ;}
% \draw [shift={(430,240)}, rotate = 270] [fill={rgb, 255:red, 0; green, 0; blue, 0 }  ][line width=0.08]  [draw opacity=0] (8.93,-4.29) -- (0,0) -- (8.93,4.29) -- cycle    ;

% % Text Node
% \draw (400,253) node [anchor=north west][inner sep=0.75pt]   [align=center] {\textsc{Refine}};
% % Text Node
% \draw (170,253) node [anchor=north west][inner sep=0.75pt]   [align=center] {\textsc{ModelCheck}};
% % Text Node
% \draw (41,252) node [anchor=north west][inner sep=0.75pt]   [align=center] {$SAFE$};
% % Text Node
% \draw (175,151) node [anchor=north west][inner sep=0.75pt]   [align=center] {\textsc{IsFeasible}};
% % Text Node
% \draw (21,151) node [anchor=north west][inner sep=0.75pt]   [align=center] {$UNSAFE$};
% % Text Node
% \draw (280,140) node [anchor=north west][inner sep=0.75pt]   [align=center] {abstract cex};
% % Text Node
% \draw (114,141) node [anchor=north west][inner sep=0.75pt]   [align=center] {cex};
% % Text Node
% \draw (298,242) node [anchor=north west][inner sep=0.75pt]   [align=center] {predicates};
% % Text Node
% \draw (98,241) node [anchor=north west][inner sep=0.75pt]   [align=center] {predicates};
% % Text Node
% \draw (224,190) node [anchor=north west][inner sep=0.75pt]   [align=center] {abstract\\cex};



\newcommand{\invariantreprclasses}[5]
{
\begin{figure}[t]
\tikzset{every picture/.style={line width=0.75pt}} %set default line width to 0.75pt        
\centering
\begin{tikzpicture}[x=0.75pt,y=0.75pt,yscale=-0.9,xscale=0.9]
%uncomment if require: \path (0,300); %set diagram left start at 0, and has height of 300

%Flowchart: Terminator [id:dp473438182569073] 
\draw  [color={rgb, 255:red, 208; green, 2; blue, 27 }  ,draw opacity=1 ] (135.1,92.13) -- (328.9,92.13) .. controls (354.08,92.13) and (374.5,132.64) .. (374.5,182.63) .. controls (374.5,232.61) and (354.08,273.13) .. (328.9,273.13) -- (135.1,273.13) .. controls (109.92,273.13) and (89.5,232.61) .. (89.5,182.63) .. controls (89.5,132.64) and (109.92,92.13) .. (135.1,92.13) -- cycle ;
%Flowchart: Terminator [id:dp2032122365868334] 
\draw  [color={rgb, 255:red, 144; green, 19; blue, 254 }  ,draw opacity=1 ][dash pattern={on 4.5pt off 4.5pt}] (295.12,92.12) -- (423.38,92.12) .. controls (458.24,92.12) and (486.5,132.64) .. (486.5,182.63) .. controls (486.5,232.61) and (458.24,273.13) .. (423.38,273.13) -- (295.12,273.13) .. controls (260.26,273.13) and (232,232.61) .. (232,182.63) .. controls (232,132.64) and (260.26,92.12) .. (295.12,92.12) -- cycle ;
%Flowchart: Terminator [id:dp6321668924624688] 
\draw  [color={rgb, 255:red, 208; green, 2; blue, 27 }  ,draw opacity=1 ] (185.92,124.77) -- (287.58,124.77) .. controls (300.79,124.77) and (311.5,150.68) .. (311.5,182.63) .. controls (311.5,214.57) and (300.79,240.48) .. (287.58,240.48) -- (185.92,240.48) .. controls (172.71,240.48) and (162,214.57) .. (162,182.63) .. controls (162,150.68) and (172.71,124.77) .. (185.92,124.77) -- cycle ;
%Flowchart: Terminator [id:dp05234016317559864] 
% \draw  [color={rgb, 255:red, 144; green, 19; blue, 254 }  ,draw opacity=1 ][dash pattern={on 4.5pt off 4.5pt}] (273.41,124.77) -- (431.34,124.77) .. controls (451.86,124.77) and (468.5,150.68) .. (468.5,182.63) .. controls (468.5,214.57) and (451.86,240.48) .. (431.34,240.48) -- (273.41,240.48) .. controls (252.89,240.48) and (236.25,214.57) .. (236.25,182.63) .. controls (236.25,150.68) and (252.89,124.77) .. (273.41,124.77) -- cycle ;

% Text Node
\draw (120,250) node [anchor=north west][inner sep=0.75pt]   [align=left]
% {\hyperref[sec:sizeelem-def]{$\sizeelemclass$}};
{$\sizeelemclass$};
% Text Node
\draw (181,217) node [anchor=north west][inner sep=0.75pt]   [align=left]
% {\hyperref[defn:elemclass]{$\elemclass$}};
{$\elemclass$};
% Text Node
\draw (415,250) node [anchor=north west][inner sep=0.75pt]   [align=left]
% {\hyperref[defn:regelemclass]{$\regelemclass$}};
{$\regclass$};
% Text Node
% \draw (415,217) node [anchor=north west][inner sep=0.75pt]   [align=left]
% {\hyperref[defn:regclass]{$\regclass$}};
% {$\regclass$};
% Text Node
\draw (110,160) node [anchor=north west][inner sep=0.75pt]   [align=left] {\hyperref[exmpl:ltgt]{#1}};
\draw (185,160) node [anchor=north west][inner sep=0.75pt]   [align=left] {\hyperref[exmpl:diag]{#2}};
% \draw (250,160) node [anchor=north west][inner sep=0.75pt]   [align=left] {\hyperref[exmpl:incdec]{$IncDec$}};
% Text Node
\draw (325,160) node [anchor=north west][inner sep=0.75pt]   [align=left] {\hyperref[exmpl:even]{#3}};
% Text Node
\draw (395,160) node [anchor=north west][inner sep=0.75pt]   [align=left] {\hyperref[exmpl:evenleft]{#4}};
\draw (395,200) node [anchor=north west][inner sep=0.75pt]   [align=left] {#5};
% \draw (480,160) node [anchor=north west][inner sep=0.75pt]   [align=left] {\hyperref[exmp:evenodd]{$EvenOdd$}};
\end{tikzpicture}
    % \caption{Сравнение выразительных свойств трёх представлений инвариантов}
    % \label{fig:representations-new}
\end{figure}
}

\newcommand{\bubbles}[5]{%
\tikzset{every picture/.style={line width=0.75pt}} %set default line width to 0.75pt        
\begin{center}
\begin{tikzpicture}[x=0.75pt,y=0.75pt,yscale=-0.9,xscale=0.9]
%uncomment if require: \path (0,300); %set diagram left start at 0, and has height of 300

%Flowchart: Terminator [id:dp473438182569073] 
\draw  [color={rgb, 255:red, 208; green, 2; blue, 27 }  ,draw opacity=1 ] (135.1,92.13) -- (328.9,92.13) .. controls (354.08,92.13) and (374.5,132.64) .. (374.5,182.63) .. controls (374.5,232.61) and (354.08,273.13) .. (328.9,273.13) -- (135.1,273.13) .. controls (109.92,273.13) and (89.5,232.61) .. (89.5,182.63) .. controls (89.5,132.64) and (109.92,92.13) .. (135.1,92.13) -- cycle ;
%Flowchart: Terminator [id:dp2032122365868334] 
\draw  [color={rgb, 255:red, 144; green, 19; blue, 254 }  ,draw opacity=1 ][dash pattern={on 4.5pt off 4.5pt}] (295.12,92.12) -- (423.38,92.12) .. controls (458.24,92.12) and (486.5,132.64) .. (486.5,182.63) .. controls (486.5,232.61) and (458.24,273.13) .. (423.38,273.13) -- (295.12,273.13) .. controls (260.26,273.13) and (232,232.61) .. (232,182.63) .. controls (232,132.64) and (260.26,92.12) .. (295.12,92.12) -- cycle ;
%Flowchart: Terminator [id:dp6321668924624688] 
\draw  [color={rgb, 255:red, 208; green, 2; blue, 27 }  ,draw opacity=1 ] (185.92,124.77) -- (287.58,124.77) .. controls (300.79,124.77) and (311.5,150.68) .. (311.5,182.63) .. controls (311.5,214.57) and (300.79,240.48) .. (287.58,240.48) -- (185.92,240.48) .. controls (172.71,240.48) and (162,214.57) .. (162,182.63) .. controls (162,150.68) and (172.71,124.77) .. (185.92,124.77) -- cycle ;
%Flowchart: Terminator [id:dp05234016317559864] 
% \draw  [color={rgb, 255:red, 144; green, 19; blue, 254 }  ,draw opacity=1 ][dash pattern={on 4.5pt off 4.5pt}] (273.41,124.77) -- (431.34,124.77) .. controls (451.86,124.77) and (468.5,150.68) .. (468.5,182.63) .. controls (468.5,214.57) and (451.86,240.48) .. (431.34,240.48) -- (273.41,240.48) .. controls (252.89,240.48) and (236.25,214.57) .. (236.25,182.63) .. controls (236.25,150.68) and (252.89,124.77) .. (273.41,124.77) -- cycle ;

% Text Node
\draw (120,250) node [anchor=north west][inner sep=0.75pt]   [align=left]
% {\hyperref[sec:sizeelem-def]{$\sizeelemclass$}};
{$\sizeelemclass$};
% Text Node
\draw (181,217) node [anchor=north west][inner sep=0.75pt]   [align=left]
% {\hyperref[defn:elemclass]{$\elemclass$}};
{$\elemclass$};
% Text Node
\draw (415,250) node [anchor=north west][inner sep=0.75pt]   [align=left]
% {\hyperref[defn:regelemclass]{$\regelemclass$}};
{$\regclass$};
% Text Node
% \draw (415,217) node [anchor=north west][inner sep=0.75pt]   [align=left]
% {\hyperref[defn:regclass]{$\regclass$}};
% {$\regclass$};
% Text Node
\draw (105,160) node [anchor=north west][inner sep=0.75pt]   [align=left] {#1};
\draw (177,160) node [anchor=north west][inner sep=0.75pt]   [align=left] {#2};
\draw (250,160) node [anchor=north west][inner sep=0.75pt]   [align=left] {$IncDec$};
% Text Node
\draw (321,160) node [anchor=north west][inner sep=0.75pt]   [align=left] {#3};
% Text Node
\draw (395,160) node [anchor=north west][inner sep=0.75pt]   [align=left] {#4};
\draw (395,200) node [anchor=north west][inner sep=0.75pt]   [align=left] {$STLC\mbox{-}tc$};

% \draw (480,160) node [anchor=north west][inner sep=0.75pt]   [align=left] {\hyperref[exmp:evenodd]{$EvenOdd$}};
\end{tikzpicture}
\end{center}

\vspace*{7mm}
\begin{flushleft}\small
\begin{minipage}[l][][c]{.9\textwidth}
\elemclass{}: представимые формулами

\sizeelemclass{}: представимые формулами с ограничениями размера

\regclass{}: представимые автоматами над деревьями
\end{minipage}
\end{flushleft}
}

\newcommand{\exampleEven} % Even
{
\begin{center}
\begin{tikzpicture}[shorten >=1pt,node distance=2cm,on grid,auto,scale=0.8,every node/.style={scale=0.8}]
    \node[state,initial,accepting,initial text=$Z$] (s0) {$s_0$};
    \node[state] (s1) [right=of s0] {$s_1$};
    \path[->]
        (s0)    edge [bend left=25] node {$S$}       (s1)
        (s1)    edge [bend left=25] node {$S$}       (s0)
    ;
\end{tikzpicture}
\end{center}
}
\newcommand{\exampleEvenSecond} % Even
{
\begin{center}
\begin{tikzpicture}[shorten >=1pt,node distance=2cm,on grid,auto,scale=0.8,every node/.style={scale=0.8}]
    \node[state,initial,accepting,initial text=$Z$] (s0) {$0$};
    \node[state] (s1) [right=of s0] {$1$};
    \path[->]
        (s0)    edge [bend left=25] node {$S$}       (s1)
        (s1)    edge [bend left=25] node {$S$}       (s0)
    ;
\end{tikzpicture}
\end{center}
}

\newcommand{\exampleTreePumpingEmpty}{
\vspace{-0.4cm}
\begin{center}
\begin{tikzpicture}[shorten >=1pt,node distance=1cm,on grid,auto,scale=0.7,every node/.style={scale=0.7}]
    \node[state] (x) at (0:0) {$Node$};
    \node[state] (Lx) at (210:2.5) {$Node$};
    \node[state] (Rx) at (330:2.5) {$Node$};
    \node[state] (LLx) at (210:4.4) {$Leaf$};
    \node[state] (RLx) at (255:2.3) {$Leaf$};
    \node[state] (LRx) at (285:2.3) {$Leaf$};
    \node[state] (RRx) at (330:4.4) {$Leaf$};
    % \node[state,initial,accepting,initial text=$Z$] (s0) {$Node$};
    % \node[state] (s1) [right=of s0] {$1$};
    \path[->]
        (x)     edge  node {}     (Lx)
                edge  node {}     (Rx)
        (Lx)     edge  node {}     (LLx)
                edge  node {}     (RLx)
        (Rx)     edge  node {}     (LRx)
                edge  node {}     (RRx)
    ;
\end{tikzpicture}
\end{center}

% \begin{center}
% \tikzset{every picture/.style={line width=0.75pt}} %set default line width to 0.75pt        

% \begin{tikzpicture}[x=0.75pt,y=0.75pt,yscale=-1,xscale=1,scale=0.7,every node/.style={scale=0.7}]
% %uncomment if require: \path (0,235); %set diagram left start at 0, and has height of 235

% %Shape: Circle [id:dp0986190165294798] 
% \draw   (271,35) .. controls (271,21.19) and (282.19,10) .. (296,10) .. controls (309.81,10) and (321,21.19) .. (321,35) .. controls (321,48.81) and (309.81,60) .. (296,60) .. controls (282.19,60) and (271,48.81) .. (271,35) -- cycle ;
% %Shape: Circle [id:dp025477163194925878] 
% \draw   (182,98) .. controls (182,84.19) and (193.19,73) .. (207,73) .. controls (220.81,73) and (232,84.19) .. (232,98) .. controls (232,111.81) and (220.81,123) .. (207,123) .. controls (193.19,123) and (182,111.81) .. (182,98) -- cycle ;
% %Shape: Circle [id:dp81987576246913] 
% \draw   (346,101) .. controls (346,87.19) and (357.19,76) .. (371,76) .. controls (384.81,76) and (396,87.19) .. (396,101) .. controls (396,114.81) and (384.81,126) .. (371,126) .. controls (357.19,126) and (346,114.81) .. (346,101) -- cycle ;
% %Shape: Circle [id:dp19011749393276545] 
% \draw   (127,179) .. controls (127,165.19) and (138.19,154) .. (152,154) .. controls (165.81,154) and (177,165.19) .. (177,179) .. controls (177,192.81) and (165.81,204) .. (152,204) .. controls (138.19,204) and (127,192.81) .. (127,179) -- cycle ;
% %Shape: Circle [id:dp8570558031502189] 
% \draw   (220,178) .. controls (220,164.19) and (231.19,153) .. (245,153) .. controls (258.81,153) and (270,164.19) .. (270,178) .. controls (270,191.81) and (258.81,203) .. (245,203) .. controls (231.19,203) and (220,191.81) .. (220,178) -- cycle ;
% %Shape: Circle [id:dp12140489944808341] 
% \draw   (297,175) .. controls (297,161.19) and (308.19,150) .. (322,150) .. controls (335.81,150) and (347,161.19) .. (347,175) .. controls (347,188.81) and (335.81,200) .. (322,200) .. controls (308.19,200) and (297,188.81) .. (297,175) -- cycle ;
% %Shape: Circle [id:dp5747552646079859] 
% \draw   (402,175) .. controls (402,161.19) and (413.19,150) .. (427,150) .. controls (440.81,150) and (452,161.19) .. (452,175) .. controls (452,188.81) and (440.81,200) .. (427,200) .. controls (413.19,200) and (402,188.81) .. (402,175) -- cycle ;
% %Straight Lines [id:da21128174001155509] 
% \draw    (296,60) -- (207,73) ;
% %Straight Lines [id:da11015726266257353] 
% \draw    (296,60) -- (371,76) ;
% %Straight Lines [id:da42937490701386716] 
% \draw    (207,123) -- (152,154) ;
% %Straight Lines [id:da9026178337666466] 
% \draw    (371,126) -- (427,150) ;
% %Straight Lines [id:da23419680399445209] 
% \draw    (371,126) -- (322,150) ;
% %Straight Lines [id:da15771106808872581] 
% \draw    (207,123) -- (245,153) ;
% %Straight Lines [id:da02293499028974022] 
% \draw [color={rgb, 255:red, 208; green, 2; blue, 27 }  ,draw opacity=1 ]   (115,121.58) -- (129.55,147.96) ;
% \draw [shift={(131,150.58)}, rotate = 241.11] [fill={rgb, 255:red, 208; green, 2; blue, 27 }  ,fill opacity=1 ][line width=0.08]  [draw opacity=0] (14.29,-6.86) -- (0,0) -- (14.29,6.86) -- (9.49,0) -- cycle    ;
% %Straight Lines [id:da46749278538446104] 
% \draw [color={rgb, 255:red, 208; green, 2; blue, 27 }  ,draw opacity=1 ]   (291,120.58) -- (305.55,146.96) ;
% \draw [shift={(307,149.58)}, rotate = 241.11] [fill={rgb, 255:red, 208; green, 2; blue, 27 }  ,fill opacity=1 ][line width=0.08]  [draw opacity=0] (14.29,-6.86) -- (0,0) -- (14.29,6.86) -- (9.49,0) -- cycle    ;

% % Text Node
% \draw (279,26) node [anchor=north west][inner sep=0.75pt]   [align=left] {Node};
% % Text Node
% \draw (190,89) node [anchor=north west][inner sep=0.75pt]   [align=left] {Node};
% % Text Node
% \draw (354,92) node [anchor=north west][inner sep=0.75pt]   [align=left] {Node};
% % Text Node
% \draw (135,170) node [anchor=north west][inner sep=0.75pt]   [align=left] {Leaf};
% % Text Node
% \draw (228,169) node [anchor=north west][inner sep=0.75pt]   [align=left] {Leaf};
% % Text Node
% \draw (305,166) node [anchor=north west][inner sep=0.75pt]   [align=left] {Leaf};
% % Text Node
% \draw (410,166) node [anchor=north west][inner sep=0.75pt]   [align=left] {Leaf};
% % Text Node
% \draw (107,100) node [anchor=north west][inner sep=0.75pt]   [align=left] {p};
% % Text Node
% \draw (283,99) node [anchor=north west][inner sep=0.75pt]   [align=left] {P};

% \end{tikzpicture}
% \end{center}
}

\newcommand{\exampleTreePumpingPointer}{
\vspace{-0.4cm}
\begin{center}
\begin{tikzpicture}[shorten >=1pt,node distance=1cm,on grid,auto,scale=0.7,every node/.style={scale=0.7}]
    \node[state] (x) at (0:0) {$Node$};
    \node[state] (Lx) at (210:2.5) {$Node$};
    \node[state] (Rx) at (330:2.5) {$Node$};
    \node[state,accepting,label=above:\textcolor{red}{$p$}] (LLx) at (210:4.4) {$Leaf$};
    \node[state] (RLx) at (255:2.3) {$Leaf$};
    \node[state] (LRx) at (285:2.3) {$Leaf$};
    \node[state] (RRx) at (330:4.4) {$Leaf$};
    % \node[state,initial,accepting,initial text=$Z$] (s0) {$Node$};
    % \node[state] (s1) [right=of s0] {$1$};
    \path[->]
        (x)     edge  node {}     (Lx)
                edge  node {}     (Rx)
        (Lx)     edge  node {}     (LLx)
                edge  node {}     (RLx)
        (Rx)     edge  node {}     (LRx)
                edge  node {}     (RRx)
    ;
\end{tikzpicture}
\end{center}
}
\newcommand{\exampleTreePumpingPointerExtended}{
\vspace{-0.4cm}
\begin{center}
\begin{tikzpicture}[shorten >=1pt,node distance=1cm,on grid,auto,scale=0.7,every node/.style={scale=0.7}]
    \node[state] (x) at (0:0) {$Node$};
    \node[state] (Lx) at (210:2.5) {$Node$};
    \node[state] (Rx) at (330:2.5) {$Node$};
    \node[state,accepting,label=above:$p$] (LLx) at (210:4.4) {$Leaf$};
    \node[state] (RLx) at (255:2.3) {$Leaf$};
    \node[state,accepting,label=above:\textcolor{red}{$P$}] (LRx) at (285:2.3) {$Leaf$};
    \node[state] (RRx) at (330:4.4) {$Leaf$};
    % \node[state,initial,accepting,initial text=$Z$] (s0) {$Node$};
    % \node[state] (s1) [right=of s0] {$1$};
    \path[->]
        (x)     edge  node {}     (Lx)
                edge  node {}     (Rx)
        (Lx)     edge  node {}     (LLx)
                edge  node {}     (RLx)
        (Rx)     edge  node {}     (LRx)
                edge  node {}     (RRx)
    ;
\end{tikzpicture}
\end{center}
}
\newcommand{\exampleTreePumpingPointerPumped}{
\vspace{-0.4cm}
\begin{center}
\begin{tikzpicture}[shorten >=1pt,node distance=1cm,on grid,auto,scale=0.7,every node/.style={scale=0.7}]
    \node[state] (x) at (0:0) {$Node$};
    \node[state] (Lx) at (210:2.5) {$Node$};
    \node[state] (Rx) at (330:2.5) {$Node$};
   \node at (210:4.4) (LLx) {};
   \node[draw,minimum size=1.5cm,inner sep=0,regular polygon,regular polygon sides=3,label=above:$p$,below=0.5cm of LLx] (LLxTriangle) {$t$};
    \node[state] (RLx) at (255:2.3) {$Leaf$};
   \node at (285:2.3) (LRx) {};
   \node[draw,minimum size=1.5cm,inner sep=0,regular polygon,regular polygon sides=3,label=above:$P$,below=0.5cm of LRx] (LRxTriangle) {$t$};
    % \node[state,accepting,label=above:\textcolor{red}{$P$}] (LRx) at (285:2.3) {$Leaf$};
    \node[state] (RRx) at (330:4.4) {$Leaf$};
    % \node[state,initial,accepting,initial text=$Z$] (s0) {$Node$};
    % \node[state] (s1) [right=of s0] {$1$};
    \path[->]
        (x)     edge  node {}     (Lx)
                edge  node {}     (Rx)
        (Lx)     edge  node {}     (LLx)
                edge  node {}     (RLx)
        (Rx)     edge  node {}     (LRx)
                edge  node {}     (RRx)
    ;
\end{tikzpicture}
\end{center}
}

\newcommand{\softwareverificationFlip}[2]{\alt<2>{#1}{#2}}

\newcommand{\softwareverification}{
\begin{tikzpicture}[x=0.75pt,y=0.75pt,yscale=-0.8,xscale=0.9]
%uncomment if require: \path (0,431); %set diagram left start at 0, and has height of 431

%Flowchart: Data [id:dp8732756601700975] 
\draw   (235,160) -- (430,160) -- (385,220) -- (190,220) -- cycle ;

%Flowchart: Process [id:dp5292853718236602] 
\draw   (110,75) -- (510,75) -- (510,310) -- (110,310) -- cycle ;
%Flowchart: Data [id:dp7895813200465741] 
\draw   (266.25,20) -- (380,20) -- (353.75,60) -- (240,60) -- cycle ;

%Flowchart: Process [id:dp25379861655195624] 
\draw   (240,90) -- (380,90) -- (380,130) -- (240,130) -- cycle ;
%Straight Lines [id:da42254132520298326] 
\draw    (310,60) -- (310,88) ;
\draw [shift={(310,90)}, rotate = 270] [color={rgb, 255:red, 0; green, 0; blue, 0 }  ][line width=0.75]    (10.93,-3.29) .. controls (6.95,-1.4) and (3.31,-0.3) .. (0,0) .. controls (3.31,0.3) and (6.95,1.4) .. (10.93,3.29)   ;
%Straight Lines [id:da6767628941551396] 
\draw    (310,130) -- (310,158) ;
\draw [shift={(310,160)}, rotate = 270] [color={rgb, 255:red, 0; green, 0; blue, 0 }  ][line width=0.75]    (10.93,-3.29) .. controls (6.95,-1.4) and (3.31,-0.3) .. (0,0) .. controls (3.31,0.3) and (6.95,1.4) .. (10.93,3.29)   ;
%Straight Lines [id:da9132665796566681] 
\draw    (310,220) -- (310,248) ;
\draw [shift={(310,250)}, rotate = 270] [color={rgb, 255:red, 0; green, 0; blue, 0 }  ][line width=0.75]    (10.93,-3.29) .. controls (6.95,-1.4) and (3.31,-0.3) .. (0,0) .. controls (3.31,0.3) and (6.95,1.4) .. (10.93,3.29)   ;
%Straight Lines [id:da5965426352542076] 
\draw    (290,290) -- (241.56,328.75) ;
\draw [shift={(240,330)}, rotate = 321.34000000000003] [color={rgb, 255:red, 0; green, 0; blue, 0 }  ][line width=0.75]    (10.93,-3.29) .. controls (6.95,-1.4) and (3.31,-0.3) .. (0,0) .. controls (3.31,0.3) and (6.95,1.4) .. (10.93,3.29)   ;
%Straight Lines [id:da6349648032700808] 
\draw    (330,290) -- (378.44,328.75) ;
\draw [shift={(380,330)}, rotate = 218.66] [color={rgb, 255:red, 0; green, 0; blue, 0 }  ][line width=0.75]    (10.93,-3.29) .. controls (6.95,-1.4) and (3.31,-0.3) .. (0,0) .. controls (3.31,0.3) and (6.95,1.4) .. (10.93,3.29)   ;
%Flowchart: Process [id:dp6399060941112645] 
\draw   (220,250) -- (400,250) -- (400,290) -- (220,290) -- cycle ;


% Text Node
\draw (245,31) node [anchor=north west][inner sep=0.75pt,minimum width=3cm] [align=center] {Program};
% Text Node
\draw (235,170) node [anchor=north west][inner sep=0.75pt,minimum width=3cm] [align=center] {\softwareverificationFlip{Constrained Horn clauses}{Verification conditions\\(logical formula)}};
% Text Node
\draw (160,332) node [anchor=north west][inner sep=0.75pt] [align=left] {\textcolor{goodInvariantColor}{\softwareverificationFlip{Saturation / Finite model}{Safety proof}}};
% Text Node
\draw (345,332) node [anchor=north west][inner sep=0.75pt] {\textcolor{badInvariantColor}{\softwareverificationFlip{Refutation}{Counterexample trace}}};
% Text Node
\draw (121,82) node [anchor=north west][inner sep=0.75pt] [align=left] {\textbf{Verifier}};
% Text Node
\draw (255,262) node [anchor=north west][inner sep=0.75pt,minimum width=2cm] [align=center] {\softwareverificationFlip{\ourtool{}+Vampire}{Logical solver}};
% Text Node
\draw (260,102) node [anchor=north west][inner sep=0.75pt,minimum width=2cm] {Preprocessing};
\end{tikzpicture}
}

\newcommand{\evenChcSystem}{%
\begin{varblock}[0.7\textwidth]{}
\begin{align*}
x = Z &\rightarrow even(x)\\
even(y) \land x = S(S(y)) &\rightarrow even(x)\\
even(x) \land even(S(x)) &\rightarrow \bot
\end{align*}
\end{varblock}
}

\newcommand{\toolplot}[2]{%
% #1 is backend solver name
% #2 is a .csv file name
\pgfplotstableread[col sep = comma]{#2}\toolplotcsv
\begin{axis}[xmode=log, ymode=log, legend pos= north west, xlabel={\collab(#1)}, ylabel style = {align=center}, ylabel={{\color{red}\spacer{}} and {\color{blue}\ringen(#1)}}]
    \addplot[dashed,no marks,very thin] coordinates {(10,10) (1200000,1200000)};
    \addplot [dashed, no marks, thin] coordinates {(10,10) (1200000,1200000)};
    \addplot [dashed, no marks, thin] coordinates {(10,600000) (600000,600000)};
    \addplot [dashed, no marks, thin] coordinates {(600000, 10) (600000,600000)};
    \addplot [dashed, no marks, thin] coordinates {(10, 1200000) (1200000,1200000)};
    \addplot [dashed, no marks, thin] coordinates {(1200000, 10) (1200000,1200000)};
    
    % one second
    \addplot [no marks, thin] coordinates {(1000, 10) (1000,1000)};
    \addplot [no marks, thin] coordinates {(10, 1000) (1000,1000)};

    \addplot  [only marks,  mark=o, color=red,  mark size=3pt] table [x={Collab}, y={Z3}] {\toolplotcsv};
    \addplot  [only marks,  mark=triangle, color=blue, mark size=3pt] table [x={Collab}, y={RInGen}] {\toolplotcsv};
\end{axis}
}

\begin{frame}[plain]
\titlepage
\end{frame}

\begin{frame}
\frametitle{Содержание}
\tableofcontents[
% currentsection,
% currentsubsection,
subsectionstyle=show/show/hide
]
\end{frame}

% \AtBeginSection[]{
% % \begin{frame}{\currentname}
% \begin{frame}{Содержание}
% \tableofcontents[
% currentsection,
% currentsubsection,
% subsectionstyle=show/show/hide
% ]
% \end{frame}
% }

\whenFullCompile{\section{Обзор предметной области}% 3-5 слайдов, очень лайтово

\begin{frame}[fragile,t]{Верификация программ путём вывода индуктивных инвариантов}%
\vspace*{-4mm}
\begin{tikzpicture}[ampersand replacement=\&,
  column 1/.style={anchor=base west},
  column 2/.style={anchor=base east}]
\only<-6>{\node at (0, -0.3) {\begin{minipage}{5cm}
\begin{align*}%
    &\Alt<2->{\{x = 0 \land y = 0\}}{x, y := 0, 0}\\
    &\texttt{\textcolor{blue}{while }} * \texttt{ \textcolor{blue}{do}}\\
    &\qquad y := y + x\\
    &\qquad x := x + 1\\
    &\Alt<2->{\{y\geq 0\}}{\texttt{\textcolor{blue}{assert}} (y\geq 0)}
\end{align*}
\end{minipage}};}
\node at (0, -1.5) {};
\begin{scope}[rounded corners=3mm,opacity=0.95]
\onslide<3->{\draw[fill=blue!30] (-6.5,-2.5) rectangle +(8.5,.6) node[midway] {Как доказать корректность этой тройки Хоара?};}
\only<4->{\draw[fill=green!40] (-3.5,-3.2) rectangle +(11,.6) node[midway] {При помощи \emph{пользовательского} \textbf{индуктивного инварианта} $\phi$};}
\only<5->{\draw[fill=blue!30] (-6.5,-3.9) rectangle +(9,.6) node[midway] {\textbf{Пользователь}: $y\geq 0$~--- индуктивный инвариант?};}
\only<8->{\draw[fill=green!40] (-3.5,-4.6) rectangle +(11,.6) node[midway] {\textbf{SMT-решатель}: Нет, индуктивность нарушается при $x \mapsto -1$};}
\only<9->{\draw[fill=blue!30] (-6.5,-5.3) rectangle +(9.6,.6) node[midway] {\textbf{Пользователь}: А усиленная формула: $x \geq 0 \land y \geq 0$?};}
\only<10->{\draw[fill=green!40] (-5.0,-6.0) rectangle +(12.5,.6) node[midway] {\textbf{SMT-решатель}: Да, эта формула является индуктивным инвариантом};}
\end{scope}
\only<7-8>{
\begin{scope}[shift={(-8,-6)}]
\node at (5,0.5) (formula) {$VC$};
\node [rectangle, draw,
    text width=6em, text centered, rounded corners, minimum height=3em] at (8,0.5) (smt) {SMT-решатель};
\node at (12,1) (ok) {$\color{green}\checkmark$ (безопасно)};
\node at (12.55,0)  (bad) {$\color{red}\xmark$
% $\mathbin{\tikz [x=1.4ex,y=1.4ex,line width=.2ex, red] \draw (0,0) -- (1,1) (0,1) -- (1,0);}$
($\phi(\overline{x})$~--- не инд. инв.)};
%
\path [->] (formula) edge node {} (smt);
\path [->] (smt) edge node {} (ok);
\alt<8->{\path [->] [line width=0.4mm, red] (smt) edge node {} (bad);}{\path [->] (smt) edge node {} (bad);}
\end{scope}}
\end{tikzpicture}
\onslide<4->{
\begin{tikzpicture}[remember picture,overlay]
\node (bot) at (current page text area.south) {};
\node[above=-4mm of bot] {
% \btVFill
\begin{minipage}{10cm}
% \hspace*{-5mm}
\begin{align*}
\onslide<6-8>{VC := \left\{}
\begin{array}{rl}
    \onslide<6-8>{\forall x, y.\big(}x = 0 \land y = 0 &\rightarrow \Alt<6-8>{y\geq 0}{\phi(x, y)}\onslide<6-8>{\big)\land}\\
    \onslide<6-8>{\forall x, y, x', y'. \big(} \Alt<6-8>{y\geq 0}{\phi(x, y)}\land x' = x + 1 \land y' = y + x &\rightarrow \Alt<6-8>{y'\geq 0}{\phi(x', y')}\onslide<6-8>{\big)\land}\\
    \onslide<6-8>{\forall x, y.\big(}\Alt<6-8>{y\geq 0}{\phi(x, y)} &\rightarrow y \geq 0\onslide<6-8>{\big)}
\end{array}
\onslide<6-8>{\right.}%
\end{align*}
% }
\end{minipage}
};
\end{tikzpicture}
}
\end{frame}

\begin{frame}{Дизъюнкты Хорна с ограничениями}
\begin{tikzpicture}
\onslide<1->{\node[draw,fill=blue!30,opacity=0.95,rounded rectangle,minimum height=6mm] at (0, 0) {\textbf{Как автоматизировать} вывод индуктивных инвариантов?};}
\onslide<2->{\node[draw,fill=green!40,opacity=0.95,rounded rectangle] at (2, -0.7) {Заменить пользовательскую формулу на \textbf{неинтерпретированный} символ $I$};}
\begin{scope}[shift={(-1,+2)}]
\node (chcs) at (2, -5) {\begin{minipage}{10cm}
\begin{align*}
x = 0 \land y = 0 &\rightarrow I(x, y)\\
I(x, y)\land x' = x + 1 \land y' = y + x &\rightarrow I(x', y')\\
I(x, y) &\rightarrow y \geq 0
\end{align*}
\end{minipage}};
\onslide<3->{
\draw[draw=nicegreen!70,ultra thick] (-0.15, -4.2) rectangle ++(4, -1.1);
\calloutquote[width=30mm,position={(-2.0,-0.5)},bubblePosition={(5.1, -3.4)},callout pointer width=6mm,fill=nicegreen!70,rounded corners]{\Large с\Alt<2>{О}{}ограничениями}
\calloutquote[width=30mm,position={(2.2,-0.75)},bubblePosition={(.9, -3.4)},callout pointer width=6mm,fill=red!60,rounded corners]{\Large Дизъюнкты Хорна}
}
\end{scope}
\end{tikzpicture}
\end{frame}

\begin{frame}{Дизъюнкты Хорна формально}
\begin{block}{}
\attention{Дизъюнкт Хорна} $C$~--- это формула первого порядка следующего вида:
\vspace*{10pt}
\begin{align*}
\phi\land P_1(\overline{x}_1)\land\ldots\land P_n(\overline{x}_n) \rightarrow H
\end{align*}
\vspace*{-20pt}
\begin{itemize}
    \item \attention{ограничение} $\phi$~--- это формула теории
    \item \attention{голова} $H$~--- это либо ложь $\bot$, либо атом $P(\overline{x})$
    \item $P_1,\ldots,P_n, P$~--- это неинтерпретированные символы
    \item все переменные (неявно) универсально квантифицированы
\end{itemize}
\attention{Система дизъюнктов Хорна}~--- это конъюнкция дизъюнктов Хорна
\attention{Хорн-решатель}~--- программа, проверяющая выполнимость системы дизъюнктов
\end{block}
\end{frame}

\begin{frame}{Применения Хорн-решателей}
\vspace*{3mm}\hspace*{-3mm}\begin{tikzpicture}[every node/.append style={draw,rounded rectangle,minimum height=1.cm,minimum width=2cm,align=center,opacity=0.95,scale=.8}, X/.tip={latex}]
    \node[fill=green!40,minimum height=1.2cm] (solver) {Хорн-решатель};
    \node[above= 30mm of solver] (smart) {Верификация самоисполняющихся контрактов$^3$};
    \node[above left= 30mm of solver] (refTypes) {Вывод ограниченных типов$^2$};
    \node[below= 10mm of refTypes] (loops) {Вывод инвариантов циклов$^1$};
    \node[above right= 30mm of solver] (conc) {Верификация параллельных программ$^4$};
    \node[below= 10mm of conc] (hyp) {Верификация гипербезопастности$^5$};
    \draw[-X] (refTypes) -- (solver);
    \draw[-X] (loops) -- (solver);
    \draw[-X] (hyp) -- (solver);
    \draw[-X] (conc) -- (solver);
    \draw[-X] (smart) -- (solver);
\end{tikzpicture}
\blfootnote{$^1$ Gurfinkel и др. The SeaHorn Verification Framework. CAV'15}
\blfootnote{$^2$ Tan и др. SolType: refinement types for arithmetic overflow in solidity. POPL'22}
\blfootnote{$^3$ Alt и др. SolCMC: Solidity Compiler’s Model Checker. CAV'22}
\blfootnote{$^4$ Hoenicke и др. Thread Modularity at Many Levels. POPL'17}
\blfootnote{$^5$ Shemer и др. Property Directed Self Composition. CAV'19}
\end{frame}

\begin{frame}[fragile]{Дизъюнкты Хорна над алгебраическими типами данных (АТД)}
\begin{exampleblock}{Пример программы на языке \textsc{Haskell}:}
% \begin{minipage}[t][][t]{\textwidth}
\begin{minted}[tabsize=4,escapeinside=@@]{haskell}
data Nat = Z | S Nat
data List = nil | cons Nat List
drop Z xs = xs
drop _ nil = nil
drop (S n) (cons(_, xs)) = drop n xs
assert (@$\neg\exists$@ n xs . xs /= nil && drop n xs == drop (S n) xs)
\end{minted}
% \end{minipage}
\end{exampleblock}
\begin{exampleblock}{Условия верификации в виде дизъюнктов Хорна над АТД:}
\begin{align*}
\top &\rightarrow drop(Z, xs, xs)\\
\top &\rightarrow drop(S(n), nil, nil)\\
drop(n, xs, rs) &\rightarrow drop(S(n), cons(x, xs), rs)\\
\neg (xs = nil) \land drop(n, xs, ys) \land drop(S(n), xs, ys) &\rightarrow \bot
\end{align*}
\end{exampleblock}
\end{frame}

% \section{Constrained Horn Clauses}
% \begin{frame}[fragile]{SMT-based program verification}
% \vspace*{-1cm}
% \focus{How to verify such programs?}
% \vspace*{5mm}
% \begin{lstlisting}
% assume (crc != 0 && buf.size() > n);
% for (size_t i = 0; i < buf.size(); i++) {
%   crc ^= buf[i];
%   for (int k = 0; k < 8; k++)
%     crc = crc & 1 ? (crc >> 1) ^ 0x42f0e1eba9ea3693 : crc >> 1;
% }
% assert (crc != 0);
% \end{lstlisting}
% \onslide<2->{
% \begin{tikzpicture}[overlay,remember picture]
% \draw[draw=red!70,ultra thick] (0.9, 1.4) rectangle ++(13, 0.5);
% \draw[draw=red!70,ultra thick] (2.0, 2.25) rectangle ++(1.5, 0.5);
% \draw[draw=red!70,ultra thick] (5, 2.65) rectangle ++(2.5, 0.5);
% \calloutquote[position={(-0.5,-1.9)},bubblePosition={(10.5,4)},callout pointer width=.4cm,fill=red!70,rounded corners]{Bit vector operations}
% \calloutquote[position={(0.5,-1.)},bubblePosition={(2,4)},callout pointer width=.4cm,fill=red!70,rounded corners]{Array operations}
% \calloutquote[position={(0.5,-0.6)},bubblePosition={(6,4)},callout pointer width=.4cm,fill=red!70,rounded corners]{String operations}
% \node[overlay,draw,fill=green!40,opacity=0.95,rounded rectangle,minimum height=1.5cm,minimum width=12cm,rotate=0,align=center] at (7,0.2) {{\huge SMT-solvers successfully handle such theories}\\{\onslide<3>{\Large But how to employ them to \textbf{verify} programs with \textbf{loops}?}}};
% \end{tikzpicture}
% }
% \end{frame}

% \begin{frame}[fragile,t]{Program verification via loop inductive invariants}%
% % \vspace*{-4mm}
% \begin{tikzpicture}[ampersand replacement=\&,
%   column 1/.style={anchor=base west},
%   column 2/.style={anchor=base east}]
% \only<-6>{\node at (0, -0.3) {\begin{minipage}{5cm}
% \begin{align*}%
%     &\Alt<2->{\{x = 0 \land y = 0\}}{x, y := 0, 0}\\
%     &\texttt{\textcolor{blue}{while }} * \texttt{ \textcolor{blue}{do}}\\
%     &\qquad y := y + x\\
%     &\qquad x := x + 1\\
%     &\Alt<2->{\{y\geq 0\}}{\texttt{\textcolor{blue}{assert}} (y\geq 0)}
% \end{align*}
% \end{minipage}};}
% \begin{scope}[rounded corners=3mm,opacity=0.95]
% \onslide<3->{\draw[fill=blue!30] (-6.5,-2.5) rectangle +(8.5,.6) node[midway] {How to prove the correctness of this Hoare triple?};}
% \only<4->{\draw[fill=green!40] (-2.5,-3.2) rectangle +(9.8,.6) node[midway] {By induction, using the \emph{user-specified} \textbf{inductive invariant}};}
% \only<5->{\draw[fill=blue!30] (-6.5,-3.9) rectangle +(6.7,.6) node[midway] {\textbf{User}: Is $y\geq 0$ an inductive invariant?};}
% \only<8->{\draw[fill=green!40] (-1.,-4.6) rectangle +(8.3,.6) node[midway] {\textbf{SMT solver}: No, it can be violated for $x \mapsto -1$};}
% \only<9->{\draw[fill=blue!30] (-6.5,-5.3) rectangle +(9.8,.6) node[midway] {\textbf{User}: I see\dots What if I strengthen it like: $x \geq 0 \land y \geq 0$?};}
% \only<10->{\draw[fill=green!40] (-3.0,-6.0) rectangle +(10.3,.6) node[midway] {\textbf{SMT solver}: Yes! Your formula is a valid inductive invariant};}
% \end{scope}
% \only<7-8>{
% \begin{scope}[shift={(-8,-6)}]
% \node at (5,0.5) (formula) {$VC$};
% \node [rectangle, draw,
%     text width=6em, text centered, rounded corners, minimum height=3em] at (8,0.5) (smt) {SMT solver};
% \node at (12,1) (ok) {$\color{green}\checkmark$ (safe)};
% \node at (12.55,0)  (bad) {$\color{red}\xmark$
% % $\mathbin{\tikz [x=1.4ex,y=1.4ex,line width=.2ex, red] \draw (0,0) -- (1,1) (0,1) -- (1,0);}$
% ($\phi(\overline{x})$ is bad)};
% %
% \path [->] (formula) edge node {} (smt);
% \path [->] (smt) edge node {} (ok);
% \alt<8->{\path [->] [line width=0.4mm, red] (smt) edge node {} (bad);}{\path [->] (smt) edge node {} (bad);}
% \end{scope}}
% % \onslide<7-8>{
% % \draw[draw=red!70,ultra thick] (-1.5, -4.3) rectangle ++(7.8, 0.6);
% % \calloutquote[width=4.1cm,position={(1,-2.08)},bubblePosition={(-2,-1.4)},callout pointer width=.4cm,fill=red!70,rounded corners]{Can be violated for $x < 0$}
% % }
% % \onslide<8>{\node[draw,fill=green!55,text width=5cm] at (-2,-2.5) {Should be \textbf{strengthened}, e.g.,$$I(x,y)\equiv x \geq 0 \land y \geq 0$$};}
% \end{tikzpicture}
% % \begin{tikzpicture}[remember picture,overlay]
% % \node at (0, -5) {
% \only<4->{
% \btVFill
% % \begin{minipage}{10cm}
% \hspace*{-5mm}\begin{align*}
% \onslide<6-8>{VC := \left\{}
% \begin{array}{rl}
%     \onslide<6-8>{\forall x, y.\big(}x = 0 \land y = 0 &\rightarrow \Alt<6-8>{y\geq 0}{\phi(x, y)}\onslide<6-8>{\big)\land}\\
%     \onslide<6-8>{\forall x, y, x', y'. \big(} \Alt<6-8>{y\geq 0}{\phi(x, y)}\land x' = x + 1 \land y' = y + x &\rightarrow \Alt<6-8>{y'\geq 0}{\phi(x', y')}\onslide<6-8>{\big)\land}\\
%     \onslide<6-8>{\forall x, y.\big(}\Alt<6-8>{y\geq 0}{\phi(x, y)} &\rightarrow y \geq 0\onslide<6-8>{\big)}
% \end{array}
% \onslide<6-8>{\right.}%
% \end{align*}
% }
% % \end{minipage}
% % };
% % \end{tikzpicture}
% \end{frame}

% \begin{frame}{Constrained Horn Clauses}
% \begin{tikzpicture}
% \onslide<1->{\node[draw,fill=blue!30,opacity=0.95,rounded rectangle,minimum height=6mm] at (0, 0) {\textbf{How to automate} an inference of inductive invariants?};}
% \onslide<2->{\node[draw,fill=green!40,opacity=0.95,rounded rectangle] at (3.7, -0.7) {Instead of user-specified formula introduce \textbf{uninterpreted} symbol $I$};}
% \begin{scope}[shift={(-1,+0.3)}]
% \node (chcs) at (2, -3) {\begin{minipage}{10cm}
% \begin{align*}
% x = 0 \land y = 0 &\rightarrow I(x, y)\\
% I(x, y)\land x' = x + 1 \land y' = y + x &\rightarrow I(x', y')\\
% I(x, y) &\rightarrow y \geq 0
% \end{align*}
% \end{minipage}};
% \onslide<3->{
% \draw[draw=nicegreen!70,ultra thick] (-0.15, -2.2) rectangle ++(4, -1.1);
% \calloutquote[width=2.4cm,position={(0.6,-0.25)},bubblePosition={(2,-1.7)},callout pointer width=.4cm,fill=nicegreen!70,rounded corners]{\Large Constrained}
% \calloutquote[width=2.6cm,position={(-0.55,-0.5)},bubblePosition={(4.8,-1.7)},callout pointer width=.2cm,fill=red!60,rounded corners]{\Large Horn clauses}
% }
% \end{scope}
% \end{tikzpicture}
% \end{frame}

\begin{frame}[t]{Индуктивный инвариант}
Пусть $\structur$~--- модель теории АТД, $\prog$~--- система дизъюнктов Хорна.

\attention{Индуктивный инвариант} $\invariant$~--- расширение модели $\invariant=\tuple{\structur, \relations}$, такое что $\invariant\models\prog$.

\pause
\vspace*{-10pt}\begin{align*}
x = Z \land y = S(Z) &\rightarrow inc(x, y)\\
x' = S(x) \land y' = S(y) \land inc(x, y) &\rightarrow inc(x', y')\\
x = y \land inc(x, y) &\rightarrow \bot
\end{align*}
\begin{tikzpicture}[remember picture,overlay,every node/.append style={draw,rounded rectangle,align=center,opacity=0.95}]
\begin{scope}[shift={(+3,-1.8)}]
% \onslide<2>{\node[fill=green!40] at (4,-0.3) {\LARGE Индуктивные инварианты составляют решётку};}
\onslide<3->{\node[fill=red!40] at (1,-0.3) {\large Как представлять эти \textbf{бесконечные} множества?};}
\onslide<4->{
  \node[fill=darkyellow!40] at (5,-1.3) {\large Инварианты обычно представляются формулами теории\\Они задают т.н. \emph{класс элементарных инвариантов}
  };
  \begin{scope}[draw,darkyellow!40,ultra thick]
  \node[minimum width=16mm,rectangle,opacity=1., minimum height=6mm] at (5.85, 2.0) {};
  \node[minimum width=16mm,rectangle,opacity=1., minimum height=6mm] at (5.85, 1.2) {};
  \end{scope}
}
% \onslide<5>{\node[fill=green!40,minimum height=1.4cm] at (4,-2.1) {\Large $\Rightarrow$ CHC solving is concerned with building \textbf{infinite} models \textbf{automatically}};}
\end{scope}
\end{tikzpicture}
\vspace*{-6mm}\begin{align*}
    \invariant_1 &= \structur\big\{ inc\mapsto \Alt<4->{}{\{ (x, y) \mid}\ y = S(x) \big\}\\
    \invariant_2 &= \structur\big\{ inc\mapsto \Alt<4->{}{\{ (x, y) \mid}\ \Alt<4->{\neg(x = y)}{x\neq y} \big\}\\
    \invariant_3 &= \ldots
\end{align*}
\end{frame}

\begin{frame}[t]{Проблема выразимости класса элементарных инвариантов}
\vspace*{-6mm}\hspace*{-20mm}\begin{tikzpicture}
\node[text width=.9\textwidth] (chcSystem) at (-1, 0) {\evenChcSystem{}};
\onslide<2->{\node (unsafe) at (2.2,-2.7) {\beamermessage{.17\textwidth}{Невыполнима}};}
\onslide<2->{\draw[line width=1.2pt, -{Latex}] (chcSystem) -- (unsafe) ;}

\onslide<3->{\node (safe) at (-4.2,-2.7) {\beamermessage{.17\textwidth}{Выполнима}};}
\onslide<3->{\draw[line width=1.2pt, -{Latex}] (chcSystem) -- (safe) ;}

\onslide<4->{\node[text width=.5\textwidth] (invariant) at (-6.5, -4.7) {\begin{varexampleblock}[.8\textwidth]{\centering Инвариант выразим элементарно}\centering $\invariant(even) \equiv \phi$\end{varexampleblock}};}
\onslide<4->{\draw[line width=1.2pt, -{Latex}] (safe) -- (-5.5,-4) ;}

\onslide<5->{\node[text width=.5\textwidth] (nothing) at (-1.2, -4.7) {%
\begin{varalertblock}[.8\textwidth]{\centering Инвариант невыразим элементарно}
\onslide<5->{\centering\LARGE\textbf{???}}
\end{varalertblock}};}
\onslide<5->{\draw[line width=1.2pt, -{Latex}] (safe) -- (-2.5,-4) ;}

% \onslide<6-7>{\calloutquote[width=80mm,position={(0,-1.2)},bubblePosition={(-1,-3)},callout pointer width=.3cm,fill=yellow!80,rounded corners]{\begin{minipage}{80mm}
% \centering Язык АТД первого порядка:
% $$\phi \Coloneqq t = t' \mid \neg\psi \mid \psi \land \psi' \mid \psi \lor \psi' \mid \forall x\,.\,\psi \mid \exists x\,.\,\psi$$
% \end{minipage}}}

% \onslide<7>{
% \draw[red!80,line width=0.6mm] (0.7,-3.25) -- (3,-3.25) node [midway] (qeLineCenter) {};
% \calloutquote[width=80mm,position={(0.6,0.6)},bubblePosition={(0.3,-4.1)},callout pointer width=.3cm,fill=red!80,rounded corners]{Теория АТД допускает устранение кванторов}}

\onslide<6->{\node[overlay,draw,fill=red!40,opacity=0.95,rounded rectangle,minimum height=2cm,minimum width=15cm,align=center] at (-1.2,-2) {\alt<7>{\Large \textbf{Проблема:} класс элементарных инвариантов невыразителен}{{\huge Вывод инвариантов в языке АТД \textbf{расходится}!}\\{например, \textsc{Z3/Spacer} расходится}}};}
% \onslide<11>{\node[overlay,draw,fill=blue!30,opacity=0.95,rounded rectangle,minimum height=2cm,minimum width=15cm,rotate=0,align=center] at (-1,-2) {\huge Can we infer not FOL-based inductive invariants\\\huge by employing automated theorem provers?};}
\end{tikzpicture}
\end{frame}}


\begin{framesection}{Постановка задачи}
\textbf{Цель работы}~--- предложение новых классов индуктивных инвариантов для программ с АТД и создание для них методов автоматического вывода. \textbf{Задачи:}

\begin{enumerate}
\item Предложить методы вывода инвариантов в существующих классах
\item Предложить новый класс индуктивных инвариантов программ с АТД
\item Предложить метод автоматического вывода инвариантов в новом классе
\item Выполнить пилотную программную реализацию предложенных методов
\item Провести экспериментальное сопоставление реализованного инструмента с существующими на представительном тестовом наборе
\end{enumerate}
\end{framesection}

\begin{framesection}{Результаты}
\begin{enumerate}
\item Предложен метод вывода регулярных инвариантов при помощи поиска конечных моделей
\item Предложен метод вывода синхронных регулярных инвариантов при помощи поиска конечных моделей
\item Предложен класс инвариантов, основанный на булевой комбинации элементарных и регулярных инвариантов\\Также предложен метод совместного вывода инвариантов в этом классе посредством вывода инвариантов в подклассах
\item Проведено теоретическое сравнение рассмотренных классов инвариантов\\Доказаны леммы о <<накачке>> для элементарных инвариантов
\item Выполнена пилотная реализация предложенных методов на языке \fsharp{} в рамках инструмента \theringen{}\\Разработанный инструмент решил из бенчмарка <<Tons of Inductive Problems>> в 3.74 раза больше задач, чем наилучший из существующих инструментов
\end{enumerate}
\end{framesection}

\section{Результаты}

% \newcommand{\evenCHCSystem}{\begin{varexampleblock}[0.75\textwidth]{CHC system \onslide<2->{as a FOL-formula}}
%     \begin{align*}
%     \onslide<2->{\forall x. (}x = Z &\rightarrow even(x)\onslide<2->{)\land} \\
%     \onslide<2->{\forall x, y. (}x = S(S(y)) \land even(y) &\rightarrow even(x)\onslide<2->{)\land} \\
%     \onslide<2->{\forall x, y. (}even(x) \land even(y) \land y = S(x) &\rightarrow \bot\onslide<2->{)}
%     \end{align*}
% \end{varexampleblock}}
\newcommand{\evenCHCSystem}{\begin{varexampleblock}[0.72\textwidth]{Система дизъюнктов \onslide<2->{как формула ЛПП}}
    \begin{align*}
    \top&\rightarrow even(Z)\onslide<2->{)\land} \\
    \onslide<2->{\forall y. (}even(y) &\rightarrow even(S(S(y)))\onslide<2->{)\land} \\
    \onslide<2->{\forall x. (}even(x) \land even(S(x)) &\rightarrow \bot\onslide<2->{)}
    \end{align*}
\end{varexampleblock}}

\newcount\rPos
\newcount\modelFloat

\begin{frame}{Вывод регулярных инвариантов при помощи поиска конечных моделей}
\animate<3-43>
\transduration<3-43>{0.02}
\animatevalue<3-23>{\rPos}{20}{0}
\pgfmathsetmacro\rPosFloat{0.1*\rPos}     % [10;0] ~> [3.5;-1.5]
\pgfmathsetmacro\chcSystemSize{(\rPos+20) / 40} % [10;0] ~> [1.0;.5]
\pgfmathsetmacro\chcSystemOpacity{(\rPos) / 20} % [10;0] ~> [1.0;.5]
\animatevalue<24-44>{\modelFloat}{0}{20}
\pgfmathsetmacro\modelPos{0.05*\modelFloat}
\pgfmathsetmacro\modelSize{(\modelFloat+20) / 40}
\pgfmathsetmacro\modelOpacity{(\modelFloat) / 20}
\centering
\begin{tikzpicture}[remember picture, overlay]
\node[text width=\textwidth,scale=\chcSystemSize,opacity=\chcSystemOpacity] (evenCHCs) at (-0.32,\rPosFloat) {\evenCHCSystem{}};
\onslide<1>{\calloutquote[width=7.8cm,position={(-1,0.7)},bubblePosition={(0,-0.5)},callout pointer width=.4cm,fill=yellow!60,rounded corners]{\Large АТД ограничения устранены}}
\node<3->[inner sep=0pt] (fmfinder) at (0.3,-2.2) {\includegraphics[width=.52\textwidth,height=.28\textwidth]{resources/blender.png}};
\node<3-> (fmfinderName) [above left=-39mm and -58mm of fmfinder,align=center,text width=50mm] {{\large Инструмент поиска конечных моделей}};
\node<3-> (modelStart) [below right=-1.5cm and -4.6cm of fmfinder] {};
\node<3-> (leftpath) [above left=2cm and 0.1cm of fmfinder] {};
\node<3-> (rightpath) [above right=2cm and -0.5cm of fmfinder] {};
\path<3-> (modelStart) |- (leftpath) node [text width=\textwidth,scale=\modelSize,opacity=\modelOpacity,pos=\modelPos] (model) {\begin{varblock}[0.5\textwidth]{}
\begin{align*}
    \mathattention<45>{\domain{Nat}}&\mathattention<45>{=\{0,1\}}\\
    \mathattention<46>{\mystructur(Z)}&\mathattention<46>{=0}\\
    \mathattention<47>{\mystructur(S)(x)}&\mathattention<47>{=1-x}\\
    \mathattention<48>{\mystructur(even)}&\mathattention<48>{=\{0\}}
\end{align*}
\end{varblock}};
\onslide<45->{\node<45-> (automaton) at (rightpath) {
\begin{minipage}{.3\textwidth}
\begin{varblock}[\textwidth]{}
\begin{center}
\vphantom{\begin{tikzpicture}[shorten >=1pt,node distance=2cm,on grid,auto]
    \node[state,initial,initial text=$Z$,accepting] (s0) {$0$};
    \node[state] (s1) [right=of s0] {$1$};
    \path[->]
        (s0)    edge [bend left=25] node {$S$}       (s1)
        (s1)    edge [bend left=25] node {$S$}       (s0)
    ;
\end{tikzpicture}}\hspace*{-12mm}\raisebox{9mm}{
\begin{tikzpicture}[remember picture,overlay,shorten >=1pt,node distance=2cm,on grid,auto]
    \node[state,onslide=<46->{initial,initial text=$Z$},onslide=<48->{accepting}] (s0) {$0$};
    \node[state] (s1) [right=of s0] {$1$};
    \path<47->[onslide=<47->,->]
        (s0)    edge [bend left=25] node {$S$}       (s1)
        (s1)    edge [bend left=25] node {$S$}       (s0)
    ;
    \onslide<49>{\node[below=12mm of s0] {$\invariant(even) = \langOf{\automat}$};}
\end{tikzpicture}}
\end{center}
\end{varblock}
\end{minipage}};}
\end{tikzpicture}
\end{frame}

\begin{frame}{\textbf{Результат 2.} Предложен метод вывода синхронных регулярных инвариантов при помощи поиска конечных моделей и доказана его корректность}
\only<2>{\framesubtitle{\textbf{Этап 1.} Устранить АТД ограничения при помощи унификации и введения новых дизъюнктов}}
\only<3>{\framesubtitle{\textbf{Этап 2.} Построить декларативное описание синхронного автомата, выражающего инвариант системы}}
\only<4>{\framesubtitle{\textbf{Этап 3.} Передать формулу в сторонний инструмент поиска конечных моделей}}
\only<5>{\framesubtitle{\textbf{Этап 4.} По конечной модели построить синхронный автомат над деревьями}}
\vspace*{-4pt}
\begin{columns}
\begin{column}{0.49\textwidth}
\vspace*{-8pt}
\begin{exampleblock}{Регулярные языки не позволяют представлять синхронные отношения}
\small\vspace*{-10pt}
\begin{align*}
    \top &\rightarrow lt(Z, S(x))\\
    lt(x, y) &\rightarrow lt(S(x), S(y))\\
    lt(x, y) \land lt(y, x) &\rightarrow \bot
\end{align*}
\end{exampleblock}
\end{column}
\begin{column}{0.48\textwidth}
\onslide<3->{
\begin{exampleblock}{Декларативное описание синхронного автомата}
\small\vspace*{-10pt}
\begin{align*}
&R(q) \rightarrow R(p(d(f,g, q), d(f, g, q)))\\
&R(p(q_1, q_2)) \rightarrow \big(F(q_1) \rightarrow F(d(S, S, q_2))\big)\\
&\dots
\end{align*}
\end{exampleblock}}
\end{column}
\end{columns}
\onslide<5->{
\vspace*{-15pt}
\begin{varexampleblock}[\textwidth]{Синхронный автомат над деревьями, выражающий инвариант: $$A = \sautomaton{0, 1}{2}{1}$$}
\small\vspace*{-20pt}
\begin{align*}
\tuple{Z, Z} &\mapsto 0 &Z &\mapsto 0 &S(q) &\mapsto 0\\
\tuple{Z, S}(q) &\mapsto 1 &\tuple{S, Z}(q) &\mapsto 0 &\tuple{S, S}(q) &\mapsto q
\end{align*}
$$\langOf{A} = \left\{ \tuple{S^n(Z), S^m(Z)} \mid n < m \right\}$$
\end{varexampleblock}
}
\end{frame}

\begin{frame}{\textbf{Результат 3.} Предложен новый класс инвариантов, основанный на булевой комбинации элементарных и регулярных инвариантов}
Новый класс \emph{комбинированных инвариантов} представляется формулами вида:
$$\phi \Coloneqq \overline{t}\in \langOf{A} \mid t = t' \mid \neg\psi \mid \psi \land \psi' \mid \psi \lor \psi'$$
\vspace*{-8mm}
\begin{itemize}
    \item $\overline{t}\in \langOf{A}$~--- принадлежность кортежа термов регулярном языку автомата $A$
\end{itemize}
% \begin{tikzpicture}
% \node[draw,fill=green!40,opacity=0.95,rounded rectangle,minimum height=2cm,align=center] at (-1,0) {\begin{minipage}{80mm}$$\phi \Coloneqq \overline{x}\in \langOf{A} \mid t = t' \mid \neg\psi \mid \psi \land \psi' \mid \psi \lor \psi'$$\end{minipage}};
% \end{tikzpicture}
\end{frame}

\begin{frame}{\textbf{Результат 4.} Предложен метод совместного вывода инвариантов в классе комбинированных инвариантов посредством вывода инвариантов в подклассах и доказана его корректность}
\whenFullCompile{\centering\ciciPic}
\end{frame}

\begin{frame}{\textbf{Результат 5.} Проведено теоретическое сравнение рассмотренных классов инвариантов, доказаны леммы о накачке}
\framesubtitle{\textcolor{myResult}{Теоремы, доказанные в диссертации}; \textcolor{trivialResult}{тривиальные теоремы}}
\begin{table}
\scriptsize
\begin{tabular}{| m{31mm} || c | c | c | c | c | c |}
\hline
\diagbox[width=35mm]{Свойство}{Класс} & \elemclass{} & \sizeelemclass{} & \regclass{} & \syncRegFlatClass{} & \syncRegFullClass{} & \regelemclass{} \\
\hline
Замкнут по $\cap$       & \itsTrivial{Да} & \itsTrivial{Да} & Да & \itsMyresult{Да} & \itsMyresult{Да} & \itsTrivial{Да} \\
Замкнут по $\cup$       & \itsTrivial{Да} & \itsTrivial{Да} & Да & \itsMyresult{Да} & \itsMyresult{Да} & \itsTrivial{Да} \\
Замкнут по $\setminus$       & \itsTrivial{Да} & \itsTrivial{Да} & Да & \itsMyresult{Да} & \itsMyresult{Да} & \itsTrivial{Да} \\
Разрешимо $\overline{t} \in I$          & Да & Да & Да & Да & \itsMyresult{Да} & Да \\
Разрешимо $I = \emptyset$    & Да & Да & Да & Да & \itsMyresult{Да} & Да\\
Выразимы рекурсивные отношения & \itsTrivial{Нет} & \itsTrivial{Частично} & \itsTrivial{Да} & \itsTrivial{Да} & \itsTrivial{Да} & \itsTrivial{Да} \\
Выразимы синхронные отношения & \itsTrivial{Да} & \itsTrivial{Да} & \itsTrivial{Нет} & \itsTrivial{Частично} & \itsTrivial{Да} & \itsTrivial{Да} \\
\hline
\end{tabular}
\end{table}
\begin{table}
\scriptsize
\centering
\begin{tabular}{| m{15mm} || x{13mm} | x{17mm} | x{10mm} | c | c | c |}
\hline
Класс & \elemclass{} & \sizeelemclass{} & \regclass{} & \syncRegFlatClass{} & \syncRegFullClass{} & \regelemclass{} \\
\hline
\elemclass{} & \itsTrivial$\emptyset$ & \itsTrivial$\emptyset$ & \itsMyresult\exLR{} & \itsMyresult\exLR{} & \itsMyresult\exLR{} & \itsTrivial$\emptyset$\\
\sizeelemclass{} & \itsTrivial$\infty$ & \itsTrivial$\emptyset$ & \itsMyresult\exLR{} & \itsMyresult\exLR{} & \itsMyresult\exLR{} & \itsMyresult\exLt{} \\
\regclass{} & \itsMyresult\exEvenLeft{} & \itsMyresult\exEvenLeft{} & \itsTrivial$\emptyset$ & \itsTrivial$\emptyset$ & \itsTrivial$\emptyset$ & \itsTrivial$\emptyset$\\
\syncRegFlatClass{} & \itsMyresult\exEvenLeft{} & \itsMyresult\exEvenLeft{} & \itsTrivial$\infty$ & \itsTrivial$\emptyset$ & \itsTrivial$\emptyset$ & \itsMyresult\exLt{}\\
\syncRegFullClass{} & \itsMyresult\exEvenLeft{} & \itsMyresult\exEvenLeft{} & \itsTrivial$\infty$ & \itsTrivial$\infty$ & \itsTrivial$\emptyset$ & \itsMyresult\exLt{}\\
\regelemclass{} & \itsTrivial$\infty$ & \itsMyresult\exEvenLeft{} & \itsTrivial$\infty$ & \itsMyresult\exLR{} & \itsMyresult\exLR{} & \itsTrivial$\emptyset$\\
\hline
\end{tabular}
\end{table}
\end{frame}

% \begin{frame}{\textbf{Результат 6.} Реализация}
% \begin{figure}[h]
% % https://online.visual-paradigm.com/share.jsp?id=323637353533352d31
% \centering
% \includegraphics[width=\textwidth]{resources/arch.png}
% \caption{Хорн-решатель \theringen{}: \url{https://github.com/Columpio/RInGen}}
% % \label{fig:ringen-arch}
% \end{figure}
% \end{frame}

\begin{frame}{\textbf{Результат 6.} Выполнена реализация и проведены эксперименты}
\vspace*{-10pt}
\begin{columns}
\begin{column}{.42\textwidth}
\begin{table}
% \caption{Результаты экспериментов. <<SAT>> обозначает, что система безопасна (есть индуктивный инвариант), <<UNSAT>> обозначает, что система небезопасна.}
% \label{table:eval-all}
\scriptsize
\centering
\begin{tabular}{ |l||c|c| }
\hline
Инструмент & SAT & UNSAT\\\hline\hline
\racer{} & 26 & 22\\
\eldarica{} & 46 & 12\\
\vericat{} & 16 & 10\\
\cvcind{} & 0 & 13\\
\hline
\ringen{\cvc{}} & 25 & 21\\
\ringen{\vampire{}} & 135 & 46\\
\ringenSync{} & 43 & 21\\
\ringenCICI{\cvc{}} & 117 & 19\\
\ringenCICI{\vampire{}} & 189 & 28\\
\hline
\end{tabular}
\end{table}
\end{column}
\begin{column}{.5\textwidth}
\pgfplotstableread[col sep = comma]{resources/experiments1.csv}\all
\begin{figure}[h]
\begin{center}
\begin{tikzpicture}[scale=0.55]
\begin{axis}[xmode=log, ymode=log, legend pos= north west, xlabel={Время работы \ringen{\cvc{}}, мс}, xlabel style = {align=center,font=\footnotesize}, ylabel style = {align=center,font=\footnotesize}, ylabel={Время работы {\color{red}\racer{}}, {\color{blue}\eldarica{}},\\{\color{brown}\cvcind{}} и {\color{cyan}\verimap{}}, мс}]
% \begin{axis}[xmode=log, ymode=log, legend pos= north west, xlabel={ Regular model construction by \theringen{}}, ylabel style = {align=center}, ylabel={Elementary model construction by \\{\color{red}\textsc{Spacer}}, {\color{blue}\eldarica{}} and {\color{brown}\cvcind{}}}]
    \addplot[dashed,no marks,very thin] coordinates {(10,10) (600000,600000)};
    \addplot [dashed, no marks, thin] coordinates {(10,10) (600000,600000)};
    \addplot [dashed, no marks, thin] coordinates {(10,300000) (300000,300000)};
    \addplot [dashed, no marks, thin] coordinates {(300000, 10) (300000,300000)};
    \addplot [dashed, no marks, thin] coordinates {(10, 600000) (600000,600000)};
    \addplot [dashed, no marks, thin] coordinates {(600000, 10) (600000,600000)};

    \addplot  [only marks,  mark=triangle, color=blue, mark size=3pt] table [x={CVC4Finite}, y={Eldarica}] {\all};
    \addplot  [only marks,  mark=o, color=red,  mark size=3pt] table [x={CVC4Finite}, y={Z3}] {\all};
    \addplot  [only marks,  mark=x, color=brown, mark size=3pt] table [x={CVC4Finite}, y={CVC4Ind}] {\all};
    \addplot  [only marks,  mark=square, color=cyan, mark size=3pt] table [x={CVC4Finite}, y={VeriMAP-iddt}] {\all};
\end{axis}

\end{tikzpicture}
    % \caption{Сравнение производительности инструментов. Каждая точка на графике представляет пару длительностей выполнения.}
% \label{fig:toolplotOne}
\end{center}
\end{figure}
\end{column}
\end{columns}
\vspace*{-5pt}
\begin{figure}
\toolplotTwo{\cvc{}}{resources/toolplot_cvc.csv}
~
\toolplotTwo{\vampire{}}{resources/toolplot_vampire.csv}
% \caption{Сравнение времени работы инструментов}
% \label{fig:toolplot}
\end{figure}
\end{frame}


\begin{framesection}{Результаты}
\begin{enumerate}
\item Предложен метод вывода регулярных инвариантов при помощи поиска конечных моделей
\item Предложен метод вывода синхронных регулярных инвариантов при помощи поиска конечных моделей
\item Предложен класс инвариантов, основанный на булевой комбинации элементарных и регулярных инвариантов\\Также предложен метод совместного вывода инвариантов в этом классе посредством вывода инвариантов в подклассах
\item Проведено теоретическое сравнение рассмотренных классов инвариантов\\Доказаны леммы о <<накачке>> для элементарных инвариантов
\item Выполнена пилотная реализация предложенных методов на языке \fsharp{} в рамках инструмента \theringen{}\\Разработанный инструмент решил из бенчмарка <<Tons of Inductive Problems>> в 3.74 раза больше задач, чем наилучший из существующих инструментов
\end{enumerate}
\end{framesection}

\begin{frame}{Соответствие результатов паспорту специальности 2.3.5}

Результаты соответствуют направлению исследования  № 1
\begin{itemize}
\item Модели, \textbf{методы и алгоритмы} проектирования, анализа, трансформации, \textbf{верификации} и тестирования \textbf{программ} и программных систем
\end{itemize}
из паспорта специальности.
\end{frame}

\begin{framesection}{Научная новизна}
\begin{enumerate}
\item Впервые предложены алгоритмы вывода регулярных и синхронных
регулярных инвариантов для программ с АТД, \textbf{отличающиеся тем, что} с
целью автоматизации перебора автоматов над деревьями используют поиск
конечных моделей
\item Впервые предложен класс комбинированных инвариантов, \textbf{отличающийся
тем, что} с целью объединения выразительных возможностей классов
элементарных и регулярных инвариантов предложенный класс построен как
их булева комбинация
\item Предложен новый алгоритм вывода инвариантов в классе комбинированных
инвариантов, \textbf{отличающийся тем, что} с целью переиспользования
эффективных алгоритмов для вывода инвариантов в комбинируемых классах
применяет их на промежуточных артефактах друг друга
\item Впервые введены и доказаны леммы о «накачке» для языков первого
порядка \textbf{отличающиеся тем, что} с целью отделения классов элементарных
инвариантов от других классов используют неспособность языков первого
порядка различать заранее не ограниченные по высоте термы
\end{enumerate}
\end{framesection}

\section{Публикации и выступления}
\begin{frame}{Публикации по теме диссертации}
% \begin{table}
% \centering
% \begin{tabular}{p{.3\textwidth}|p{.7\textwidth}}
% \citep{kostyukov2019auto} & автор выполнил реализацию сведения поиска индуктивных инвариантов функций над сложными структурами данных к решению систем дизъюнктов Хорна, а также спроектировал эксперименты с различными существующими Хорн-решателями\\
% % ; соавторы предложили саму идею и проработали её теоретические аспекты
% \citep{10.1145/3453483.3454055} & автор провёл теоретическое сопоставление классов индуктивных инвариантов, предложил и доказал леммы о <<накачке>> для языков первого порядка над АТД, реализовал предлагаемый подход, поставил эксперименты\\
% % ; соавторы участвовали в обсуждении основных идей статьи, выполнили обзор существующих решений
% \citep{LPAR2023:Collaborative_Inference_of_Combined} & автор предложил и формально обосновал коллаборационный подход к выводу инвариантов, реализовал прототип и поставил эксперименты\\
% % ; соавторы участвовали в обсуждении презентации идей статьи и выполнили обзор существующих решений
% \citep{miso2022gen} & вклад автора заключается в формальном описании теории вычисления предусловий программ со сложными структурами данных
% % ; соавторы участвовали в обсуждении основных идей и реализовали подход
%     \end{tabular}
% \end{table}
\footnotesize
\bibliographystyle{ugost2008ls}
\bibliography{bibliography.bib}
\nocite{kostyukov2019auto}
\nocite{10.1145/3453483.3454055}
\nocite{LPAR2023:Collaborative_Inference_of_Combined}
\nocite{miso2022gen}
% \begin{enumerate}
% \item \citee{kostyukov2019auto}
% \item \citee{10.1145/3453483.3454055}
% \item \citee{LPAR2023:Collaborative_Inference_of_Combined}
% \item \citee{miso2022gen}
% \end{enumerate}
\end{frame}

\begin{frame}{Выступления по теме диссертации}
\begin{itemize}
\item Международный семинар HCVS 2021 (28 марта 2021, Люксембург)
\item Семинар компании Huawei (18-19 ноября 2021, Санкт-Петербург)
\item Ежегодный внутренней семинар JetBrains Research (18 декабря 2021, Санкт-Петербург)
\item Конференция PLDI 2021 (23-25 июня 2021, Канада)
\item Внутренний семинар Венского технического университета (3 июня 2022, Австрия)
\item Конференция LPAR 2023 (4-9 июня 2023, Колумбия)
\end{itemize}

Разработанный инструмент в 2021 и 2022 годах занял, соответственно, 2 и 1 место на АТД секции международных соревнований \chccomp{}.
\end{frame}

% Extra
\begin{frame}{Сравнение Хорн-решателей с поддержкой АТД}
\begin{table}
\centering\footnotesize
% \caption{Сравнение Хорн-решателей с поддержкой АТД}\label{tab:hornSolvers}%
\begin{tabular}{| m{38mm} || x{18mm} | x{18mm} | x{18mm} | x{23mm} |}
% \hline
\hline
Инструмент & Класс\quad\ \ \  инвариантов & Метод & Возвращает инвариант & Полностью автоматический\\\hline\hline
\spacer{} & \elemclass{} & \pdr{} & Да & Да\\
\racer{} & \catelemclass{} & \pdr{} & Нет & Нет\\
\eldarica{} & \sizeelemclass{} & \cegar{} & Да & Да\\
\vericat{} & -- & Трансф. & Нет & Да\\
\hoice{} & \elemclass{} & \ice{} & Да & Да\\
\rchc{}  & \syncRegFlatClass{} & \ice{} & Да & Да\\\hline
\ringen{\cvc} & \regclass{} & Трансф. + \fmf{} & Да & Да\\
\ringen{\vampire} & -- & Трансф. + \satur{} & Нет & Да\\
\ringenSync{} & \syncRegFullClass{} & Трансф. + \fmf{} & Да & Да\\
\ringenCICI{\cvc} & \regelemclass{} & \ourCEGAR{} & Да & Да\\
\ringenCICI{\vampire} & -- & \ourCEGAR{} & Нет & Да\\
\hline
% \hline
\end{tabular}
\end{table}
\end{frame}

\end{document}


\begin{frame}{ATP-based CHC solving\footnote{Details in our PLDI'21 paper:\\\quad``Beyond the elementary representations of program invariants over algebraic data types''}}
\begin{itemize}
    \item<+-> Can we infer \attention{not FOL-based} inductive invariants by employing ATPs?
    \item<+-> Note that a Herbrand model is \attention{almost} a model of the ADT theory
    \item<+-> Yet it \attention{lacks} testers, selectors, and \attention{treats} disequality constraints \attention{differently}
    \item<+-> By syntactic manipulations, CHCs over ADTs can be reduced to the \attention{\textit{free theory}}$^1$
    \vspace*{4mm}\begin{itemize}
        \item i.e., pure first-order logic without any theory axioms
    \end{itemize}
    \item<+-> After that, \attention{any} automated theorem prover can be \attention{soundly used}
\end{itemize}
\end{frame}

\section{Collaborative Inference of Combined Invariants}

\begin{frame}{Implementation}
    \centering\begin{tikzpicture}[font=\Large]
        \node (pic) {\includegraphics[width=.5\textwidth]{resources/spacer_and_vampire.png}};
        \node at (-2.2, -3) {\textsc{Z3 Spacer}};
        \node at (2.2, -3) {\textsc{Vampire}};
        \calloutquote[position={(-2.1,-0.8)},bubblePosition={(1.9,3.3)},callout pointer width=.4cm,fill=focusRed!70,rounded corners]{residual system}
        \calloutquote[position={(1.9,-1.6)},bubblePosition={(-1.9,3.3)},callout pointer width=.4cm,fill=focusBlue!50,rounded corners]{part of invariant}
    \end{tikzpicture}
\end{frame}

\begin{frame}{Experiments: solved systems}
    \centering
    \Large
\begin{table}[h]
    % \small
    % \begin{tabular}{ |c|c|r|c|c|c|c|c|  }
    \begin{tabular}{ |r|c|c|c|c|c|  }
     \hline
     % \multirow{2}{*}{Benchmark} & \multirow{2}{*}{\#} &
     \multirow{2}{*}{Answer} & \multicolumn{2}{c|}{\ringen{}} & \multirow{2}{*}{\spacer{}} & \multicolumn{2}{c|}{\collab{}}\\
     % \cline{4-5}\cline{7-8}
     \cline{2-3}\cline{5-6}
     % &&
     & \cvc{} & \vampire{} && \cvc{} & \vampire{}\\\hline

     % \multirow{2}{*}{\emph{TIP}} & \multirow{2}{*}{454} &
     SAT                   & 25    & 135    & 20   & 117  & \textbf{189}  \\
     % && Unique SAT          & 8     & 48     &  6    &  19   &  21 \\
     % &&
     UNSAT               & 21    & \textbf{46}     & 15    & 19    & 28\\
     % && Unique UNSAT        & 0     & 18     & 2     & 0     & 0\\
     \hline
    \end{tabular}

    % \caption{Time limit: 600 seconds}
\end{table}
Time limit: 600 seconds

% \onslide<2->{
\begin{align*}
  \collab(\spacer{}, \vampire{}) &> \spacer{} \parallel \vampire{}\\
  \collab(\spacer{}, \cvc{}) &> \spacer{} \parallel \cvc{}
\end{align*}
% }
\end{frame}

\begin{frame}{Highlights}
\large
\begin{itemize}
\item<+-> Many practical tasks reduce to CHC solving
\item<+-> CHC solvers build infinite models modulo theories
\item<+-> CHC solvers compute approximations instead of least fixed points
\item<+-> ATPs can be used as CHC solvers for ADTs via simple trick
\item<+-> CHC solvers and ATPs can collaborate instead of competing
\item<+-> Collaboration of CHC solvers and ATPs is strictly stronger than parallel
\end{itemize}

\onslide<7->{\vspace*{10mm}
\centering\Large Thank you! Questions?}
\end{frame}

\appendix

\begin{frame}{Experiments: solving time}
\centering
\Large
\begin{tikzpicture}[scale=.8]
\begin{scope}[shift={(-5,0)}]
\toolplot{\cvc{}}{resources/toolplot_cvc.csv}
\end{scope}
\begin{scope}[shift={(4,0)}]
\toolplot{\vampire{}}{resources/toolplot_vampire.csv}
\end{scope}
\end{tikzpicture}

Runtimes in milliseconds
\end{frame}

\begin{frame}{CHC solvers and CHC Competition 2022}
\includegraphics[width=\textwidth]{resources/chccomp_all.png}
\begin{tikzpicture}[remember picture,overlay]
\begin{scope}[shift={(-0.5,-0.3)}]
% \onslide<2->{\calloutquote[width=3.7cm,position={(-1.08,-0.55)},bubblePosition={(2.8,6.15)},callout pointer width=.5cm,fill=blue!30,rounded corners]{IC3/PDR engine in \textsc{Z3}}
% \calloutquote[width=2.5cm,position={(-1.5,-1)},bubblePosition={(6.4,6.15)},callout pointer width=.5cm,fill=green!40,rounded corners]{Portfolio solver}
% \calloutquote[width=5.6cm,position={(-2.,-1)},bubblePosition={(11,6.15)},callout pointer width=.7cm,fill=yellow!50,rounded corners]{CEGAR with predicate abstraction}
% }
\onslide<2->{
\draw[draw=nicegreen!70,ultra thick] (7.8, 0) rectangle ++(4.2, 3.8);
\calloutquote[width=2.4cm,position={(-1.1,-0.3)},bubblePosition={(12,2.7)},callout pointer width=.4cm,fill=nicegreen!70,rounded corners]{\Large\centering Our journey begins here}}
\end{scope}
\end{tikzpicture}
\end{frame}

\begin{frame}{How RInGen works}
% \centering\begin{tikzpicture}[remember picture,overlay]
% \node[draw,fill=blue!30,opacity=0.8,rounded rectangle,rotate=0,align=center,font=\small] at (0,-2) {* Details in our PLDI'21 paper\\``Beyond the elementary representations of program invariants over algebraic data types''};
% \end{tikzpicture}
% \begin{tikzpicture}[draw,opacity=0.95,rounded rectangle]
% \node[fill=blue!30] at (0, 0) {}
% \end{tikzpicture}
\begin{tikzpicture}
\node[draw,fill=blue!30,opacity=0.95,rounded rectangle] at (0, 0) {Can we infer not FOL-based inductive invariants by employing ATPs?};
\onslide<2->{\node[draw,fill=green!40,opacity=0.95,rounded rectangle] at (2.5, -0.7) {Because obtained model may violate ADT theory axioms};}
\onslide<3->{\node[draw,fill=green!40,opacity=0.95,rounded rectangle] at (1.85, -1.4) {The \textbf{only} reason why Herbrand model does not fit are \textbf{equalities}};}
\end{tikzpicture}
\onslide<4->{
\centering $\Rightarrow$ Eliminate equalities by adding extra CHCs, e.g.:
\begin{align*}
    \bot&\leftarrow \neg(Z = S(Z))\\
    \text{should}&\text{ become}\\
    \bot&\leftarrow diseq_{Nat}(Z, S(Z))\\
    diseq_{Nat}(Z, S(x))&\leftarrow \top\\
    diseq_{Nat}(S(x), Z)&\leftarrow \top\\
    diseq_{Nat}(S(x), S(y))&\leftarrow diseq_{Nat}(x, y)
\end{align*}}
\begin{tikzpicture}[remember picture,overlay]
\onslide<5->{\node[draw,fill=green!40,opacity=0.95,rounded rectangle] at (2, 0.3) {The obtained CHC system can be soundly thrown into ATP};}
\onslide<6>{\node[overlay,draw,fill=blue!30,opacity=0.95,rounded rectangle,minimum height=1.5cm,minimum width=6cm, rotate=15,align=center] at (0,3) {{\huge Details in our PLDI'21 paper}\\{``Beyond the elementary representations of program invariants over algebraic data types''}};}
\end{tikzpicture}
\end{frame}

\whenFullCompile{\input{RInGenHowTo}}

\begin{frame}{ATP-based CHC solving}
\begin{tikzpicture}
\onslide<1->{\node[draw,fill=blue!30,opacity=0.95,rounded rectangle] at (-2, 0) {What kind of inductive invariants can be inferred this way?};}
\onslide<2->{\node[draw,fill=green!40,opacity=0.95,rounded rectangle,align=center] at (2.3, -0.9) {Depends on ATP backend: e.g., if it is a finite-model finder, then\\
it infers \textbf{regular} inductive invariants based on tree automata};}
\end{tikzpicture}

\onslide<3->{
For $even$ example it automatically infers the following inductive invariant:
$$ \invariant \equiv \mathcal{H}\left\{ even \mapsto \mathcal{L}\left(\mathcal{A}\right) \right\} \equiv \mathcal{H}\left\{ even \mapsto\left\{ S^{2n}(Z) \mid n \geq 0 \right\}\right\}$$
based on the tree automaton
$$ \mathcal{A} = \theAutomatoN{s_0, s_1}{\{Z, S\}}{s_0}{\Delta} $$
\vspace*{-10mm}\exampleEven{}}
\begin{tikzpicture}
\onslide<4->{\node[draw,fill=green!40,opacity=0.95,rounded rectangle,align=center] at (0, 0) {\Large The power of the technique is that \textbf{any ATP} can be soundly used};}
\end{tikzpicture}
\end{frame}

\end{document}

% \begin{frame}{Our success on Horn solving competition: \ringen{} with Vampire backend}
% \includegraphics[width=\textwidth]{resources/chccompres.png}
% \begin{tikzpicture}[remember picture,overlay]
% \onslide<2->{\calloutquote[width=3cm,position={(-1,-0.02)},bubblePosition={(5.5,6.4)},fill=red!30,rounded corners]{ADT track on CHC-COMP 2022}}
% \onslide<3->{\calloutquote[width=3.2cm,position={(-1,1.39)},bubblePosition={(2.5,2)},fill=green!30,rounded corners]{CHC-COMP winner for last 4 years}}
% \onslide<4->{
% \draw[draw=blue!30,ultra thick] (5,3) rectangle ++(2.45,2.6);
% \draw[draw=blue!30,ultra thick] (9.97,3) rectangle ++(1.45,2.6);
% \calloutquote[width=4.4cm,position={(-2,-0.55)},bubblePosition={(10,6.4)},callout pointer width=.5cm,fill=blue!30,rounded corners]{win by sat, unsat and time}
% \calloutquote[width=4.4cm,position={(.5,-0.55)},bubblePosition={(10,6.4)},callout pointer width=.3cm,fill=blue!30,rounded corners]{win by sat, unsat and time}
% }
% \end{tikzpicture}
% \end{frame}



% \begin{frame}{Useful links}
% \begin{itemize}
% \item State-of-art Horn solver \spacer{}/\textsc{Spacer}: \url{https://github.com/z3prover/z3}
% \item Our Horn solver \ringen{}: \url{https://github.com/Columpio/RInGen}
% \item Our PLDI paper on \ringen{}: \url{https://arxiv.org/abs/2104.04463}
% \item Our Vampire fork: \url{https://github.com/Columpio/Vampire/tree/chc-comp22}
% \item Demo files for these slides: \url{https://gitlab.com/Columpio/demos}
% \item CHC-COMP results: \href{https://chc-comp.github.io/CHC-COMP2022_presentation.pdf}{\detokenize{chc-comp.github.io/CHC-COMP2022_presentation.pdf}}
% \end{itemize}
% \end{frame}