\newcommand{\evenCHCSystem}{\begin{varexampleblock}[0.83\textwidth]{Система дизъюнктов \onslide<2->{как формула свободной теории}}
    \begin{align*}
    \top&\rightarrow even(Z)\onslide<2->{)\land} \\
    \onslide<2->{\forall y. (}even(y) &\rightarrow even(S(S(y)))\onslide<2->{)\land} \\
    \onslide<2->{\forall x. (}even(x) \land even(S(x)) &\rightarrow \bot\onslide<2->{)}
    \end{align*}
\end{varexampleblock}}

\newcount\rPos
\newcount\modelFloat

\begin{frame}{\textbf{Результат 1.} Предложен метод вывода регулярных инвариантов при помощи поиска конечных моделей и доказана его корректность}
\only<1>{\framesubtitle{\textbf{Этап 1.} Устранить АТД ограничения при помощи унификации и введения новых дизъюнктов}}
\only<2>{\framesubtitle{\textbf{Этап 2.} Трансформировать систему в формулу свободной теории введением логических связок}}
\only<3-4>{\framesubtitle{\textbf{Этап 3.} Передать формулу в сторонний инструмент поиска конечных моделей}}
\only<5->{\framesubtitle{\textbf{Этап 4.} По конечной модели построить автомат над деревьями}}

\centering
\begin{tikzpicture}[remember picture, overlay]
\node<-3>[text width=\textwidth] (evenCHCs) at (-0.32,1.8) {\evenCHCSystem{}};
\onslide<1>{\calloutquote[width=7.8cm,position={(-1,0.7)},bubblePosition={(0,-0.5)},callout pointer width=.4cm,fill=yellow!60,rounded corners]{\Large АТД ограничения устранены}}
\node<3-4>[inner sep=0pt] (fmfinder) at (0.3,-1.8) {\includegraphics[width=.52\textwidth,height=.28\textwidth]{resources/blender.png}};
\node<3-4> (fmfinderName) [above left=-39mm and -58mm of fmfinder,align=center,text width=50mm] {{\large Инструмент поиска конечных моделей}};
\node<3-> (leftpath) [above left=17mm and 0.1cm of fmfinder] {};
\node<3-> (rightpath) [above right=17mm and -0.5cm of fmfinder] {};
\node<3-4> (modelStart) [below=20mm of evenCHCs] {};
\draw<3>[->] (evenCHCs) edge (modelStart) {};
\node<4-> (model) [text width=\textwidth] at (leftpath) {\begin{varblock}[0.5\textwidth]{}
\begin{align*}
    \mathattention<5>{\domain{Nat}}&\mathattention<5>{=\{0,1\}}\\
    \mathattention<6>{\mystructur(Z)}&\mathattention<6>{=0}\\
    \mathattention<7>{\mystructur(S)(x)}&\mathattention<7>{=1-x}\\
    \mathattention<8>{\mystructur(even)}&\mathattention<8>{=\{0\}}
\end{align*}
\end{varblock}};
\node<3-> (modelEnd) [right=10mm of leftpath] {};
\draw<4>[->] (modelStart) |- (modelEnd);
\onslide<9>{\node[overlay,draw,fill=green!40,opacity=0.95,rounded rectangle,minimum height=15mm,minimum width=15cm,align=center] at (0, -1) {Язык построенного автомата является регулярным инвариантом исходной системы\\$\invariant(even) = \langOf{\automat} = \{ S^{2n}(Z) \mid n \geq 0 \}$};}
\onslide<5->{\node<5-> (automaton) at (rightpath) {
\begin{minipage}{.3\textwidth}
\begin{varblock}[\textwidth]{}
\begin{center}
\vphantom{\begin{tikzpicture}[shorten >=1pt,node distance=2cm,on grid,auto]
    \node[state,initial,initial text=$Z$,accepting] (s0) {$0$};
    \node[state] (s1) [right=of s0] {$1$};
    \path[->]
        (s0)    edge [bend left=25] node {$S$}       (s1)
        (s1)    edge [bend left=25] node {$S$}       (s0)
    ;
\end{tikzpicture}}\hspace*{-12mm}\raisebox{9mm}{
\begin{tikzpicture}[remember picture,overlay,shorten >=1pt,node distance=2cm,on grid,auto]
    \node[state,onslide=<6->{initial,initial text=$Z$},onslide=<8->{accepting}] (s0) {$0$};
    \node[state] (s1) [right=of s0] {$1$};
    \path<7->[onslide=<7->,->]
        (s0)    edge [bend left=25] node {$S$}       (s1)
        (s1)    edge [bend left=25] node {$S$}       (s0)
    ;
\end{tikzpicture}}
\end{center}
\end{varblock}
\end{minipage}};}
\end{tikzpicture}
\end{frame}