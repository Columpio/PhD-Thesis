
% There are many different themes available for Beamer. A comprehensive
% list with examples is given here:
% http://deic.uab.es/~iblanes/beamer_gallery/index_by_theme.html
% You can uncomment the themes below if you would like to use a different
% one:
%\usetheme{AnnArbor}
%\usetheme{Antibes}
%\usetheme{Bergen}
%\usetheme{Berkeley}
%\usetheme{Berlin}
%\usetheme{Boadilla}
\usetheme{boxes}
%\usetheme{CambridgeUS}
%\usetheme{Copenhagen}
%\usetheme{Darmstadt}
%\usetheme{default}
%\usetheme{Frankfurt}
%\usetheme{Goettingen}
%\usetheme{Hannover}
%\usetheme{Ilmenau}
%\usetheme{JuanLesPins}
%\usetheme{Luebeck}
% \usetheme{Madrid}
%\usetheme{Malmoe}
%\usetheme{Marburg}
%\usetheme{Montpellier}
%\usetheme{PaloAlto}
%\usetheme{Pittsburgh}
%\usetheme{Rochester}
%\usetheme{Singapore}
%\usetheme{Szeged}
% \usetheme{Warsaw}
% \setbeamertemplate{footline}{} % remove footnote line

\usepackage[T2A]{fontenc}           % кодировка
\usepackage[utf8]{inputenc}         % кодировка исходного текста
\usepackage[english,russian]{babel}
\hypersetup{unicode=true}
% \usepackage[T1]{fontenc}
\usepackage{mathtools}
\usepackage{listings}
\usepackage{tikz}
\usetikzlibrary{matrix, patterns, patterns.meta, backgrounds, automata, positioning, tikzmark, calc, arrows, arrows.meta, shapes, shadows.blur, shapes.misc, shapes.callouts}
\usepackage{minted}
\usepackage{advdate}
\usepackage[normalem]{ulem}
\usepackage[export]{adjustbox}
\usepackage{makecell}
\usepackage{environ}
\usepackage{amsmath}
\usepackage{multirow}
\usepackage{pgfplots}
\usepackage{natbib}
\usepackage{bibentry}
\usepackage{nameref}
\usepackage{etoolbox}
\usepackage{hyperref}
\usepackage{pifont} % for ding
\usepackage{subcaption} % for subfigure
\usepackage{appendixnumberbeamer}
\usepackage{diagbox}
\usepackage{stmaryrd}
\setbeamertemplate{blocks}[rounded][shadow=false]

\setbeamertemplate{footline}{}
\beamertemplatenavigationsymbolsempty
\setbeamertemplate{footline}[frame number]
\let\otp\titlepage
\renewcommand{\titlepage}{\otp\addtocounter{framenumber}{-1}}


% \BeforeBeginEnvironment{itemize}{\vskip-2ex} % remove spaces before itemize

\newtoggle{fastCompile}
\toggletrue{fastCompile}
\newcommand{\whenFullCompile}[1]{\iftoggle{fastCompile}{
{%\setbeamercolor{background canvas}{bg=red}\begin{frame}{FAST COMPILATION MOCK}\end{frame}
}}{#1}}

\newcommand{\btVFill}{\vskip0pt plus 1filll}


\newcommand\blfootnote[1]{%
  \begingroup
  \renewcommand\thefootnote{}\footnote{#1}%
  \addtocounter{footnote}{-1}%
  \endgroup
}

\tikzset{
    onslide/.code args={<#1>#2}{\only<#1>{\pgfkeysalso{#2}}}, 
}
\tikzset{/handlers/.provide style/.code={%
    \pgfkeysifdefined{\pgfkeyscurrentpath/.@cmd}{}%
        {\pgfkeys {\pgfkeyscurrentpath /.code=\pgfkeysalso {#1}}}%
}}

\AtBeginDocument{\setlength\abovedisplayskip{0pt}}
% \AtBeginDocument{\setlength\belowdisplayskip{0pt}}

\pgfkeys{%
    /calloutquote/.cd,
    width/.code                   =  {\def\calloutquotewidth{#1}},
    position/.code                =  {\def\calloutquotepos{#1}}, 
    bubblePosition/.code          =  {\def\calloutquoteBubblePos{#1}}, 
    author/.code                  =  {\def\calloutquoteauthor{#1}},
    /calloutquote/.unknown/.code   =  {\let\searchname=\pgfkeyscurrentname
                                 \pgfkeysalso{\searchname/.try=#1,                                
    /tikz/\searchname/.retry=#1},\pgfkeysalso{\searchname/.try=#1,
                                  /pgf/\searchname/.retry=#1}}
                            }  

\newcommand\calloutquote[2][]{%
       \pgfkeys{/calloutquote/.cd,
         width               = 5cm,
         position            = {(0,-1)},
         author              = {}}
  \pgfqkeys{/calloutquote}{#1}                   
  \node [rectangle callout,callout relative pointer={\calloutquotepos},/calloutquote/.cd,
     #1] (tmpcall) at \calloutquoteBubblePos {#2};
  \node at (tmpcall.pointer){\calloutquoteauthor};    
}  

\newenvironment<>{framesection}[1]{\section{#1}\begin{frame}{#1}#2}{\end{frame}}

\let\oldfootnote\footnote
\renewcommand\footnote[1][]{\oldfootnote[frame,#1]}
\renewcommand{\footref}[1]{\textsuperscript{\ref{#1}}}

\newenvironment<>{varblock}[2][0.9\textwidth]{%
  \begin{center}
    \begin{minipage}{#1}
    \setlength{\textwidth}{#1}%
    \setlength{\linewidth}{\textwidth}% required to itemize respect the width of block
  \begin{actionenv}#3%
    \def\insertblocktitle{#2}%
    \par%
    \usebeamertemplate{block begin}}
  {\par%
  \usebeamertemplate{block end}%
  \end{actionenv}
  \end{minipage}
    \end{center}}



% Variable width example block
\newenvironment<>{varexampleblock}[2][0.9\textwidth]{%
  \begin{center}
    \begin{minipage}{#1}
    \setlength{\textwidth}{#1}%
    \setlength{\linewidth}{\textwidth}%
  \begin{actionenv}#3%
    \def\insertblocktitle{#2}%
    \par%
    \setbeamercolor{local structure}{parent=example text}%
    \usebeamertemplate{block example begin}}
  {\par%
  \usebeamertemplate{block example end}%
    \end{actionenv}
  \end{minipage}
    \end{center}}

% Variable width alert block
\newenvironment<>{varalertblock}[2][0.9\textwidth]{%
  \begin{center}
    \begin{minipage}{#1}
    \setlength{\textwidth}{#1}%
    \setlength{\linewidth}{\textwidth}%
  \begin{actionenv}#3%
    \def\insertblocktitle{#2}%
    \par%
    \setbeamercolor{local structure}{parent=alerted text}%
    \usebeamertemplate{block alerted begin}}
  {\par%
  \usebeamertemplate{block alerted end}%
    \end{actionenv}
  \end{minipage}
    \end{center}}


\tikzset{
    invisible/.style={opacity=0,text opacity=0},
    visible on/.style={alt=#1{}{invisible}},
    alt/.code args={<#1>#2#3}{%
      \alt<#1>{\pgfkeysalso{#2}}{\pgfkeysalso{#3}} 
    },
    beameralert/.style={alt=<#1>{fill=red!30,rounded corners}{},anchor=base},
    BeamerAlert/.style={alt=#1{fill=red!30,rounded corners}{},anchor=base}
}
  \newcommand<>{\tikzMe}[1]{% previously: \def\tikzMe<#1>#2{…
    \tikz[baseline]\node[BeamerAlert=#2,anchor=base] {#1};
}

\newcommand{\mattention}[1]{\color{focusColor}#1} % TODO: \bm
\newcommand<>{\mathattention}[1]{\alt#2{\mattention{#1}}{#1}}
% \addtobeamertemplate{block begin}{\setlength\abovedisplayskip{-1cm}}


\newcommand{\currentTableOfContents}{\begin{frame}{Содержание}\tableofcontents[currentsection]\end{frame}}

\newcommand\eqby[1]{\mathrel{\stackrel{\makebox[0pt]{\mbox{\normalfont\tiny #1}}}{=}}}
\newcommand\eqbyref[1]{\mathrel{\stackrel{\makebox[0pt]{\mbox{\normalfont\tiny\eqref{#1}}}}{=}}}
\newcommand\restr[2]{{% we make the whole thing an ordinary symbol
  \left.\kern-\nulldelimiterspace % automatically resize the bar with \right
  #1 % the function
  \vphantom{\big|} % pretend it's a little taller at normal size
  \right|_{#2} % this is the delimiter
  }}

\newcommand{\csharp}{\textsc{C\#}}
\newcommand{\java}{\textsc{Java}}
\newcommand{\scala}{\textsc{Scala}}
\newcommand{\dotnet}{\textsc{.NET}}
\newcommand{\clang}{\textsc{C}}
\newcommand{\cpplang}{\textsc{C++}}
\newcommand{\coq}{\textsc{Coq}}

\lstdefinelanguage{CSharp}
{
 morecomment = [l]{//}, 
 morecomment = [l]{///},
 morecomment = [s]{/*}{*/},
 morestring=[b]", 
 sensitive = true,
 morekeywords = {abstract,  event,  new,  struct,
  as,  explicit,  null,  switch,
  base,  extern,  object,  this,
  bool,  false,  operator,  throw,
  break,  finally,  out,  true,
  byte,  fixed,  override,  try,
  case,  float,  params,  typeof,
  catch,  for,  private,  uint,
  char,  foreach,  protected,  ulong,
  checked,  goto,  public,  unchecked,
  class,  if,  readonly,  unsafe,
  const,  implicit,  ref,  ushort,
  continue,  in,  return,  using,
  decimal,  int,  sbyte,  virtual,
  default,  interface,  sealed,  volatile,
  delegate,  internal,  short,  void,
  do,  is,  sizeof,  while, where,
  double,  lock,  stackalloc,   
  else,  long,  static,   
  enum,  namespace,  string, Span, var,
  ValueListBuilder}
}

\lstdefinestyle{sharpc}{language=CSharp,
    %frame=lr,
    % frame=l,
    rulecolor=\color{blue!80!black},
    basicstyle=\fontsize{9}{13}\selectfont\ttfamily,
    keywordstyle=\color{blue}\ttfamily,
    stringstyle=\color{red}\ttfamily,
    commentstyle=\color{green}\ttfamily,
    morecomment=[l][\color{magenta}]{\#}
}

\lstdefinelanguage{mycpp}{
  keywords={assume, for, assert},
  keywordstyle=\color{blue}\bfseries,
  keywords=[2]{size_t, int},
  keywordstyle=[2]\color{red}\bfseries,
  identifierstyle=\color{black},
  sensitive=false,
  comment=[l]{//},
  morecomment=[s]{/*}{*/},
  commentstyle=\color{purple}\ttfamily,
  stringstyle=\color{red}\ttfamily,
  morestring=[b]',
  morestring=[b]"
}

\lstset{
   language=mycpp,
   extendedchars=true,
   basicstyle=\small\ttfamily,
   showstringspaces=false,
   showspaces=false,
   tabsize=2,
   breaklines=true,
   showtabs=false
}

\newcommand\eqdef{\mathrel{\stackrel{\makebox[0pt]{\mbox{\normalfont\tiny def}}}{=}}}

\newcommand\drawCodeBox[2]{%
  \begin{tikzpicture}[remember picture,overlay]
    \coordinate (start) at ([yshift=1.7ex]pic cs:#1);
    \coordinate (end) at ([yshift=-0.3ex]pic cs:#2);
    \node[line width=2pt, inner sep=2pt,draw=red,fit=(start) (end)] {};
  \end{tikzpicture}%
}


\newcommand\pair[2]{\langle#1, #2\rangle}
\newcommand\mg[2]{#1=#2}
\newcommand\nmg[2]{#1\neq#2}
\newcommand\li[1]{LI(#1)}
\let\emptyheap\epsilon
\newcommand\agrec{Rec}
\newcommand\agmerge{Merge}
\newcommand\agcompose{\bigcirc}
\newcommand\GRec[1]{\agrec\big(#1\big)}
\newcommand\GMerge[1]{\agmerge\big(#1\big)}
\newcommand\GCompose[2]{#1\agcompose#2}
\newcommand\agho{App}
\newcommand\gapp[1]{\agho(#1)}
\newcommand\GApp[1]{\agho\big(#1\big)}
\newcommand\aunion{union}
\newcommand\union[1]{\aunion\big(#1\big)}
\newcommand\Union[1]{\aunion\Big(#1\Big)}
\newcommand\aderef{read}
\newcommand\afind{find}
\newcommand\find[5]{\afind(#1,#2,#3,#4,#5)}
\newcommand\finD[5]{\afind\big(#1,#2,#3,#4,#5\big)}
\newcommand\Find[5]{\afind\Big(#1,#2,#3,#4,#5\Big)}
\newcommand\deref[2]{\aderef(#1,#2)}
\newcommand\Deref[2]{\aderef\big(#1,#2\big)}
\newcommand\compose[2]{#1\circ#2}
\newcommand\lmbd[2]{\lambda #1.#2}
\newcommand\lmbdx[1]{\lambda x.#1}
\newcommand\dom[1]{dom(#1)}
\newcommand\Dom[1]{dom\big(#1\big)}

\newcommand\exprset{Expr}
\newcommand\guardset{Guard}

\newcommand\pro{\item[$+$]}
\newcommand\contra{\item[$-$]}

\newcommand{\ruquote}[1]{«#1»}

\theoremstyle{plain}
\newtheorem{thm}{Теорема}%[section]
\newtheorem{lem}{Лемма}%[section]
% \newtheorem{crlr}{Corollary}%[section]

\theoremstyle{definition}
\newtheorem{defn}{Определение}
\newtheorem{remk}{Замечание}
\newtheorem{prop}{Утверждение}
\newtheorem{exmp}{Пример}

\newcommand\encircle[1]{%
  \tikz[baseline=(X.base)] 
    \node (X) [draw, shape=circle, inner sep=0] {\strut #1};}
    
    

\definecolor{dgreen}{rgb}{0.,0.6,0.}
\definecolor{focusGreen}{RGB}{7,225,19}
\definecolor{focusRed}{RGB}{254,31,31}
\definecolor{focusViolet}{RGB}{143,00,255}
\definecolor{focusBlue}{HTML}{1620A6}

\makeatletter
\pgfdeclarepatternformonly[\LineSpace]{my north east lines}{\pgfqpoint{-1pt}{-1pt}}{\pgfqpoint{\LineSpace}{\LineSpace}}{\pgfqpoint{\LineSpace}{\LineSpace}}%
{
    \pgfsetcolor{\tikz@pattern@color}
    \pgfsetlinewidth{0.4pt}
    \pgfpathmoveto{\pgfqpoint{0pt}{0pt}}
    \pgfpathlineto{\pgfqpoint{\LineSpace + 0.1pt}{\LineSpace + 0.1pt}}
    \pgfusepath{stroke}
}

\pgfdeclarepatternformonly[\LineSpace]{my north west lines}{\pgfqpoint{-1pt}{-1pt}}{\pgfqpoint{\LineSpace}{\LineSpace}}{\pgfqpoint{\LineSpace}{\LineSpace}}%
{
    \pgfsetcolor{\tikz@pattern@color}
    \pgfsetlinewidth{0.4pt}
    \pgfpathmoveto{\pgfqpoint{0pt}{\LineSpace}}
    \pgfpathlineto{\pgfqpoint{\LineSpace + 0.1pt}{-0.1pt}}
    \pgfusepath{stroke}
}
\makeatother

\newdimen\LineSpace
\tikzset{
    line space/.code={\LineSpace=#1},
    line space=7pt
}

\newcommand{\codefontsize}{\footnotesize}

\NewEnviron{onslideenv}{\onslide<2>{\BODY}}{}


\renewcommand{\phi}{\varphi}

\newcommand{\pdr}{IC3/PDR}
\newcommand{\cegar}{CEGAR}
\newcommand{\ice}{ICE}
\newcommand{\fmf}{FMF}
\newcommand{\satur}{Насыщение}

\newcommand{\theregclass}{Reg}
\newcommand{\regclass}{\textsc{\theregclass{}}}
\newcommand{\theelemclass}{Elem}
\newcommand{\elemclass}{\textsc{\theelemclass{}}}
\newcommand{\elemextclass}{\textsc{\theelemclass{}\textsubscript{*}}}
\newcommand{\sizeelemclass}{\textsc{Size\theelemclass{}}}
\newcommand{\sizeelemextclass}{\textsc{Size\theelemclass{}\textsubscript{*}}}
\newcommand{\catelemclass}{\textsc{Cat\theelemclass{}}}
\newcommand{\syncRegFlatClass}{\regclass{}\textsubscript{+}} %TODO: take it from Haudebourg
\newcommand{\syncRegFullClass}{\regclass{}\textsubscript{$\times$}}
\newcommand{\regelemclass}{\textsc{\theelemclass{}\theregclass{}}}

\newcommand{\zprover}{\textsc{Z3}}
\newcommand{\cvc}{\textsc{cvc5}}
\newcommand{\princess}{\textsc{Princess}}
\newcommand{\chaff}{\textsc{Chaff}}
\newcommand{\cvcind}{\textsc{\cvc{}-Ind}}
\newcommand{\mace}{\textsc{Mace4}}
\newcommand{\kodkod}{\textsc{Kodkod}}
\newcommand{\paradox}{\textsc{Paradox}}
\newcommand{\vampire}{\textsc{Vampire}}
\newcommand{\eprover}{\textsc{E}}
\newcommand{\zipperposition}{\textsc{Zipperposition}}

\newcommand{\eldarica}{\textsc{Eldarica}}
\newcommand{\theringen}{\textsc{RInGen}}
\newcommand{\ringenAny}[2]{#1(#2)}
\newcommand{\ringenSyncAny}[1]{\textsc{#1-Sync}}
\newcommand{\theringenCICIAny}[1]{\textsc{#1-CICI}}
\newcommand{\ringenCICIAny}[2]{\theringenCICIAny{#1}(#2)}
\newcommand{\ringen}[1]{\ringenAny{\theringen}{#1}}
\newcommand{\ringenSync}{\ringenSyncAny{\theringen}}
\newcommand{\theringenCICI}{\theringenCICIAny{\theringen}}
\newcommand{\ringenCICI}[1]{\ringenCICIAny{\theringen}{#1}}
\newcommand{\verifier}{\mathcal{V}}
\newcommand{\spacer}{\textsc{Spacer}}
\newcommand{\racer}{\textsc{Racer}}
\newcommand{\hoice}{\textsc{HoIce}}
\newcommand{\rchc}{\textsc{RCHC}}
\newcommand{\vericat}{\textsc{VeriCaT}}
\newcommand{\verimap}{\textsc{VeriMAP-iddt}}

\newcommand{\signature}{\Sigma}
\newcommand{\theory}{\mathcal{T}}
\newcommand{\relations}{\mathcal{R}}
\newcommand{\oracle}{\mathcal{O}}
\newcommand{\ourCEGAR}{\cegar{}($\oracle$)}
\newcommand{\prog}{\mathcal{S}}
\newcommand{\fsymbs}{\signature_F}
\newcommand{\btuple}[1]{\big\langle #1 \big\rangle}
\newcommand{\theautomaton}[4]{\btuple{#1, #2, #3, #4}}
\newcommand{\autInit}{s_0}
\newcommand{\autStates}{S}
\newcommand{\autFinStates}{\autStates_F}
\newcommand{\autTrans}{\Delta}
\newcommand{\autApp}[1]{A\left(#1\right)}
\newcommand{\automaton}[3]{\theautomaton{#1}{\fsymbs}{#2}{#3}}
\newcommand{\automatonDef}{\automaton{\autStates}{\autFinStates}{\autTrans}}
\newcommand{\Automaton}[2]{\automaton{\{#1\}}{\{#2\}}{\autTrans}}
\newcommand{\sautomatonGen}[4]{\theautomaton{#1}{\fsymbs^{\leq #2}}{#3}{#4}}
\newcommand{\sautomaton}[3]{\sautomatonGen{\{#1\}}{#2}{\{#3\}}{\autTrans}}


\newcommand{\structur}{\mathcal{H}}
\newcommand{\mystructur}{\mathcal{M}}
\newcommand{\invariant}{\mathcal{I}}
\newcommand{\thedomain}{\left|\mystructur\right|}
\newcommand{\domain}[1]{\thedomain_{#1}}
\newcommand{\tuple}[1]{\langle #1 \rangle}


\let\oldemptyset\emptyset
\let\emptyset\varnothing

\newcommand{\formallang}{\mathbf{L}}
\newcommand{\automat}{\mathcal{A}}
\newcommand{\langOf}[1]{\mathcal{L}(#1)}
\newcommand{\height}[1]{\mathcal{H}eight(#1)}

\definecolor{badInvariantColor}{RGB}{255,50,50}
\definecolor{goodInvariantColor}{RGB}{5,150,2}
\definecolor{darkyellow}{RGB}{153,153,0}
\definecolor{focusColor}{RGB}{230,5,5}
\definecolor{nicegreen}{RGB}{5,150,2}
% \definecolor{nicegreen}{rgb}{0.0, 0.8, 0.3}
\newcommand{\xmark}{\text{\ding{53}}}%

\newcommand{\attention}[1]{\textcolor{badInvariantColor}{\textbf{#1}}}

\makeatletter
\newcommand*{\currentname}{\@currentlabelname}
\makeatother

\newcommand{\beamermessage}[2]{%
\begin{beamercolorbox}[wd=#1,center,rounded=true]{author in head/foot}
#2
%[wd=#1,ht=3ex,dp=2ex,center,rounded=true]{author in head/foot}
%\hfill\parbox[c][7ex][c]{#1}{\centering\LARGE #2}\hfill\null
\end{beamercolorbox}}

\makeatletter
\newcommand*\Alt{\alt{\Alt@branch0}{\Alt@branch1}}

\newcommand\Alt@branch[3]{%
  \begingroup
  \ifbool{mmode}{%
    \mathchoice{%
      \Alt@math#1{\displaystyle}{#2}{#3}%
    }{%
      \Alt@math#1{\textstyle}{#2}{#3}%
    }{%
      \Alt@math#1{\scriptstyle}{#2}{#3}%
    }{%
      \Alt@math#1{\scriptscriptstyle}{#2}{#3}%
    }%
  }{%
    \sbox0{#2}%
    \sbox1{#3}%
    \Alt@typeset#1%
  }%
  \endgroup
}

\newcommand\Alt@math[4]{%
  \sbox0{$#2{#3}\m@th$}%
  \sbox1{$#2{#4}\m@th$}%
  \Alt@typeset#1%
}

\newcommand\Alt@typeset[1]{%
  \ifnumcomp{\wd0}{>}{\wd1}{%
    \def\setwider   ##1##2{##2##1##2 0}%
    \def\setnarrower##1##2{##2##1##2 1}%
  }{%
    \def\setwider   ##1##2{##2##1##2 1}%
    \def\setnarrower##1##2{##2##1##2 0}%
  }%
  \sbox2{}%
  \sbox3{}%
  \setwider2{\wd}%
  \setwider2{\ht}%
  \setwider2{\dp}%
  \setnarrower3{\ht}%
  \setnarrower3{\dp}%
  \leavevmode
  \rlap{\usebox#1}%
  \usebox2%
  \usebox3%
}
\makeatother

\newcommand{\Focus}[2]{{\begin{center}#1 #2\end{center}}}
\newcommand{\focus}[1]{\Focus{\huge}{#1}}

% \DeclareFieldFormat*{citetitle}{<<#1>>} % removes ``quotes'' from title
% \newcommand{\citee}[1]{\citetitle{#1}}
\newcommand{\citee}[1]{\bibentry{#1}}

\newcommand{\chccomp}{CHC-COMP}
\newcommand{\fsharp}{F\#}

\delimitershortfall=-1pt % вложенные скобки

\newcolumntype{x}[1]{>{\centering\let\newline\\\arraybackslash\hspace{0pt}}p{#1}}
\newcommand{\tnote}[1]{$^{#1}$}
\newcommand{\exRef}[1]{\textit{#1}}
\newcommand{\exEvenLeft}{\exRef{even}}
\newcommand{\exLR}{\exRef{lr}}
\newcommand{\exLt}{\exRef{lt}}
\newcommand{\exNode}{\exRef{node}}