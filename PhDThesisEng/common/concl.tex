%% Согласно ГОСТ Р 7.0.11-2011:
%% 5.3.3 В заключении диссертации излагают итоги выполненного исследования, рекомендации, перспективы дальнейшей разработки темы.
%% 9.2.3 В заключении автореферата диссертации излагают итоги данного исследования, рекомендации и перспективы дальнейшей разработки темы.

\begin{enumerate}
% \item Предложен эффективный метод автоматического вывода индуктивных инвариантов, основанных на автоматах над деревьями, при этом данные инварианты позволяют выражать рекурсивные отношения в большем количестве реальных программ; метод базируется на поиске конечных моделей.
\item We have proposed an efficient method for automatic inductive invariant inference based on tree automata. With that, these invariants can express recursive relationships across a broad spectrum of real-world programs. The method relies on finite model search.

% \item Предложен метод автоматического вывода индуктивных инвариантов, основанный на трансформации программы и поиске конечных моделей, в классе инвариантов, основанном на синхронных автоматах над деревьями; этот класс инвариантов позволяет выражать рекурсивные отношения и обобщает классические символьные инварианты.
\item We have proposed a method for automatic inductive invariant inference based on program transformation and finite model search within the invariant class based on synchronous tree automata. This class of invariants allows expressing recursive relations and generalizes classical symbolic invariants.
% \item Предложен класс индуктивных инвариантов, основанный на булевой комбинации классических инвариантов и автоматов над деревьями, который, с одной стороны, позволяет выражать рекурсивные отношения в реальных программах, а, с другой стороны, позволяет эффективно выводить индуктивные инварианты; также предложен эффективный метод совместного вывода индуктивных инвариантов в этом классе посредством вывода инвариантов в комбинируемых подклассах.
\item We have proposed a class of inductive invariants based on a Boolean combination of classical invariants and tree automata, which, on the one hand, allows to express recursive relations in real programs, and, on the other hand, allows to effectively infer inductive invariants. We have also proposed an efficient method of combined inductive invariant inference in this class, which infers invariants in the combined subclasses.
% \item Проведено теоретическое сравнение существующих и предложенных классов индуктивных инвариантов; в том числе сформулированы и доказаны леммы о <<накачке>> для языка ограничений и для языка ограничений расширенного функцией размера терма, которые позволяют доказывать невыразимость инварианта в языке ограничений.
\item We have conducted a theoretical comparison of existing and proposed classes of inductive invariants, including the formulation and proof of pumping lemmas for the constraint language and for the constraint language extended with the term size function, which allow to prove the inexpressibility of an invariant in the constraint language.
% \item Выполнена пилотная программная реализация предложенных методов на языке \fsharp{} в рамках инструмента \theringen{}; инструмент сопоставлен с существующими методами на общепринятом тестовом наборе задач верификации функциональных программ <<Tons of Inductive Problems>>: реализация наилучшего из предложенных методов смогла за отведённое время решить в 3.74 раза больше задач, чем наилучший из существующих инструментов.
\item We have completed a pilot software implementation of the proposed methods in the \fsharp{} language as part of the \theringen{} tool; we have then compared this tool with existing methods on a commonly accepted test set of functional program verification tasks "Tons of Inductive Problems": the implementation of the best of the proposed methods was able to solve 3.74 times more tasks in the allotted time than the best of the existing tools.
\end{enumerate}

% В рамках \textbf{рекомендации по применению результатов работы} в индустрии и научных исследованиях указывается, что разработанные методы применимы для автоматизации рассуждений о системах дизъюнктов Хорна над теорией алгебраических типов данных, а также что их реализация выполнена в публично доступном инструменте \theringen{}. Созданный инструмент может быть использован в качестве основной компоненты для верификации в статических анализаторах кода и верификаторах для языков с алгебраическими типами данных, таких как \rust{}, \scala{}, \solidity{}, \haskell{} и \ocaml{}. Инструмент может быть использован для доказательства недостижимости ошибок или заданных фрагментов кода~--- важных для компьютерной безопасности и обеспечения качества задач.
Concerning the \textbf{recommendations for applying the thesis results} in industry and scientific research, the developed methods are applicable for automating reasoning about Horn clause systems over the theory of algebraic data types, and that their implementation is made in a publicly available tool \theringen{}. The created tool can be used as a main component for verification in static code analyzers and verifiers for languages with algebraic data types, such as \rust{}, \scala{}, \solidity{}, \haskell{} and \ocaml{}. The tool can also be used to prove unreachability of errors or specified code fragments, which are important tasks for computer security and quality assurance.

% Также были определены \textbf{перспективы дальнейшей разработки тематики}, основной из которых является расширение предложенных классов индуктивных инвариантов и методов их вывода на комбинации алгебраических типов данных с другими типами данных, распространённых в языках программирования, таких как целые числа, массивы, строковые типы данных. Это позволит выводить инварианты программ со сложными функциональными взаимосвязями между структурами и лежащими в них данными, что существенно расширит практическую применимость предложенных методов.

FInally, we have also defined the \textbf{prospects for further development of the topic}, the main one of which is the extension of the proposed classes of inductive invariants and methods of their inference to combinations of algebraic data types with other data types common in programming languages, such as integers, arrays, and strings. This will allow to infer invariants of programs with complex functional relationships between structures and the data contained therein, which will significantly expand the practical applicability of the proposed methods.
