\chapter{Сравнение и соотнесения}\label{ch:relatedWork}

% Работ, которые касаются, столько-то: много. На первый взгляд только.
% На самом деле.. сужается до работ из введения.
% По странице на каждую из этих работ:
% Должен коррелировать с результатами
% У них то-то, не хватает того-то, таблица
% должен соотносится с разделом научной новизны во введении - даже по структуре, возможно, во введении как выводы к разделам главы
% Минимум - группами обо всём. Тут то, так-то сделали, но из-за того-то, но из-за того-то у них вот этого нет, а мы хотим
% Заточены на соотнесении: все молодцы, но вопрос не решили, поэтому вот я
% Для практики не пригодно, а в работе то, что даёт надежды
% Результаты уже есть, к ним апеллирую
% Работы из соседних областей, вдохновили, там отголоски того-то

Раздел~\cref{sec:relatedWork/hornSolvers} данной главы посвящён сравнению предложенных методов решения систем дизъюнктов Хорна с алгебраическими типами данных с существующими.
В разделе~\cref{sec:relatedWork/modelBuilding} альтернативные рассмотренным способы представлять бесконечные множества термов, основанные на обобщениях автоматов над деревьями, которые могут послужить в качестве классов индуктивных инвариантов программ в будущем.

\section{Хорн-решатели с поддержкой АТД}\label{sec:relatedWork/hornSolvers}

Результаты сравнения приведены в таблице~\cref{tab:hornSolvers}.
Стоит отметить, что если инструмент выводит инварианты в конкретном классе, то проблема невыразительности этого класса (невозможность выразить определённые типы отношений, см. главу~\cref{ch:comparison}), становится проблемой незавершаемости инструмента. Иными словами, т.к. ни один из существующих инструментов не проверяет\footnote{С одной стороны, эта задача по сложности сравнима с самой задачей верификации, с другой же стороны до сих пор ей были посвящены лишь отдельные работы (см., например,~\cite{10.1145/3022187,10.1145/2837614.2837640})}, существует ли в принципе инвариант для заданной системы в его классе, в случае отсутствия такового инструмент не будет завершаться.

\begin{table} [htbp]
    \centering
    \begin{threeparttable}% выравнивание подписи по границам таблицы
        \caption{Сравнение Хорн-решателей с поддержкой АТД}\label{tab:hornSolvers}%
        \begin{tabular}{| m{30mm} || c | c | c | c | c | c |}
            \hline
            \hline
            % Раздел & \cref{sec:relatedWork/spacer} & \cref{sec:relatedWork/racer} & \cref{sec:relatedWork/eldarica} & \cref{sec:relatedWork/vericat} & \cref{sec:relatedWork/hoice} & \cref{sec:relatedWork/rchc}\\\hline
            Инструмент & \spacer{} & \racer{} & \eldarica{} & \vericat{} & \hoice{} & \rchc{} \\\hline
            Класс инвариантов & \elemclass{} & \catelemclass{} & \sizeelemclass{} & -- & \elemclass{} & \syncRegFlatClass{}\\\hline
            Метод & \pdr{} & \pdr{} & \cegar{} & Трансф. & \ice{} & \ice{}\\\hline
            Возвращает инвариант & Да & Нет & Да & Нет & Да & Да\\\hline
            Полностью автоматический & Да & Нет & Да & Да & Да & Да\\
            \hline
            \hline
        \end{tabular}
    \end{threeparttable}
\end{table}

% \subsection{Хорн-решатель \spacer{}}\label{sec:relatedWork/spacer}
\paragraph{Инструмент \spacer{}~\cite{komuravelli2016smt}} строит элементарные модели (класс \elemclass{} из раздела~\cref{sec:elem-def}).
Этот инструмент построен на базе классической разрешающей процедуры для АТД, процедуры интерполяции и устранения кванторов~\cite{bjorner2015playing}.
Ядро инструмента, подход \spacer{}, основан на технике, называемой \emph{достижимость, направляемая свойством} \foreignlanguage{english}{(property-directed reachability, \pdr{})}, которая равномерно распределяет время анализа между поиском контрпримеров и построением безопасного индуктивного инварианта, распространяя факты о достижимости небезопасных свойств и частичные леммы о безопасности.
Позволяет выводить инварианты в комбинации алгебраических и других типов данных, возвращает проверяемые сертификаты, подход корректен и полон.
Минусом инструмента является то, что он выражает инварианты в языке ограничений, а потому часто не завершается на проблемах с АТД.

% \subsection{Хорн-решатель \racer{}}\label{sec:relatedWork/racer}
\paragraph{Инструмент \racer{}~\cite{10.1145/3498722}} является развитием инструмента \spacer{}: он позволяет выводить инварианты в языке ограничений, расширенном катаморфизмами, обозначенном в таблице~\ref{tab:hornSolvers} как \catelemclass{}. Он также наследует все плюсы подхода \spacer{}. Минусом подхода является то, что он не полностью автоматический, т.к. он требует вручную описывать катаморфизмы, что может быть затруднительно на практике: по заданной проблеме сложно понять, какие катаморфизмы потребуются в её инварианте. Минусом самого инструмента на данный момент является то, что он не возвращает какие-либо проверяемые сертификаты с катаморфизмами.

% \subsection{Хорн-решатель \eldarica{}}\label{sec:relatedWork/eldarica}
\paragraph{Инструмент \eldarica{}~\cite{8603013}} строит модели с ограничениями размера термов, которые подсчитывают общее количество вхождений в них конструкторов (\sizeelemclass{} из раздела~\cref{sec:sizeelem-def}).
Это расширение крайне ограниченно увеличивает выразительность языка ограничений, поскольку введённая функция считает количество всех конструкторов одновременно, и потому при помощи неё невозможно выразить многие свойства, например, ограничение на высоту дерева.
Инструмент \eldarica{} использует подход \cegar{} с абстракцией предикатов и встроенный SMT-решатель \princess{}~\cite{princess}, который предоставляет разрешающую процедуру и процедуру интерполяции для АТД с ограничениями размера термов. Эти процедуры построены на сведении этой теории к комбинации теорий неинтерпретируемых функций и линейной арифметики~\cite{hojjat2017deciding}.

% \subsection{Другие инструменты с поддержкой АТД}
% Инструмент \cvc{} в режиме индукции~\cite{reynolds2015induction} (обозначается \cvcind{}\footnote{Данный режим реализуется запуском инструмента \cvc{} со следующими флагами: \texttt{quant-ind, quant-cf, conjecture-gen, conjecture-gen-per-round=3, full-saturate-quant.}
% }) использует ряд методов индуктивного рассуждения в рамках классического SMT-подхода.
% Индуктивная техника, применяемая в \cvc{}, глубоко интегрирована в SMT-решатель: она реализует сколемизацию с индуктивным усилением и подсчётом термов для поиска дочерних целей.

% \subsection{Хорн-решатель \vericat{}}\label{sec:relatedWork/vericat}
\paragraph{Инструмент \vericat{}~\cite{10.1093/logcom/exab090,pettorossi_proietti_2022,10.1007/978-3-030-51074-9_6,angelis_fioravanti_pettorossi_proietti_2018}} обрабатывает условия проверки над теориями линейной арифметики и АТД и полностью устраняет АТД из исходной системы дизъюнктов путём сворачивания (fold), разворачивания (unfold), введения новых дизъюнктов и других трансформаций.
После работы инструмента получается система дизъюнктов Хорна без АТД, на которой может быть эффективный Хорн-решатель, например, \spacer{} или \eldarica{}.
Основным плюсом подхода является тот факт, что он рассчитан на работу с проблемами, где алгебраические типы данных комбинированы с другими теориями.
Основные минусы подхода следующие: сам процесс трансформации может также не завершаться, а кроме того из-за трансформации невозможно восстановить инвариант исходной системы, т.е. инструмент не возвращает проверяемого сертификата.

% Также были предложены несколько работ для автоматизации индукции над АТД.
% Поддержка индуктивных доказательств существует в дедуктивных верификаторах, таких как \textsc{Dafny}~\cite{Leino12}, и SMT-решателях~\cite{reynolds2015induction}.
% Работа~\cite{de2018solving} представляет метод устранения АТД из систем дизъюнктов Хорна путём преобразования их техникой сворачивания-разворачивания (fold/unfold) в системы над теориями линейной арифметики и неинтерпретированных функций.
% В работе~\cite{yang2019lemma} предложен подход, основанный на синтаксически-направляемом синтезе~\cite{sygus13}, который строит индуктивные инварианты путём порождения вспомогательных лемм на основе неудачных подцелей доказательств и пользовательских шаблонов.
% Проверка индуктивности инвариантов, выраженных в ЯПП, разрешима, поскольку АТД является разрешимой теорией. Вывод инвариантов является полуразрешимым в том смысле, что могут быть перечислены все возможные формулы (кандидаты в инварианты), но если система не имеет инварианта, представимого в ЯПП, такой поиск не завершится.


% \subsection{Хорн-решатель \hoice{}}\label{sec:relatedWork/hoice}
\paragraph{Инструмент \hoice{}~\cite{10.1007/978-3-030-02768-1_8}} строит элементарные инварианты путём подхода, основанного на машинном обучении, \ice{}~\cite{10.1007/978-3-319-08867-9_5}.
Его плюсом является возможность выводить инварианты в комбинации АТД с другими теориями, корректность и полнота, способность возвращать проверяемые сертификаты корректности.
Его минусом является то, что он выводит инварианты в невыразительном языке ограничений, а потому часто не завершается.

% \subsection{Хорн-решатель \rchc{}}\label{sec:relatedWork/rchc}
\paragraph{Инструмент \rchc{}~\cite{haude2020}} также построен на подходе \ice{}, основанном на машинном обучении, однако выражает индуктивные инварианты программ над АТД при помощи \emph{автоматов над деревьями}~\cite{tata}.
Однако из-за сложностей с выражением кортежей термов автоматами, описанных в разделе~\cref{sec:comparison/syncRegStandard}, подход часто неприменим для простейших примеров, где существуют классические символьные инварианты.

\section{Конечные представления множеств термов}\label{sec:relatedWork/modelBuilding}
Существует ряд работ, рассматривающих различные представления моделей теории алгебраических типов данных~\cite{fermuller2007model,fermuller2005model,teucke2019expressivity,gramlich2002algorithmic}.
В частности, известно множество эффективных алгоритмов для работы с автоматами над деревьями.
Тем не менее эти автоматы ограничены по своей выразительной силе, потому некоторые из их расширений широко изучались различными исследователями в области автоматического построения моделей~\cite{caferra2013automated}.
Обзор вычислительных представлений эрбрановских моделей, их свойств, выразительной силы и эффективности необходимых для работы с ними процедур представлен в работах~\cite{matzinger1998computational, matzinger2000computational}.
Также исследуются различные расширения автоматов над деревьями, обладающие свойствами разрешимости и замкнутости базовых языковых операций (например, проверки на пустоту пересечения языков)~\cite{chabin2007visibly, gouranton2001synchronized, limet2001weakly, chabin2006synchronized, jacquemard2009rigid, engelfriet2017multiple}.
В контексте вывода инвариантов важно, что проверка \emph{индуктивности} возможного инварианта (представленного конечным автоматом) разрешима, а проверка \emph{существования} инварианта, представимого конечным автоматом~--- нет.
Для проверки существования однако есть полуразрешающая процедура: если существует конечная модель, её можно найти перебором, но такой перебор не завершится, если система имеет только бесконечные модели.
