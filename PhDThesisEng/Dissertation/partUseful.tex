\chapter{Cheatsheet}

В таблице \cref{tab:tab_pref} приложения~\cref{app:B4} приведён список рекомендуемых
к использованию стандартных префиксов.

Кроме того, для  нумерованных формул \verb|alignedat| делает вертикальное
выравнивание номера формулы по центру формулы. Например, выравнивание
компонент вектора:
\begin{equation}
    \label{eq:2p3}
    \begin{alignedat}{2}
        {\mathbf{N}}_{o1n}^{(j)} = \,{\sin} \phi\,n\!\left(n+1\right)
        {\sin}\theta\,
        \pi_n\!\left({\cos} \theta\right)
        \frac{
        z_n^{(j)}\!\left( \rho \right)
        }{\rho}\,
        &{\boldsymbol{\hat{\mathrm e}}}_{r}\,+   \\
        +\,
        {\sin} \phi\,
        \tau_n\!\left({\cos} \theta\right)
        \frac{
        \left[\rho z_n^{(j)}\!\left( \rho \right)\right]^{\prime}
        }{\rho}\,
        &{\boldsymbol{\hat{\mathrm e}}}_{\theta}\,+   \\
        +\,
        {\cos} \phi\,
        \pi_n\!\left({\cos} \theta\right)
        \frac{
        \left[\rho z_n^{(j)}\!\left( \rho \right)\right]^{\prime}
        }{\rho}\,
        &{\boldsymbol{\hat{\mathrm e}}}_{\phi}\:.
    \end{alignedat}
\end{equation}

Ещё об отступах. Иногда для лучшей <<читаемости>> формул полезно
немного исправить стандартные интервалы \LaTeX\ с учётом логической
структуры самой формулы. Например в формуле~\cref{eq:2p3} добавлен
небольшой отступ \verb+\,+ между основными сомножителями, ниже
результат применения всех вариантов отступа:
\begin{align*}
    \backslash!             & \quad f(x) = x^2\! +3x\! +2         \\
    \mbox{по-умолчанию}     & \quad f(x) = x^2+3x+2               \\
    \backslash,             & \quad f(x) = x^2\, +3x\, +2         \\
    \backslash{:}           & \quad f(x) = x^2\: +3x\: +2         \\
    \backslash;             & \quad f(x) = x^2\; +3x\; +2         \\
    \backslash \mbox{space} & \quad f(x) = x^2\ +3x\ +2           \\
    \backslash \mbox{quad}  & \quad f(x) = x^2\quad +3x\quad +2   \\
    \backslash \mbox{qquad} & \quad f(x) = x^2\qquad +3x\qquad +2
\end{align*}

Можно использовать разные математические алфавиты:
\begin{align}
    \mathcal{ABCDEFGHIJKLMNOPQRSTUVWXYZ} \nonumber  \\
    \mathfrak{ABCDEFGHIJKLMNOPQRSTUVWXYZ} \nonumber \\
    \mathbb{ABCDEFGHIJKLMNOPQRSTUVWXYZ} \nonumber
\end{align}

А вот так пишется нумерованная формула:
\begin{equation}
    \label{eq:equation1}
    e = \lim_{n \to \infty} \left( 1+\frac{1}{n} \right) ^n
\end{equation}

Нумерованных формул может быть несколько:
\begin{equation}
    \label{eq:equation2}
    \lim_{n \to \infty} \sum_{k=1}^n \frac{1}{k^2} = \frac{\pi^2}{6}
\end{equation}

Впоследствии на формулы~\cref{eq:equation1, eq:equation2} можно ссылаться.

Уравнения~\cref{eq:subeq_1,eq:subeq_2} демонстрируют возможности
окружения \verb|\subequations|.
\begin{subequations}
    \label{eq:subeq_1}
    \begin{gather}
        y = x^2 + 1 \label{eq:subeq_1-1} \\
        y = 2 x^2 - x + 1 \label{eq:subeq_1-2}
    \end{gather}
\end{subequations}
Ссылки на отдельные уравнения~\cref{eq:subeq_1-1,eq:subeq_1-2,eq:subeq_2-1}.
\begin{subequations}
    \label{eq:subeq_2}
    \begin{align}
        y & = x^3 + x^2 + x + 1 \label{eq:subeq_2-1} \\
        y & = x^2
    \end{align}
\end{subequations}

Числа форматируются при помощи команды \verb|\num|:
\num{5,3};
\num{2,3e8};
\num{12345,67890};
\num{2,6 d4};
\num{1+-2i};
\num{.3e45};
\num[exponent-base=2]{5 e64};
\num[exponent-base=2,exponent-to-prefix]{5 e64};
\num{1.654 x 2.34 x 3.430}
\num{1 2 x 3 / 4}.
Для написания последовательности чисел можно использовать команды \verb|\numlist| и \verb|\numrange|:
\numlist{10;30;50;70}; \numrange{10}{30}.

\subsection{Заголовки с формулами: \texorpdfstring{\(a^2 + b^2 = c^2\)}{%
        a\texttwosuperior\ + b\texttwosuperior\ = c\texttwosuperior},
    \texorpdfstring{\(\left\vert\textrm{{Im}}\Sigma\left(
            \protect\varepsilon\right)\right\vert\approx const\)}{|ImΣ (ε)| ≈ const},
    \texorpdfstring{\(\sigma_{xx}^{(1)}\)}{σ\_\{xx\}\textasciicircum\{(1)\}}
}\label{subsec:with_math}

Пакет \texttt{hyperref} берёт текст для закладок в pdf-файле из~аргументов
команд типа \verb|\section|, которые могут содержать математические формулы,
а~также изменения цвета текста или шрифта, которые не отображаются в~закладках.
Чтобы использование формул в заголовках не вызывало в~логе компиляции появление
предупреждений типа <<\texttt{Token not allowed in~a~PDF string
    (Unicode):(hyperref) removing...}>>, следует использовать конструкцию
\verb|\texorpdfstring{}{}|, где в~первых фигурных скобках указывается
формула, а~во~вторых "--- запись формулы для закладок.

\section{Работа со списком сокращений и~условных обозначений}\label{sec:acronyms}

С помощью пакета \texttt{nomencl} можно создавать удобный сортированный список
сокращений и условных обозначений во время написания текста. Вызов
\verb+\nomenclature+ добавляет нужный символ или сокращение с~описанием
в~список, который затем печатается вызовом \verb+\printnomenclature+
в~соответствующем разделе.
Для того, чтобы эти операции прошли, потребуется дополнительный вызов
\verb+makeindex -s nomencl.ist -o %.nls %.nlo+ в~командной строке, где вместо
\verb+%+ следует подставить имя главного файла проекта (\verb+dissertation+
для этого шаблона).
Затем потребуется один или два дополнительных вызова компилятора проекта.
\begin{equation}
    \omega = c k,
\end{equation}
где \( \omega \) "--- частота света, \( c \) "--- скорость света, \( k \) "---
модуль волнового вектора.
\nomenclature{\(\omega\)}{частота света\nomrefeq}
\nomenclature{\(c\)}{скорость света\nomrefpage}
\nomenclature{\(k\)}{модуль волнового вектора\nomrefeqpage}
Использование
\begin{verbatim}
\nomenclature{\(\omega\)}{частота света\nomrefeq}
\nomenclature{\(c\)}{скорость света\nomrefpage}
\nomenclature{\(k\)}{модуль волнового вектора\nomrefeqpage}
\end{verbatim}
после уравнения добавит в список условных обозначений три записи.
Ссылки \verb+\nomrefeq+ на последнее уравнение, \verb+\nomrefpage+ "--- на
страницу, \verb+\nomrefeqpage+ "--- сразу на~последнее уравнение и~на~страницу,
можно опускать и~не~использовать.

Группировкой и сортировкой пунктов в списке можно управлять с~помощью указания
дополнительных аргументов к команде \verb+nomenclature+.
Например, при вызове
\begin{verbatim}
\nomenclature[03]{\( \hbar \)}{постоянная Планка}
\nomenclature[01]{\( G \)}{гравитационная постоянная}
\end{verbatim}
\( G \) будет стоять в списке выше, чем \( \hbar \).
Для корректных вертикальных отступов между строками в описании лучше
не~использовать многострочные формулы в~списке обозначений.

\nomenclature{%
    \( \begin{rcases}
        a_n \\
        b_n
    \end{rcases} \)%
}{коэффициенты разложения Ми в дальнем поле соответствующие электрическим и
    магнитным мультиполям}
\nomenclature[a\( e \)]{\( {\boldsymbol{\hat{\mathrm e}}} \)}{единичный вектор}
\nomenclature{\( E_0 \)}{амплитуда падающего поля}
\nomenclature{\( j \)}{тип функции Бесселя}
\nomenclature{\( k \)}{волновой вектор падающей волны}
\nomenclature{%
    \( \begin{rcases}
        a_n \\
        b_n
    \end{rcases} \)%
}{и снова коэффициенты разложения Ми в дальнем поле соответствующие
    электрическим и магнитным мультиполям. Добавлено много текста, так что
    описание группы условных обозначений значительно превысило высоту этой
    группы...}
\nomenclature{\( L \)}{общее число слоёв}
\nomenclature{\( l \)}{номер слоя внутри стратифицированной сферы}
\nomenclature{\( \lambda \)}{длина волны электромагнитного излучения в вакууме}
\nomenclature{\( n \)}{порядок мультиполя}
\nomenclature{%
    \( \begin{rcases}
        {\mathbf{N}}_{e1n}^{(j)} & {\mathbf{N}}_{o1n}^{(j)} \\
        {\mathbf{M}_{o1n}^{(j)}} & {\mathbf{M}_{e1n}^{(j)}}
    \end{rcases} \)%
}{сферические векторные гармоники}
\nomenclature{\( \mu \)}{магнитная проницаемость в вакууме}
\nomenclature{\( r, \theta, \phi \)}{полярные координаты}
\nomenclature{\( \omega \)}{частота падающей волны}

\subsection{Пробелы}

В~русском наборе принято:
\begin{itemize}
    \item единицы измерения, знак процента отделять пробелами от~числа:
          10~кВт, 15~\% (согласно ГОСТ 8.417, раздел 8);
    \item \(\tg 20\text{\textdegree}\), но: 20~{\textdegree}C
          (согласно ГОСТ 8.417, раздел 8);
    \item знак номера, параграфа отделять от~числа: №~5, \S~8;
    \item стандартные сокращения: т.\:е., и~т.\:д., и~т.\:п.;
    \item неразрывные пробелы в~предложениях.
\end{itemize}

\subsection{Кавычки}
В английском языке приняты одинарные и двойные кавычки в~виде ‘...’ и~“...”.
В~России приняты французские («...») и~немецкие („...“) кавычки (они называются
«ёлочки» и~«лапки», соответственно). ,,Лапки`` обычно используются внутри
<<ёлочек>>, например, <<... наш гордый ,,Варяг``...>>.

\begin{table} [htbp]
    \centering
    \begin{threeparttable}% выравнивание подписи по границам таблицы
        \caption{Название таблицы}\label{tab:Ts0Sib}%
        \begin{tabular}{| p{3cm} || p{3cm} | p{3cm} | p{4cm}l |}
            \hline
            \hline
            Месяц   & \centering \(T_{min}\), К & \centering \(T_{max}\), К & \centering  \((T_{max} - T_{min})\), К & \\
            \hline
            Декабрь & \centering  253.575       & \centering  257.778       & \centering      4.203                  & \\
            Январь  & \centering  262.431       & \centering  263.214       & \centering      0.783                  & \\
            Февраль & \centering  261.184       & \centering  260.381       & \centering     \(-\)0.803              & \\
            \hline
            \hline
        \end{tabular}
    \end{threeparttable}
\end{table}

\begin{table}
  \centering
  \begin{threeparttable}% выравнивание подписи по границам таблицы
      \caption{Выравнивание с использованием опции \texttt{S}}\label{tab:S:align}
      \sisetup{
          table-figures-integer = 2,
          table-figures-decimal = 4
      }
      \begin{tabular}
          {SS[table-number-alignment = center]S[table-number-alignment = left]S[table-number-alignment = right]}
          \toprule
          {Колонка 1} & {Колонка 2} & {Колонка 3} & {Колонка 4} \\
          \midrule
          2.3456      & 2.3456      & 2.3456      & 2.3456      \\
          34.2345     & 34.2345     & 34.2345     & 34.2345     \\
          56.7835     & 56.7835     & 56.7835     & 56.7835     \\
          90.473      & 90.473      & 90.473      & 90.473      \\
          \bottomrule
      \end{tabular}
  \end{threeparttable}
\end{table}
