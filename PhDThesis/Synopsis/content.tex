\pdfbookmark{Общая характеристика работы}{characteristic}             % Закладка pdf
\section*{Общая характеристика работы}

\newcommand{\actuality}{\pdfbookmark[1]{Актуальность}{actuality}\underline{\textbf{\actualityTXT}}}
\newcommand{\progress}{\pdfbookmark[1]{Степень разработанности темы}{progress}\underline{\textbf{\progressTXT}}}
\newcommand{\aim}{\pdfbookmark[1]{Цели}{aim}\underline{{\textbf\aimTXT}}}
\newcommand{\tasks}{\pdfbookmark[1]{Задачи}{tasks}\underline{\textbf{\tasksTXT}}}
\newcommand{\aimtasks}{\pdfbookmark[1]{Цели и задачи}{aimtasks}\aimtasksTXT}
\newcommand{\novelty}{\pdfbookmark[1]{Научная новизна}{novelty}\underline{\textbf{\noveltyTXT}}}
\newcommand{\influencePr}{\textbf{\influencePrTXT}}
\newcommand{\influenceTh}{\textbf{\influenceThTXT}}
\newcommand{\methods}{\pdfbookmark[1]{Методология и методы исследования}{methods}\underline{\textbf{\methodsTXT}}}
\newcommand{\defpositions}{\pdfbookmark[1]{Положения, выносимые на защиту}{defpositions}\underline{\textbf{\defpositionsTXT}}}
\newcommand{\reliability}{\pdfbookmark[1]{Достоверность}{reliability}\underline{\textbf{\reliabilityTXT}}}
\newcommand{\probation}{\pdfbookmark[1]{Апробация}{probation}\underline{\textbf{\probationTXT}}}
\newcommand{\contribution}{\pdfbookmark[1]{Личный вклад}{contribution}\underline{\textbf{\contributionTXT}}}
\newcommand{\publications}{\pdfbookmark[1]{Публикации}{publications}\underline{\textbf{\publicationsTXT}}}

{\actuality}
Программные системы охватывают всё больше сфер человеческой деятельности, и всё острее стоит вопрос об их корректности.
Область формальных методов традиционно занимается вопросами качества программ. С 90-х годов XX века в этой области началась новая страница~--- появились бинарные диаграммы решений, а затем символьная проверка моделей на основе эффективных SAT-решателей, что позволило верифицировать системы с \(10^{120}\) возможными состояниями~\cite{10.1007/3-540-61474-5_93}. Благодаря SAT-революции всё меньше статических анализаторов создаётся <<с нуля>>, всё чаще они надстраиваются над стеком верификации: SAT-решатели для логики высказываний, построенные на их основе SMT-решатели для теорий логики первого порядка, и далее~--- Хорн-решатели для вывода индуктивных инвариантов.

Новые подходы к статическому анализу дают индустрии много плодов.
Так, например, в 2008 году около трети всех детектируемых ошибок при разработке Windows~7 нашёл инструмент SAGE~\cite{10.1145/2090147.2094081}, основанный на символьном исполнении и активно использующий SMT-решатель для проверки достижимости ветвей исполнения программ.

 В формальных методах большое значение имеют типы данных, так как для них требуются подходящие формализации, чтобы учитывать их при верификации программ. Однако большинство исследований здесь направлено на поддержку <<классических>> типов данных, таких как целые числа и массивы. Менее исследованными оказываются новые, набирающие популярность, типы данных, например, \emph{алгебраические типы данных (АТД)}\footnote{В зависимости от подхода их также называют \emph{абстрактными типами данных}, \emph{индуктивными типами данных} и \emph{рекурсивными типами данных}.}.
Последние строятся рекурсивно, при помощи объединения и декартового перемножения типов. Используя АТД, можно описывать односвязные списки,  бинарные деревья и другие сложные структуры данных.  АТД активно используются в функциональных языках, таких как \haskell{} и \ocaml{},  являясь альтернативой перечислениям и объединениям в языках \clanguage{} и \cplusplus{}. Алгебраические типы данных всё чаще включают в современные языки программирования, используемые в индустрии, например, в языки \rust{} и \scala{}, а также в языки самовыполняющихся контрактов, например, \solidity{}~\cite{8327565}. Так, например, Twitter использует язык \scala{} для большинства своих серверных приложений~\cite{10.1145/1900160.1900170}, Dropbox~--- язык \rust{} для управления потоками данных~\cite{dropboxRust}, и в обоих случая активно используются алгебраические типы данных.

Таким образом становится насущной задача обеспечения корректности программ, использующих АТД.
Эта задача может быть формализована, а её решение~--- частично автоматизировано в рамках дедуктивной верификации на основе логики Флойда-Хоара~\cite{Floyd1993,10.1145/363235.363259} или уточняющих типов (refinement types)~\cite{713327}, как, например, в системах \flux{}~\cite{https://doi.org/10.48550/arxiv.2207.04034} для языка \rust{} и \leon{}~\cite{10.1007/978-3-642-23702-7_23} для языка \scala{}.
Однако такие подходы требуют от пользователя предоставления \emph{индуктивных инвариантов} для доказательства корректности программы, формулировка которых на практике является крайне трудоёмкой задачей. Системы верификации, основанные на самостоятельных языках программирования и поддерживающие АТД, такие как \dafny{}~\cite{10.1007/978-3-642-17511-4_20}, \whyThree{}~\cite{10.1007/978-3-642-37036-6_8}, \viper{}~\cite{10.1007/978-3-662-49122-5_2}, \fstar{}~\cite{10.1145/2914770.2837655}, сталкиваются с той же проблемой.
Также следует отметить, что алгебраические типы данных лежат в основе многочисленных интерактивных систем проверки доказательств (interactive theorem prover, ITP), таких как \coq{}~\cite{barras1999coq}, \idris{}~\cite{brady_2013}, \agda{}~\cite{10.1145/3341691}, \lean{}~\cite{10.1007/978-3-030-79876-5_37}.
Методы автоматизации индукции в таких системах, как правило, ограничены синтаксическим перебором, поэтому в процессе доказательства пользователь вынужден осуществлять трудоёмкую деятельность по формулировке точной индукционной гипотезы, что тождественно проблеме вывода индуктивных инвариантов.

Таким образом эти задачи сводятся к задаче автоматического вывода индуктивных инвариантов программ с алгебраическими типами данных. В общем виде она может быть сформулирована при помощи систем \emph{дизъюнктов Хорна с ограничениями} (constrained Horn clauses, CHCs)~--- логических формул специального вида, которые позволяют точно моделировать работу программы~\cite{MAKOWSKY1987266}.

Поскольку задача автоматического вывода индуктивных инвариантов сводится к задаче поиска модели для системы дизъюнктов Хорна с ограничениями, инструменты автоматического поиска таких моделей (т.\:н. <<Хорн-решатели>>) могут быть применены в различных контекстах верификации программ~\cite{10.1145/2254064.2254112,Bjorner2015}.
Так, например, инструмент \rustHorn{}~\cite{10.1145/3462205} использует Хорн-решатели для верификации \rust{}-программ, а инструмент \solCMC{}~\cite{10.1007/978-3-031-13185-1_16} применяется для верификации самовыполняющихся контрактов на языке \solidity{}.

Существуют эффективные Хорн-решатели, поддерживающие АТД, такие как \spacer{}~\cite{komuravelli2016smt} и его приемник \racer{}~\cite{10.1145/3498722}, а также \eldarica{}~\cite{8603013}, \hoice{}~\cite{10.1007/978-3-030-02768-1_8}, \rchc{}~\cite{haude2020}, \vericat{}~\cite{de_angelis_proietti_fioravanti_pettorossi_2022}.
Среди Хорн-решателей проводятся ежегодные международные соревнования \chccomp{}~\cite{De_Angelis_2022}, где отдельная секция посвящена решению систем дизъюнктов Хорна с алгебраическими типами данных.

Решение выполнимой системы дизъюнктов Хорна классически представляется в виде т.\:н. \emph{символьной модели} (symbolic model)~\cite{Bjorner2015}, т.\:е. модели, выраженной при помощи формул логики первого порядка в языке ограничений системы дизъюнктов.
Поэтому класс всех индуктивных инвариантов, выразимых с помощью языка ограничений, будем называем \emph{классическими символьными инвариантами}.
Так, например, все Хорн-решатели, участвовавшие в соревнованиях \chccomp{} за последние два года, строят классические символьные инварианты.

Проблема символьных инвариантов в контексте алгебраических типов данных заключается в том, что язык ограничений АТД \emph{не позволяет выразить индуктивные инварианты большинства программ, востребованных на практике}.
А если у безопасной программы нет индуктивных инвариантов, выразимых на языке ограничений, ни один алгоритм вывода индуктивных инвариантов на этом языке не сможет построить для неё индуктивный инвариант. 
Это приводит к тому, что \emph{Хорн-решатели, строящие классические символьные инварианты, не завершаются на большинстве систем с алгебраическими типами данных}.

Термы алгебраических типов имеют \emph{рекурсивную структуру}. Например, бинарное дерево~--- это либо лист, либо вершина с двумя потомками, которые тоже являются бинарными деревьями.
Поэтому основная причина, по которой язык ограничений АТД не позволяет выразить индуктивные инварианты многих программ, состоит в том, что он не позволяет выражать \emph{рекурсивные отношения} над термами алгебраических типов.

{\progress} Проблема невыразимости языка ограничений хорошо известна в научном сообществе. Предпринималось несколько попыток решить эту проблему.

Так в 2018 году Ф.~Рюммер (P.~Ruemmer, Швеция) в рамках Хорн-решателя \eldarica{}~\cite{8603013} предложил выводить индуктивные инварианты в языке ограничений, расширенном функцией размера, подсчитывающей число конструкторов в терме. Однако проблемой этого подхода является то, что любое расширение языка ограничений требует существенной переработки всей процедуры вывода индуктивных инвариантов.
Также в 2022 году в рамках Хорн-решателя \racer{} (H.~Govind, A.~Gurfinkel, США)~\cite{10.1145/3498722} было предложено расширить язык ограничений катаморфизмами~--- рекурсивными функциями некоторого простого вида. Однако в этом случае от пользователя требуется заранее задавать катаморфизмы, которые будут использованы для построения индуктивного инварианта, поэтому этот подход нельзя назвать вполне автоматическим.
% Общей проблемой этих двух подходов является то, что любое расширение языка ограничений требует существенной переработки всей процедуры вывода индуктивных инвариантов.

C 2018 года ведётся отдельная линия исследований (E.~De~Angelis, F.~Fioravant, A.~Pettorossi, Италия)~\cite{10.1093/logcom/exab090,pettorossi_proietti_2022,10.1007/978-3-030-51074-9_6,angelis_fioravanti_pettorossi_proietti_2018}, посвящённая методам устранения алгебраических типов из системы дизъюнктов путём сведения её к системе над более простой теорией, например, над линейной арифметикой. Такой подход реализован в инструменте \vericat{}~\cite{de_angelis_proietti_fioravanti_pettorossi_2022}. Ограничением подобных подходов является невозможность восстановления индуктивного инварианта исходной системы из индуктивного инварианта более простой системы.

В 2020 году в рамках инструмента \rchc{} (T.~Haudebourg, Франция)~\cite{haude2020} было предложено выражать индуктивные инварианты программ над АТД при помощи \emph{автоматов над деревьями}~\cite{tata}.
Однако автоматы над деревьями не позволяют представлять \emph{синхронные отношения}, такие как равенство и неравенство термов, поэтому предложенный подход часто оказывается неприменимым для простейших программ, где инварианты легко находятся классическими методами.

{\aim} данной работы является предложение новых классов индуктивных инвариантов для программ с алгебраическими типами данных и создание для них методов автоматического вывода.
Для реализации этой цели были сформулированы следующие {\tasks}.
\begin{enumerate}[beginpenalty=10000] % https://tex.stackexchange.com/a/476052/104425
  \item Предложить новые классы индуктивных инвариантов программ с алгебраическими типами данных, позволяющие выражать рекурсивные и синхронные отношения.
  \item Создать методы автоматического вывода индуктивных инвариантов в новых классах.
  \item Выполнить пилотную программную реализацию предложенных методов.
  \item Провести экспериментальное сопоставление реализованного инструмента с существующими на представительном тестовом наборе.
\end{enumerate}

{\methods}
Методология исследования заключается в проектировании применимых на практике классов индуктивных инвариантов совместно с разработкой соответствующих алгоритмов, активно используя существующие результаты этой области.
В работе используется логика первого порядка, а также базовые концепции теории автоматов и формальных языков, включая автоматы над деревьями, синхронные автоматы, язык автомата, лемму о <<накачке>>.
Пилотная программная реализация теоретических результатов выполнена на языке \fsharp{}, а также частично на языке \cplusplus{} в рамках кодовой базы Хорн-решателя \racer{} (входит в SMT-решатель \zprover{}, Microsoft Research).

{\defpositions}
% В папке Documents можно ознакомиться с решением совета из Томского~ГУ (в~файле \verb+Def_positions.pdf+), где обоснованно даются рекомендации по~формулировкам защищаемых положений.
\begin{enumerate}[beginpenalty=10000] % https://tex.stackexchange.com/a/476052/104425
  \item Предложен эффективный метод автоматического вывода индуктивных инвариантов, основанных на автоматах над деревьями; при этом данные инварианты позволяют выражать рекурсивные отношения в большем количестве реальных программ; метод базируется на поиске конечных моделей.
  \item Предложен метод автоматического вывода индуктивных инвариантов, основанный на трансформации программы и поиске конечных моделей, в сложном для автоматического вывода инвариантов классе, основанном на синхронных автоматах над деревьями; этот класс инвариантов позволяет выражать рекурсивные и синхронные отношения.
  \item Предложен класс индуктивных инвариантов, основанный на булевой комбинации классических инвариантов и автоматов над деревьями, который, с одной стороны, позволяет выражать рекурсивные отношения в реальных программах, а, с другой стороны, позволяет эффективно выводить индуктивные инварианты; также предложен эффективный метод совместного вывода индуктивных инвариантов в этом классе посредством вывода инвариантов в подклассах.
  \item Проведено теоретическое сравнение существующих и предложенных в рамках диссертации классов индуктивных инвариантов; в том числе сформулированы и доказаны леммы о <<накачке>> для языка ограничений и для языка ограничений расширенного функцией размера терма.
  \item Выполнена пилотная программная реализация предложенных методов на языке \fsharp{} в рамках инструмента \theringen{}; разработанный инструмент сопоставлен с существующими методами на общепринятом тестовом наборе задач верификации функциональных программ <<Tons of Inductive Problems>>; реализация наилучшего из предложенных методов смогла за отведённое время решить в 3.74 раза больше задач, чем наилучший из существующих инструментов.
\end{enumerate}

{\novelty{} полученных результатов состоит в следующем.}
\begin{enumerate}[beginpenalty=10000] % https://tex.stackexchange.com/a/476052/104425
  \item Впервые предложен класс индуктивных инвариантов, основанный на булевой комбинации классов классических и инвариантов, основанных на автоматах над деревьями.
  \item Впервые предложен алгоритм вывода индуктивных инвариантов для программ с алгебраическими типами данных, основанный на поиске конечных моделей.
  \item Предложен новый алгоритм совместного вывода индуктивных инвариантов в комбинации классов инвариантов на базе имеющихся методов вывода инвариантов для отдельных классов.
  \item Впервые введены и доказаны леммы о <<накачке>> для языков первого порядка в сигнатуре теории алгебраических типов данных.
\end{enumerate}

{\influenceTh}
Диссертационное исследование предлагает новые подходы к выводу индуктивных инвариантов программ. Поскольку эти подходы ортогональны существующим, они могут быть перенесены на программы над другими теориями, например, над теорией массивов, а также могут усилить уже существующие подходы к выводу индуктивных инвариантов. Также важным теоретическим вкладом является адаптация лемм о <<накачке>> к языкам первого порядка: эти леммы открывают путь к фундаментальному исследованию проблемы невыразимости индуктивных инвариантов в языках первого порядка и проектированию новых классов индуктивных инвариантов программ.

{\influencePr}
Предложенные методы могут быть применены при создании статических анализаторов для языков с алгебраическими типами данных: поскольку индуктивные инварианты аппроксимируют циклы и функции, они позволяют анализатору корректно <<срезать>> целые классы недостижимых состояний программы и не <<увязать>> в циклах и рекурсии.
Например, предложенные методы могут быть полезны в разработке верификаторов и генераторов тестовых покрытий для таких языков, как \rust{}, \scala{}, \solidity{}, \haskell{} и \ocaml{}.
Поскольку для предложенных методов была выполнена пилотная программная реализация, полученный Хорн-решатель также может быть использован в качестве <<ядра>> статического анализатора, например, для языка \rust{} при помощи фреймворка \rustHorn{}.

{\reliability} полученных результатов обеспечивается формальными доказательствами, а также компьютерными экспериментами на публичных общепринятых тестовых наборах.
Полученные в диссертации результаты согласуются с результатами других авторов в области вывода индуктивных инвариантов.

{\probation}
Основные результаты работы докладывались на следующих научных конференциях и семинарах:
международном семинаре HCVS 2021 (28 марта 2021, Люксембург),
семинаре компании Huawei (18-19 ноября 2021, Санкт-Петербург),
ежегодном внутреннем семинаре JetBrains Research (18 декабря 2021, Санкт-Петербург),
конференции PLDI 2021 (23-25 июня 2021, Канада),
внутреннем семинаре Венского технического университета (3 июня 2022, Австрия),
конференции LPAR 2023 (4-9 июня 2023, Колумбия).

Разработанный инструмент в 2021 и 2022 годах занял, соответственно, 2 и 1 место на международных соревнованиях \chccomp{} (секция по выводу индуктивных инвариантов для программ с алгебраическими типами данных).


\ifnumequal{\value{bibliosel}}{0}
{%%% Встроенная реализация с загрузкой файла через движок bibtex8. (При желании, внутри можно использовать обычные ссылки, наподобие `\cite{vakbib1,vakbib2}`).
    {\publications} Основные результаты по теме диссертации изложены
    в~XX~печатных изданиях,
    X из которых изданы в журналах, рекомендованных ВАК,
    X "--- в тезисах докладов.
}%
{%%% Реализация пакетом biblatex через движок biber
    \begin{refsection}[bl-author, bl-registered]
        % Это refsection=1.
        % Процитированные здесь работы:
        %  * подсчитываются, для автоматического составления фразы "Основные результаты ..."
        %  * попадают в авторскую библиографию, при usefootcite==0 и стиле `\insertbiblioauthor` или `\insertbiblioauthorgrouped`
        %  * нумеруются там в зависимости от порядка команд `\printbibliography` в этом разделе.
        %  * при использовании `\insertbiblioauthorgrouped`, порядок команд `\printbibliography` в нём должен быть тем же (см. biblio/biblatex.tex)
        %
        % Невидимый библиографический список для подсчёта количества публикаций:
        \printbibliography[heading=nobibheading, section=1, env=countauthorvak,          keyword=biblioauthorvak]%
        \printbibliography[heading=nobibheading, section=1, env=countauthorwos,          keyword=biblioauthorwos]%
        \printbibliography[heading=nobibheading, section=1, env=countauthorscopus,       keyword=biblioauthorscopus]%
        \printbibliography[heading=nobibheading, section=1, env=countauthorconf,         keyword=biblioauthorconf]%
        \printbibliography[heading=nobibheading, section=1, env=countauthorother,        keyword=biblioauthorother]%
        \printbibliography[heading=nobibheading, section=1, env=countregistered,         keyword=biblioregistered]%
        \printbibliography[heading=nobibheading, section=1, env=countauthorpatent,       keyword=biblioauthorpatent]%
        \printbibliography[heading=nobibheading, section=1, env=countauthorprogram,      keyword=biblioauthorprogram]%
        \printbibliography[heading=nobibheading, section=1, env=countauthor,             keyword=biblioauthor]%
        \printbibliography[heading=nobibheading, section=1, env=countauthorvakscopuswos, filter=vakscopuswos]%
        \printbibliography[heading=nobibheading, section=1, env=countauthorscopuswos,    filter=scopuswos]%
        %
        \nocite{*}%
        %
        {\publications} Основные результаты по теме диссертации изложены в~\arabic{citeauthor}~печатных изданиях,
        \arabic{citeauthorvak} из которых изданы в журналах, рекомендованных ВАК\sloppy%
        \ifnum \value{citeauthorscopuswos}>0%
            , \arabic{citeauthorscopuswos} "--- в~периодических научных журналах, индексируемых Web of~Science и Scopus, одна из которых опубликована в тезисах конференции PLDI, имеющей ранг A*, и одна опубликована в тезисах конференции LPAR, имеющей ранг A\sloppy%
        \fi%
        \ifnum \value{citeauthorconf}>0%
            , \arabic{citeauthorconf} "--- в~тезисах докладов.
        \else%
            .
        \fi%
        \ifnum \value{citeregistered}=1%
            \ifnum \value{citeauthorpatent}=1%
                Зарегистрирован \arabic{citeauthorpatent} патент.
            \fi%
            \ifnum \value{citeauthorprogram}=1%
                Зарегистрирована \arabic{citeauthorprogram} программа для ЭВМ.
            \fi%
        \fi%
        \ifnum \value{citeregistered}>1%
            Зарегистрированы\ %
            \ifnum \value{citeauthorpatent}>0%
            \formbytotal{citeauthorpatent}{патент}{}{а}{}\sloppy%
            \ifnum \value{citeauthorprogram}=0 . \else \ и~\fi%
            \fi%
            \ifnum \value{citeauthorprogram}>0%
            \formbytotal{citeauthorprogram}{программ}{а}{ы}{} для ЭВМ.
            \fi%
        \fi%
        % К публикациям, в которых излагаются основные научные результаты диссертации на соискание учёной
        % степени, в рецензируемых изданиях приравниваются патенты на изобретения, патенты (свидетельства) на
        % полезную модель, патенты на промышленный образец, патенты на селекционные достижения, свидетельства
        % на программу для электронных вычислительных машин, базу данных, топологию интегральных микросхем,
        % зарегистрированные в установленном порядке.(в ред. Постановления Правительства РФ от 21.04.2016 N 335)
    \end{refsection}%
    \begin{refsection}[bl-author, bl-registered]
        % Это refsection=2.
        % Процитированные здесь работы:
        %  * попадают в авторскую библиографию, при usefootcite==0 и стиле `\insertbiblioauthorimportant`.
        %  * ни на что не влияют в противном случае
        % \nocite{vakbib2}%vak
        % \nocite{patbib1}%patent
        % \nocite{progbib1}%program
        % \nocite{bib1}%other
        % \nocite{confbib1}%conf
    \end{refsection}%
        %
        % Всё, что вне этих двух refsection, это refsection=0,
        %  * для диссертации - это нормальные ссылки, попадающие в обычную библиографию
        %  * для автореферата:
        %     * при usefootcite==0, ссылка корректно сработает только для источника из `external.bib`. Для своих работ~--- напечатает "[0]" (и даже Warning не вылезет).
        %     * при usefootcite==1, ссылка сработает нормально. В авторской библиографии будут только процитированные в refsection=0 работы.
}

% При использовании пакета \verb!biblatex! будут подсчитаны все работы, добавленные
% в файл \verb!biblio/author.bib!. Для правильного подсчёта работ в~различных
% системах цитирования требуется использовать поля:
% \begin{itemize}
%         \item \texttt{authorvak} если публикация индексирована ВАК,
%         \item \texttt{authorscopus} если публикация индексирована Scopus,
%         \item \texttt{authorwos} если публикация индексирована Web of Science,
%         \item \texttt{authorconf} для докладов конференций,
%         \item \texttt{authorpatent} для патентов,
%         \item \texttt{authorprogram} для зарегистрированных программ для ЭВМ,
%         \item \texttt{authorother} для других публикаций.
% \end{itemize}
% Для подсчёта используются счётчики:
% \begin{itemize}
%         \item \texttt{citeauthorvak} для работ, индексируемых ВАК,
%         \item \texttt{citeauthorscopus} для работ, индексируемых Scopus,
%         \item \texttt{citeauthorwos} для работ, индексируемых Web of Science,
%         \item \texttt{citeauthorvakscopuswos} для работ, индексируемых одной из трёх баз,
%         \item \texttt{citeauthorscopuswos} для работ, индексируемых Scopus или Web of~Science,
%         \item \texttt{citeauthorconf} для докладов на конференциях,
%         \item \texttt{citeauthorother} для остальных работ,
%         \item \texttt{citeauthorpatent} для патентов,
%         \item \texttt{citeauthorprogram} для зарегистрированных программ для ЭВМ,
%         \item \texttt{citeauthor} для суммарного количества работ.
% \end{itemize}
% Счётчик \texttt{citeexternal} используется для подсчёта процитированных публикаций;
% \texttt{citeregistered} "--- для подсчёта суммарного количества патентов и программ для ЭВМ.

% Для добавления в список публикаций автора работ, которые не были процитированы в
% автореферате, требуется их~перечислить с использованием команды \verb!\nocite! в
% \verb!Synopsis/content.tex!.

{\contribution} автора в совместных публикациях распределён следующим образом.
В статье~\cite{костюков2019автоматическое} автор выполнил реализацию сведения поиска индуктивных инвариантов функций над сложными структурами данных к решению систем дизъюнктов Хорна, а также спроектировал эксперименты с различными существующими Хорн-решателями; соавторы предложили саму идею и проработали её теоретические аспекты.
В работе~\cite{10.1145/3453483.3454055} автор провёл теоретическое сопоставление классов индуктивных инвариантов, предложил и доказал леммы о <<накачке>> для языков первого порядка над АТД, реализовал предлагаемый подход, поставил эксперименты; соавторы участвовали в обсуждении основных идей статьи, выполнили обзор существующих решений.
В статье~\cite{LPAR2023:Collaborative_Inference_of_Combined} автор предложил и формально обосновал коллаборационный подход к выводу инвариантов, реализовал прототип и поставил эксперименты; соавторы участвовали в обсуждении презентации идей статьи и выполнили обзор существующих решений.
В статье~\cite{мисонижник2022генерация} вклад автора заключается в формальном описании теории вычисления предусловий программ со сложными структурами данных; соавторы участвовали в обсуждении основных идей и реализовали подход.
 % Характеристика работы по структуре во введении и в автореферате не отличается (ГОСТ Р 7.0.11, пункты 5.3.1 и 9.2.1), потому её загружаем из одного и того же внешнего файла, предварительно задав форму выделения некоторым параметрам

%Диссертационная работа была выполнена при поддержке грантов \dots

%\underline{\textbf{Объём и структура работы.}} Диссертация состоит из~введения,
%четырех глав, заключения и~приложения. Полный объем диссертации
%\textbf{ХХХ}~страниц текста с~\textbf{ХХ}~рисунками и~5~таблицами. Список
%литературы содержит \textbf{ХХX}~наименование.

% \pdfbookmark{Содержание работы}{description}                          % Закладка pdf
% \section*{Содержание работы}
% Во \underline{\textbf{введении}} обосновывается актуальность
% исследований, проводимых в~рамках данной диссертационной работы,
% приводится обзор научной литературы по~изучаемой проблеме,
% формулируется цель, ставятся задачи работы, излагается научная новизна
% и практическая значимость представляемой работы. В~последующих главах
% сначала описывается общий принцип, позволяющий \dots, а~потом идёт
% апробация на частных примерах: \dots  и~\dots.


% \underline{\textbf{Первая глава}} посвящена \dots

% картинку можно добавить так:
% \begin{figure}[ht]
%     \centerfloat{
%         \hfill
%         \subcaptionbox{\LaTeX}{%
%             \includegraphics[scale=0.27]{latex}}
%         \hfill
%         \subcaptionbox{Knuth}{%
%             \includegraphics[width=0.25\linewidth]{knuth1}}
%         \hfill
%     }
%     \caption{Подпись к картинке.}\label{fig:latex}
% \end{figure}

% Формулы в строку без номера добавляются так:
% \[
%     \lambda_{T_s} = K_x\frac{d{x}}{d{T_s}}, \qquad
%     \lambda_{q_s} = K_x\frac{d{x}}{d{q_s}},
% \]

% \underline{\textbf{Вторая глава}} посвящена исследованию

% \underline{\textbf{Третья глава}} посвящена исследованию

% Можно сослаться на свои работы в автореферате. Для этого в файле
% \verb!Synopsis/setup.tex! необходимо присвоить положительное значение
% счётчику \verb!\setcounter{usefootcite}{1}!. В таком случае ссылки на
% работы других авторов будут подстрочными.
% Изложенные в третьей главе результаты опубликованы в~\cite{vakbib1, vakbib2}.
% Использование подстрочных ссылок внутри таблиц может вызывать проблемы.

% В \underline{\textbf{четвертой главе}} приведено описание

% \FloatBarrier
% \pdfbookmark{Заключение}{conclusion}                                  % Закладка pdf
% В \underline{\textbf{заключении}} приведены основные результаты работы, которые заключаются в следующем:
% %% Согласно ГОСТ Р 7.0.11-2011:
%% 5.3.3 В заключении диссертации излагают итоги выполненного исследования, рекомендации, перспективы дальнейшей разработки темы.
%% 9.2.3 В заключении автореферата диссертации излагают итоги данного исследования, рекомендации и перспективы дальнейшей разработки темы.

\begin{enumerate}
\item Предложен эффективный метод автоматического вывода индуктивных инвариантов, основанных на автоматах над деревьями, при этом данные инварианты позволяют выражать рекурсивные отношения для большого количества реальных программ; метод базируется на поиске конечных моделей.
\item Предложен метод автоматического вывода индуктивных инвариантов, основанный на трансформации программы и поиске конечных моделей, в классе инвариантов, основанном на синхронных автоматах над деревьями; этот класс позволяет выражать рекурсивные отношения и обобщает классические символьные инварианты.
\item Предложен класс индуктивных инвариантов, основанный на булевой комбинации классических инвариантов и автоматов над деревьями, который, с одной стороны, позволяет выражать рекурсивные отношения в реальных программах, а, с другой стороны, позволяет эффективно выводить индуктивные инварианты; также предложен эффективный метод совместного вывода индуктивных инвариантов в этом классе посредством вывода инвариантов в комбинируемых подклассах.
\item Проведено теоретическое сравнение существующих и предложенных классов индуктивных инвариантов; в том числе сформулированы и доказаны леммы о <<накачке>> для языка ограничений и для языка ограничений расширенного функцией размера терма, которые позволяют доказывать невыразимость инварианта в языке ограничений.
\item Выполнена пилотная программная реализация предложенных методов на языке \fsharp{} в рамках инструмента \theringen{}; инструмент сопоставлен с существующими методами на общепринятом тестовом наборе задач верификации функциональных программ <<Tons of Inductive Problems>>: реализация наилучшего из предложенных методов смогла за отведённое время решить в 3.74 раза больше задач, чем существующие инструменты.
\end{enumerate}

В качестве \textbf{рекомендации по применению результатов работы} в индустрии и научных исследованиях следует указать, что разработанные методы применимы для автоматизации рассуждений о системах дизъюнктов Хорна над теорией алгебраических типов данных, а также что их реализация выполнена в публично доступном инструменте \theringen{}. Созданный инструмент может быть использован в качестве основной компоненты для верификации в статических анализаторах кода и верификаторах для языков с алгебраическими типами данных, таких как \rust{}, \scala{}, \solidity{}, \haskell{} и \ocaml{}. Инструмент может быть использован для доказательства недостижимости ошибок или заданных фрагментов кода, что является важным для задач компьютерной безопасности и обеспечения качества.

В качестве \textbf{перспективы дальнейшей разработки тематики} можно предложить расширение предложенных классов индуктивных инвариантов и методов их вывода на комбинации алгебраических типов данных с другими типами данных, распространённых в языках программирования, таких как целые числа, массивы, строковые типы данных. Это позволит выводить инварианты программ со сложными функциональными взаимосвязями между структурами и лежащими в них данными, что существенно расширит практическую применимость предложенных методов.


\pdfbookmark{Литература}{bibliography}                                % Закладка pdf
% При использовании пакета \verb!biblatex! список публикаций автора по теме
% диссертации формируется в разделе <<\publications>>\ файла
% \verb!common/characteristic.tex!  при помощи команды \verb!\nocite!

\ifdefmacro{\microtypesetup}{\microtypesetup{protrusion=false}}{} % не рекомендуется применять пакет микротипографики к автоматически генерируемому списку литературы
\urlstyle{rm}                               % ссылки URL обычным шрифтом
\ifnumequal{\value{bibliosel}}{0}{% Встроенная реализация с загрузкой файла через движок bibtex8
    \renewcommand{\bibname}{\large \bibtitleauthor}
    \nocite{*}
    \insertbiblioauthor           % Подключаем Bib-базы
    %\insertbiblioexternal   % !!! bibtex не умеет работать с несколькими библиографиями !!!
}{% Реализация пакетом biblatex через движок biber
    % Цитирования.
    %  * Порядок перечисления определяет порядок в библиографии (только внутри подраздела, если `\insertbiblioauthorgrouped`).
    %  * Если не соблюдать порядок "как для \printbibliography", нумерация в `\insertbiblioauthor` будет кривой.
    %  * Если цитировать каждый источник отдельной командой --- найти некоторые ошибки будет проще.
    %
    %% authorvak
    \nocite{vakbib1}%
    \nocite{vakbib2}%
    %
    %% authorwos
    \nocite{wosbib1}%
    %
    %% authorscopus
    \nocite{scbib1}%
    %
    %% authorpathent
    \nocite{patbib1}%
    %
    %% authorprogram
    \nocite{progbib1}%
    %
    %% authorconf
    \nocite{confbib1}%
    \nocite{confbib2}%
    %
    %% authorother
    \nocite{bib1}%
    \nocite{bib2}%

    \ifnumgreater{\value{usefootcite}}{0}{
        \begin{refcontext}[labelprefix={}]
            \ifnum \value{bibgrouped}>0
                \insertbiblioauthorgrouped    % Вывод всех работ автора, сгруппированных по источникам
            \else
                \insertbiblioauthor      % Вывод всех работ автора
            \fi
        \end{refcontext}
    }{
        \ifnum \totvalue{citeexternal}>0
            \begin{refcontext}[labelprefix=A]
                \ifnum \value{bibgrouped}>0
                    \insertbiblioauthorgrouped    % Вывод всех работ автора, сгруппированных по источникам
                \else
                    \insertbiblioauthor      % Вывод всех работ автора
                \fi
            \end{refcontext}
        \else
            \ifnum \value{bibgrouped}>0
                \insertbiblioauthorgrouped    % Вывод всех работ автора, сгруппированных по источникам
            \else
                \insertbiblioauthor      % Вывод всех работ автора
            \fi
        \fi
        %  \insertbiblioauthorimportant  % Вывод наиболее значимых работ автора (определяется в файле characteristic во второй section)
        \begin{refcontext}[labelprefix={}]
            \insertbiblioexternal            % Вывод списка литературы, на которую ссылались в тексте автореферата
        \end{refcontext}
        % Невидимый библиографический список для подсчёта количества внешних публикаций
        % Используется, чтобы убрать приставку "А" у работ автора, если в автореферате нет
        % цитирований внешних источников.
        \printbibliography[heading=nobibheading, section=0, env=countexternal, keyword=biblioexternal, resetnumbers=true]%
    }
}
\ifdefmacro{\microtypesetup}{\microtypesetup{protrusion=true}}{}
\urlstyle{tt}                               % возвращаем установки шрифта ссылок URL
