%% Согласно ГОСТ Р 7.0.11-2011:
%% 5.3.3 В заключении диссертации излагают итоги выполненного исследования, рекомендации, перспективы дальнейшей разработки темы.
%% 9.2.3 В заключении автореферата диссертации излагают итоги данного исследования, рекомендации и перспективы дальнейшей разработки темы.

\begin{enumerate}
\item Предложен эффективный метод автоматического вывода индуктивных инвариантов, основанных на автоматах над деревьями, при этом данные инварианты позволяют выражать рекурсивные отношения для большого количества реальных программ; метод базируется на поиске конечных моделей.
\item Предложен метод автоматического вывода индуктивных инвариантов, основанный на трансформации программы и поиске конечных моделей, в классе инвариантов, основанном на синхронных автоматах над деревьями; этот класс позволяет выражать рекурсивные отношения и обобщает классические символьные инварианты.
\item Предложен класс индуктивных инвариантов, основанный на булевой комбинации классических инвариантов и автоматов над деревьями, который, с одной стороны, позволяет выражать рекурсивные отношения в реальных программах, а, с другой стороны, позволяет эффективно выводить индуктивные инварианты; также предложен эффективный метод совместного вывода индуктивных инвариантов в этом классе посредством вывода инвариантов в комбинируемых подклассах.
\item Проведено теоретическое сравнение существующих и предложенных классов индуктивных инвариантов; в том числе сформулированы и доказаны леммы о <<накачке>> для языка ограничений и для языка ограничений расширенного функцией размера терма, которые позволяют доказывать невыразимость инварианта в языке ограничений.
\item Выполнена пилотная программная реализация предложенных методов на языке \fsharp{} в рамках инструмента \theringen{}; инструмент сопоставлен с существующими методами на общепринятом тестовом наборе задач верификации функциональных программ <<Tons of Inductive Problems>>: реализация наилучшего из предложенных методов смогла за отведённое время решить в 3.74 раза больше задач, чем существующие инструменты.
\end{enumerate}

В рамках \textbf{рекомендации по применению результатов работы} в индустрии и научных исследованиях следует указать, что разработанные методы применимы для автоматизации рассуждений о системах дизъюнктов Хорна над теорией алгебраических типов данных, а также что их реализация выполнена в публично доступном инструменте \theringen{}. Созданный инструмент может быть использован в качестве основной компоненты для верификации в статических анализаторах кода и верификаторах для языков с алгебраическими типами данных, таких как \rust{}, \scala{}, \solidity{}, \haskell{} и \ocaml{}. Инструмент может быть использован для доказательства недостижимости ошибок или заданных фрагментов кода, что является важным для задач компьютерной безопасности и обеспечения качества.

В качестве \textbf{перспективы дальнейшей разработки тематики}, можно предложить расширение предложенных классов индуктивных инвариантов и методов их вывода на комбинации алгебраических типов данных с другими типами данных, распространённых в языках программирования, таких как целые числа, массивы, строковые типы данных. Это позволит выводить инварианты программ со сложными функциональными взаимосвязями между структурами и лежащими в них данными, что существенно расширит практическую применимость предложенных методов.
