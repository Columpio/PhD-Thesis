\chapter{Вывод регулярных инвариантов}\label{ch:fmf}

Основным вкладом данной главы является новый метод автоматического вывода индуктивных инвариантов систем над АТД при помощи инструментов автоматического доказательства теорем.
В разделе~\cref{sec:fmf/partialCorrectness} представлен метод и доказана его корректность для систем дизъюнктов упрощённого вида (без ограничений), а в разделе~\cref{sec:fmf/totalCorrectness}~--- для произвольных систем.
В разделе~\cref{sec:fmf/regular} рассмотрен класс регулярных инвариантов, которые могут быть выводимы при помощи предложенного метода.
В разделе~\cref{sec:fmf/specRegular} описано, как предложенный метод может быть применён для вывода регулярных инвариантов при помощи инструментов поиска конечных моделей.
В отличие от классических элементарных инвариантов, регулярные инварианты, основанные на автоматах над деревьями, позволяют выражать рекурсивные отношения, в частности, свойства алгебраических термов произвольной глубины.
Как указано в разделе~\cref{sec:fmf/totalCorrectness}, предложенный метод также может быть совмещён с общими инструментами автоматического доказательства теорем.

\section{Метод для систем без ограничений в дизъюнктах}\label{sec:fmf/partialCorrectness}

Основная идея метода заключается в следующем. Если у системы дизъюнктов Хорна над АТД без ограничений есть модель в свободной теории, то она безопасна и этой модели соответствует некоторый индуктивный инвариант.

\begin{example}\label{ex:evenNat}
Рассмотрим следующую систему дизъюнктов Хорна над алгебраическим типом чисел Пеано $Nat ::= Z\,|\,S\,Nat$. Эта система кодирует предикат чётности чисел Пеано $even$ и свойство: <<никакие два следующих друг за другом натуральных числа не могут быть чётными одновременно>>.
\begin{align}
    x = Z &\rightarrow even(x)\\
    x = S(S(y)) \land even(y) &\rightarrow even(x)\\
    even(x) \land even(y) \land y = S(x) &\rightarrow \bot
\end{align}

Хотя эта простая система безопасна, у неё нет классического символьного инварианта, что будет показано в главе~\cref{ch:comparison}. 
\end{example}

Эта система может быть переписана в следующую эквивалентную систему без ограничений в дизъюнктах.
\begin{align*}
    \top &\rightarrow even(Z)\\
    even(x) &\rightarrow even(S(S(x)))\\
    even(x) \land even(S(x))&\rightarrow \bot
\end{align*}

Ей соответствует следующая формула в свободной теории.
\begin{align*}
    \forall x. &(\top \rightarrow even(Z))\land \\
    \forall x. &(even(x) \rightarrow even(S(S(x))))\land \\
    \forall x. &(even(x) \land even(S(x))\rightarrow \bot)
\end{align*}

Эта формула выполняется следующей конечной моделью $ \structure $.
\begin{align*}
    \domain{Nat}&=\{0,1\}\\
    \structure(Z)&=0\\
    \structure(S)(x)&=1-x\\
    \structure(even)&=\{0\}
\end{align*}

\begin{lemma}[Корректность]\label{lemma:finiteToHebrand}
Пусть система дизъюнктов Хорна $\prog$ с неинтерпретированными предикатами $\relations=\{P_1,\ldots,P_k\}$ без
%equality and disequality 
ограничений в дизъюнктах выполняется в некоторой модели $\structure$, т.\:е. $\structure\models C$ для всех $C\in\prog$. Пусть также справедливо следующее:
\[X_i\eqdef \{\tuple{t_1,\ldots,t_n} \mid \tuple{\interprets{\structure}{t_1},\ldots,\interprets{\structure}{t_n}} \in \structure(P_i)\}.\]
Тогда $\tuple{\hs,X_1,\ldots,X_k}$ является индуктивным инвариантом $\prog$.
\end{lemma}
\begin{proof}
Все дизъюнкты имеют вид
$$\forall \overline{x}. C\equiv P_1(\overline{t}_1)\land\ldots\land P_m(\overline{t}_m)\rightarrow H.$$
Возьмём некоторый подходящий по сортам кортеж замкнутых термов $\overline{x}$. Тогда из $\structure\models \clforall{C}$, по определению $X_i$ следует, что
$$\overline{t}_1\in X_i \land \ldots \land \overline{t}_m\in X_m \rightarrow H',$$
где $H'$~--- соответствующая подстановка для $H$.
По определению выполнимости дизъюнкта Хорна, из этого следует, что
\[\tuple{\hs,X_1,\ldots,X_k}\models P_1(\overline{t}_1)\land\ldots\land P_m(\overline{t}_m)\rightarrow H. \]
\end{proof}

Таким образом, из конечной модели для примера выше строим множество $X\eqdef \{t\mid \interprets{\structure}{t}=0 \} = \{S^{2n}(Z) \mid n\geq 0\}$, которое является безопасным индуктивным инвариантом исходной системы.

\section{Метод для систем с ограничениями в дизъюнктах}\label{sec:fmf/totalCorrectness}

По системе с ограничениями в дизъюнктах можно построить эквивыполнимую систему без ограничений в дизъюнктах следующим образом. Без ограничения общности можно предположить, что ограничение каждого дизъюнкта содержит отрицания только над атомами. Литералы равенств термов могут быть устранены при помощи унификации~\cite{oppen1980reasoning}. Каждый же литерал неравенства вида $ \neg (t = _{\sigma} u) $ заменим на атомарную формулу $ diseq_{ \sigma} (t, u) $.
Для этого для каждого алгебраического типа $ (C, \sigma) $ введём новый неинтерпретированный символ $ diseq _{\sigma} $ и добавим его в множество реляционных символов $ \relations '\eqdef \relations \cup \{diseq _{\sigma} \mid \sigma \in \sorts \} $.

Далее построим по системе $\prog$ систему дизъюнктов $\prog'$ над $\relations'$ следующим образом.
Для каждого алгебраического типа $ (C, \sigma) $ в $ \prog '$ добавим следующие дизъюнкты для $ diseq _{\sigma} $:
\begin{align}\notag%\label{def:diseq-base}
    \top\rightarrow diseq_{\sigma}(c(\overline{x}), c'(\overline{x}')) \text{ для всех различных конструкторов $c$ и $c'\in C$ сорта $\sigma$}
\end{align}
и
\begin{align}\notag%\label{def:diseq-step}
diseq_{\sigma'}(x, y)\rightarrow diseq_{\sigma}(c(\ldots,\underbrace{x}_{\mathclap{i\text{-ая позиция}}},\ldots), c(\ldots,\underbrace{y}_{\mathclap{i\text{-ая позиция}}},\ldots))
\end{align}
для всех конструкторов $c$ сорта $\sigma$, всех $i$ и всех $x, y$ сорта $\sigma'$.

Для каждого сорта $\sigma\in\sorts$ обозначим диагональное множество $\diseqsem{}\eqdef \{ (x,y)\in\huniv{\sigma}^2 \mid x \neq y \}$.

Хорошо известно, что универсально замкнутые дизъюнкты Хорна имеют наименьшую модель, которая является денотационной семантикой программы, моделируемой системой дизъюнктов. Эта наименьшая модель является наименьшей неподвижной точка оператора перехода. Из этого тривиально следует следующая лемма.

\begin{lemma}\label{lemma:diseq-lfp}
Наименьший индуктивный инвариант дизъюнктов для $diseq_{\sigma}$ является кортежем отношений $\diseqsem{}$.
\end{lemma}
%
Простым следствием предыдущей леммы является следующий факт.
\begin{lemma}\label{lemma:diseqTransIsOk}
Для системы дизъюнктов Хорна $ \prog' $, полученной при помощи описанной выше трансформации,
если $\tuple{\hs, X_1, \ldots, X_k, Y_1, \ldots, Y_n} \models \prog'$, то $\tuple{\hs, X_1, \ldots, X_k, \diseqsem{1}, \ldots, \diseqsem{n}} \models \prog'$ (отношения $Y_i$ и $\diseqsem{i}$ интерпретируют предикатные символы $diseq_{\sigma_i}$).
\end{lemma}

\begin{example}
Система дизъюнктов $\prog=\{ Z\neq S(Z) \rightarrow \bot\}$ трансформируется в следующую систему $\prog'$.
\begin{align*}
\top&\rightarrow diseq_{Nat}(Z, S(x))\\
\top&\rightarrow diseq_{Nat}(S(x), Z)\\
diseq_{Nat}(x, y)&\rightarrow diseq_{Nat}(S(x), S(y))\\
diseq_{Nat}(Z, S(Z))&\rightarrow \bot
\end{align*}
\end{example}


Корректность трансформации, приведённой в данном разделе, доказывается в следующей теореме.
\begin{theorem}\label{th:diseqTrCorrectness}
Пусть $ \prog $~--- система дизъюнктов Хорна, а $ \prog '$~--- система дизъюнктов, полученная описанной трансформацией. Если существует модель системы $ \prog '$ в свободной теории, то у исходной системы $ \prog $ есть индуктивный инвариант.
\end{theorem}
\begin{proof}
Без ограничения общности можно предположить, что каждый дизъюнкт $ C \in \prog $ имеет следующий вид:
% (в противном случае мы переписываем ограничение в DNF, разбиваем его на различные предложения и удаляем все атомы равенства путем объединения и подстановки):
\[ C\equiv u_1 \neq t_1 \land \ldots \land u_k \neq t_k \land R_1(\overline{u}_1) \land \ldots \land R_m(\overline{u}_m) \rightarrow H.\]
В $\prog'$ этот дизъюнкт трансформируется в следующий дизъюнкт:
\[ C'\equiv diseq(u_1, t_1) \land \ldots \land diseq(u_k, t_k) \land R_1(\overline{u}_1) \land \ldots \land R_m(\overline{u}_m) \rightarrow H.\]

Таким образом, каждое предложение в $ \prog '$ не содержит ограничений (т.\:к. правила $ diseq $ также не содержат ограничений), а значит по лемме~\ref{lemma:finiteToHebrand} у $\prog'$ есть некоторый индуктивный инвариант $ \tuple{\hs, X_1, \ldots, X_k, U_1, \ldots, U_n} $. Тогда по лемме~\ref{lemma:diseqTransIsOk} имеем
$\tuple{\hs, X_1,\ldots, X_k, \diseqsem{1}, \ldots, \diseqsem{n}}\models C'$ для каждого $C'\in\prog'$.
Однако очевидно следующее:
$$\interprets{\tuple{\hs, X_1,\ldots, X_k, \diseqsem{1}, \ldots, \diseqsem{n}}}{C'}=\interprets{\tuple{\hs, X_1,\ldots, X_k}}{C}.$$ 
Это означает, что 
$\tuple{\hs, X_1,\ldots, X_k}\models C$ для каждого $C\in\prog$~--- желаемый индуктивный инвариант исходной системы.
\end{proof}

\paragraph{Использование метода для вывода инвариантов.}
Для проверки выполнимости формул первого порядка могут быть использованы инструменты автоматического доказательства теорем, строящие насыщения, такие как \vampire{}~\cite{kovacs2013first}, \eprover{}~\cite{10.5555/1218615.1218621} и \zipperposition{}~\cite{10.1007/978-3-319-66167-4_10}.
% Насыщения, при помощи которых они представляют модели, имеют высокую выразительную силу, т.\:к. позволяют выразить многие отношения, невыразимые в других классах.
Однако насыщения не дают эффективный класс инвариантов, поскольку даже проверка принадлежности кортежа замкнутых термов множеству, выраженному насыщением, неразрешима~\cite{4556689}.
По этой причине насыщения в качестве основы для самостоятельного класса инвариантов не рассматриваются в данной работе. Однако изучение их подклассов, как и создание процедур автоматического вывода для инвариантов в них многообещающи.
О применении метода для вывода более узкого класса регулярных инвариантов повествует следующий раздел.

\section{Регулярные инварианты}\label{sec:fmf/regular}

\begin{define}[\regclass{}]
Будем говорить, что $ n $-автомат $ A $ над $ \fsymbs $ \emph{выражает} отношение $ X \subseteq \huniv{\sigma_1} \times \ldots \times \huniv{\sigma_n} $, если
 $X = \langOf{A} $.
Отношение $ X $, для которого существуют выражающий его ДКАД, называется \emph{регулярным} отношением. Класс регулярных отношений будет обозначаться \regclass{}.

Пусть $ \prog $~--- система дизъюнктов Хорна. Если $ \overline{X} = \tuple{X_1, \ldots, X_n} $, где каждый $ X_i $ регулярен, и $ \tuple{\hs, \overline{X}} \models C $ для всех $ C \in \prog $, тогда $ \tuple{\hs, \overline{X}} $ называется \emph{регулярным инвариантом} $ \prog $.
\end{define}

\begin{example}\label{ex:evenInReg}
Система дизъюнктов Хорна из примера~\ref{ex:evenNat} имеет регулярный инвариант $ \langle \hs, \langOf{A} \rangle $, где $A$~--- это $ 1 $-ДКАД $ \Automaton{s_0, s_1, s_2}{s_0} $,
со следующим отношением перехода $ \autTrans $:
\exampleTwo

Множество $\langOf{A} = \{Z, S (S (Z)), S (S (S (S (Z))))), \ldots \} = \{ S ^{2n} (Z) \mid n \geq 0 \} $ очевидным образом удовлетворяет всем дизъюнктам системы.
\end{example}

\begin{example}\label{exmpl:incdec}
Рассмотрим следующую систему дизъюнктов с несколькими различными инвариантами.
\begin{align*}
    x = Z \land y = S(Z) &\rightarrow inc(x, y)\\
    x = S(x') \land y = S(y') \land inc(x', y') &\rightarrow inc(x, y)\\
    x = S(Z) \land y = Z &\rightarrow dec(x, y)\\
    x = S(x') \land y = S(y') \land dec(x', y') &\rightarrow dec(x, y)\\
    inc(x, y) \land dec(x, y) &\rightarrow \bot
\end{align*}
Эта система имеет следующий очевидный элементарный инвариант.
$$ inc (x, y) \equiv (y = S (x)), dec (x, y) \equiv (x = S (y)).$$
Этот инвариант является наиболее сильным из возможных, т.\:к. выражает денотационную семантику $ inc $ и $ dec $ соответственно. Эти отношения нерегулярны, то есть не существует 2-автоматов представляющих эти отношения~\cite{tata}.

Однако эта система дизъюнктов имеет другой, менее очевидный, регулярный инвариант, порождённый двумя $ 2 $-ДКАД $ \automaton{\{s_0, s_1, s_2, s_3\}}{\autStates_*}{\autTrans} $ с конечными состояниями соответственно $ \autStates_{inc} = \{\tuple{s_0, s_1}, \tuple{s_1, s_2}, \tuple{s_2, s_0} \} $, $ \autStates_{dec} = \{\tuple{s_1, s_0}, \tuple{s_2, s_1}, \tuple{s_0, s_2} \} $ и с правилами перехода, имеющими следующий вид:
\exampleOne

Автомат для $ inc $ проверяет, что $ (x \, \mathit{mod} \, 3, \, y \, \mathit{mod} \, 3) \in \{(0,1), (1 , 2), (2,0) \} $, а автомат для $ dec $ проверяет, что $ (x \, \mathit{mod} \, 3, \, y \, \mathit{mod} \, 3) \in \{(1,0), (2,1), (0,2) \} $. Эти отношения аппроксимируют сверху денотационную семантику $ inc $ и $ dec $ и при этом доказывают невыполнимость формулы $ inc (x, y) \land dec (x, y) $.

Таким образом, хотя многие отношения нерегулярны, у программ могут существовать неочевидные регулярные инварианты.
\end{example}

Более подробно свойства регулярных инвариантов рассмотрены в главе~\cref{ch:comparison}.


\section{Специализация метода для вывода регулярных инвариантов}\label{sec:fmf/specRegular}

\newworkflowWithTestersAndSelectors{}

Предложенный в прошлых разделах метод может быть специализирован для вывода регулярных инвариантов, как представлено на рисунке~\cref{fig:newworkflow-with-testers-selectors}.
При помощи трансформаций из разделов~\cref{sec:fmf/partialCorrectness} и~\cref{sec:fmf/totalCorrectness} по системе дизъюнктов Хорна над АТД можно получить эквивыполнимую формулу первого порядка над свободной теорией.
Если запустить на ней инструмент поиска конечных моделей, при помощи классического построения~---
теоремы~\ref{thm:finite-to-automaton} об изоморфизме между конечными моделями и автоматами над деревьями~--- можно получить автомат над деревьями, выражающий регулярный инвариант исходной системы дизъюнктов Хорна.
Корректность всего подхода гарантируется теоремами~\ref{th:diseqTrCorrectness} и~\ref{thm:finite-to-automaton}.

Например, по конечной модели из раздела~\cref{sec:fmf/partialCorrectness} для примера $ Even $ будет получен следующий автомат $ A _{Even} $, изоморфный представленному в примере~\ref{ex:evenInReg}.
\exampleCostruction

На практике это означает, что индуктивные инварианты систем дизъюнктов Хорна над АТД можно строить при помощи \emph{инструментов поиска конечных моделей}, таких как \mace{}~\cite{https://doi.org/10.48550/arxiv.cs/0310055}, \kodkod{}~\cite{10.1007/978-3-540-71209-1_49}, \paradox{}~\cite{claessen2003new}, а также \cvc{}~\cite{reynolds2013finite} и \vampire{}~\cite{10.1007/978-3-319-40970-2_20}.

\section{Выводы}\label{sec:fmf/conclusion}
Предложенный метод позволяет свести задачу поиска индуктивного инварианта системы дизъюнктов Хорна с АТД к задаче проверки выполнимости формулы универсального фрагмента логики первого порядка.
Поэтому в целом совместно с предложенным методом могут быть использованы произвольные инструменты автоматического доказательства теорем, такие как \vampire{}~\cite{kovacs2013first}, \eprover{}~\cite{10.5555/1218615.1218621} и \zipperposition{}~\cite{10.1007/978-3-319-66167-4_10}.
Такие инструменты возвращают доказательства выполнимости в виде насыщений, которые позволяют выражать широкий класс инвариантов. Однако проверка, что насыщение выражает индуктивный инвариант заданной системы, неразрешима, поэтому использование насыщений для выражения индуктивных инвариантов не представляется возможным.
Также совместно с предложенным методом могут быть использованы инструменты поиска конечных моделей, такие как, например, \mace{}~\cite{https://doi.org/10.48550/arxiv.cs/0310055}, \kodkod{}~\cite{10.1007/978-3-540-71209-1_49}, \paradox{}~\cite{claessen2003new}, а также \cvc{}~\cite{reynolds2013finite} и \vampire{}~\cite{10.1007/978-3-319-40970-2_20} в соответствующих режимах. Композиция предложенного метода и инструмента поиска конечных моделей может выводить регулярные инварианты, которые позволяют выражать рекурсивные отношения и представлять инварианты некоторых систем, для которых классических символьных инвариантов не существует. Кроме того, проверка, что заданный автомат над деревьями выражает регулярный инвариант заданной системы, разрешима.
Ограничением регулярных инвариантов является то, что они не позволяют представлять синхронные отношения, такие как равенство и неравенство термов, а потому существуют системы, у которых есть классический символьный инвариант, но нет регулярных.
Более богатый класс \emph{синхронных} регулярных инвариантов, решающий эту проблему, а также новый метод вывода инвариантов для этого класса рассмотрены в следующей главе.
