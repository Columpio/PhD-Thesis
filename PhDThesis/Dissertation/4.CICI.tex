\chapter{Коллаборативный вывод комбинированных инвариантов}\label{ch:cici}

В данной главе предложен метод вывода комбинированных инвариантов. 
Комбинированными инвариантами (раздел~\cref{sec:cici/combinedInvs}) мы называем  индуктивные инварианты программ, выраженные в расширении языка ограничений предикатами принадлежности терма языку из произвольного класса инвариантов.
Метод, представленный в разделе~\cref{sec:cici/inference}, позволяет модифицировать любой алгоритм вывода инвариантов в языке ограничений, основанный на направляемом контрпримерами уточнении абстракций (\cegar{})~\cite{cegar},  таким образом, чтобы этот алгоритм эффективно выводил комбинированные инварианты. Модификация этого метода  заключается в коллаборативном взаимообмене информацией с алгоритмом вывода инвариантов в классе, с которым производится комбинация (эта идея описана в  разделе~\cref{sec:cici/idea}).

\section{Идея коллаборативного вывода}\label{sec:cici/idea}
Для простоты изложения ключевая идея коллаборативного вывода представлена как модификации подхода \cegar{} для систем переходов. Как будет показано в следующем разделе, системы дизъюнктов Хорна и программы на произвольных языках программирования являются частным случаем систем переходов.

\subsection{\cegar{} для систем переходов}\label{sec:cici/origCEGAR}
Пусть $\tuple{\states, \csubseteq, \cbot, \ctop, \ccap, \ccup, \cneg}$~--- это полная булева решётка, представляющая множества состояний программы. 

\begin{define}
\emph{Система переходов (программа)} является тройкой $TS = \tuple{\states, \init, \thepost}$, где  $\init \in \states$~--- это \emph{начальные состояния}, а функция $\thepost : \states \mapsto \states$ называется \emph{функцией перехода} и  имеет  следующие свойства:
\begin{itemize}
    \item $\thepost$ монотонна, т.\:е. из $s_1 \csubseteq s_2$ следует $\post{s_1} \csubseteq \post{s_2}$;
    \item $\thepost$ аддитивна, т.\:е. $\post{s_1 \ccup s_2} = \post{s_1}\ccup\post{s_2}$;
    \item $\post{\cbot}=\init$.
\end{itemize}
\end{define}

\begin{define}
    Состояния $s\in\states$ называются \emph{достижимыми} из множества состояний $s'\in\states$, если существует такое $n\geq 0$, что $s = \thepost^n(s')$.
\end{define}

\begin{define}\label{defn:ind-invariant}
\emph{Проблема безопасности} является парой, включающей программу $TS$ и некоторое свойство $\property \in \states$.
Программа называется \emph{безопасной} относительно этого свойства, если для всех~$n$ выполняется $\thepost^n(\init) \csubseteq \property$, иначе она называется \emph{небезопасной}.

Безопасность может быть доказана при помощи \emph{(безопасного) индуктивного инварианта} $I \in \states$, для которого должно выполняться следующее:
\[ \init \csubseteq I, \quad
    \post{I} \csubseteq I, \quad
    I \csubseteq \property. \]
\end{define}

Так как все индуктивные инварианты по определению являются неподвижными точками функции перехода $\thepost$, то часто при поиске индуктивного инварианта будут рассматриваться именно неподвижные точки.

\begin{theorem}[см.~\cite{Floyd1993}]
Программа безопасна тогда и только тогда, когда она имеет безопасный индуктивный инвариант.
\end{theorem}

Для того, чтобы \emph{автоматически выводить} индуктивные инварианты, обычно фиксируется некоторый \emph{класс инвариантов} $\invClass \subseteq \states$.
% , который является эффективно представимым, например, класс множеств, представимых полиэдрами, или класс множеств, определимых эффективно-пропозициональным (EPR) фрагментом FOL.
\emph{Верификатор}~--- это алгоритм, который по проблеме безопасности возвращает инвариант в классе инвариантов $\invClass$ в том случае, если программа безопасна, или контрпример в противном случае.
Класс инвариантов $\invClass$ называется \emph{доменом} верификатора. Верификатор может не завершаться, например, в том случае, когда программа безопасна, но в его домене нет инварианта, доказывающего безопасность.

\begin{define}
Полную решётку $\abstrDomain = \tuple{\theAbstrDomain, \asubseteq, \abot, \atop, \acap, \acup}$ будем называть \emph{абстрактным доменом}, а её элементы~--- \emph{абстрактными состояниями}.

\emph{Связка Галуа} (Galois connection)~\cite{loiseaux1995property} или \emph{абстракция}~--- это пара отображений $\tuple{\alpha,\gamma}$ между частично упорядоченными множествами $\tuple{\states, \csubseteq}$ и $\tuple{\abstrDomain, \asubseteq}$, для которых выполнено следующее:
\begin{align*}
  \alpha &: \states \mapsto \abstrDomain, \qquad \gamma :\abstrDomain \mapsto \states,\\
  \forall x \in \states\ &\forall y \in \abstrDomain\ \alpha\left(x\right) \asubseteq y \Leftrightarrow x \csubseteq \gamma\left(y\right).
  \end{align*}
\end{define}

Абстрактный домен вместе со связкой Галуа однозначно определяет класс инвариантов $\{ \gamma(a)\mid a \in \abstrDomain \}$, который также будет обозначаться как $\abstrDomain$.
В дальнейшем подразумевается, что проверки вида $\gamma(a) \csubseteq \property$ вычислимы.

\begin{define}
\emph{Абстрактная функция перехода} $\theapost : \abstrDomain \mapsto \abstrDomain$ <<поднимает>> функцию перехода в абстрактный домен, т.\:е. для всех $a\in\abstrDomain$ справедливо следующее:
\[\alpha(\post{\gamma(a)})\asubseteq\apost{a}.\]
\end{define}

Представленная ниже  классическая теорема~\cite{10.1145/512950.512973} показывает, как абстракции могут быть применены для верификации.

\begin{theorem}
Пусть дана программа $TS=\tuple{\states, \init, \thepost}$ и свойство $\property$. Тогда $\gamma(a)$ является индуктивным инвариантом $\tuple{TS, \property}$, если существует элемент $a \in \abstrDomain$, такой, что выполнено
$
   \alpha(\init)\asubseteq a,\quad
    \apost{a} \asubseteq a,\quad
    \gamma(a)\csubseteq \property.
$
\end{theorem}

\begin{algorithm2e}[t]
	\KwIn{программа $TS$ и свойство $\property$.}
	\KwOut{$SAFE$ и индуктивный инвариант\\\quad или $UNSAFE$ и контрпример.}
	\BlankLine
    $\tuple{\alpha, \gamma}\leftarrow \textsc{Initial}()$\;
	\While{true}{
	    $cex, A \leftarrow \textsc{ModelCheck}(TS, \property, \tuple{\alpha, \gamma})$\;
	    \If{$cex$ пуст}{
	        \Return{$SAFE(A)$}
	    } \If{\textsc{IsFeasible}(cex)}{
	        \Return{$UNSAFE(cex)$}
	    }
        $\tuple{\alpha, \gamma} \leftarrow \textsc{Refine}(\tuple{\alpha, \gamma}, cex)$\;
	}
\caption{Подход \cegar{} для систем переходов}
\label{code:oldcegar}
\end{algorithm2e}

Псевдокод подхода \cegar{} для систем переходов представлен на листинге~\ref{code:oldcegar}.
Алгоритм начинается с построения начальной абстракции $\tuple{\alpha, \gamma}$, например, с помощью тривиальных отображений $\alpha(s) = \abot$ и $\gamma(a) = \ctop$.
Далее при помощи абстракции процедура \textsc{ModelCheck} строит по программе конечную последовательность
абстрактных состояний $\overline{a}=\tuple{a_{0},\ldots,a_n}$ таких, что:
\begin{align}\label{prop:cegar-inv}
    a_0 = \alpha(\init)\quad\text{ и }\quad
    a_{i+1} = a_i \acup \apost{a_i}  \quad\forall i \in\{0,\ldots,n-1\}.
\end{align}
Если для какого-то $i$ имеем $\gamma(a_i) \not\csubseteq \property$, то возвращается \emph{абстрактный контрпример} $cex$~--- либо в паре с $A=\cbot$ (если $i=0$), либо в паре с $A=\gamma(a_{i-1})$, причём $\gamma(a_{i-1}) \csubseteq \property$.
Если для всех $i$ справедливо $\gamma(a_i) \csubseteq \property$, и на некотором шаге выполняется $\apost{a_n} \asubseteq a_n$, то $\gamma(a_n)$ является индуктивным инвариантом, и поэтому \textsc{ModelCheck} возвращает пустой $cex$ и при этом $A=\gamma(a_n)$.
Понятие абстрактного контрпримера определяется в каждой конкретной реализации \cegar{}. Таким образом, возвращаемое процедурой \textsc{ModelCheck} значение удовлетворяет следующему свойству:
\begin{align}\label{prop:cegar-inv-A}
    A=\cbot\quad\text{ или }\quad\init \subseteq A \subseteq \property.
\end{align}

Если процедура \textsc{ModelCheck} вернула пустой абстрактный контрпример, то программа безопасна и \cegar{} возвращает $\gamma(a_n)$ в качестве индуктивного инварианта. В противном случае необходима проверка того, соответствует ли абстрактному контрпримеру какой-либо конкретный контрпример в исходной программе (процедура \textsc{IsFeasible}).
Если это так, то \cegar{} останавливается и возвращает этот контрпример, а иначе переходит к итеративному уточнению абстракции $\tuple{\alpha,\gamma}$ для исключения контрпримера $cex$ (процедура \textsc{Refine}).

\subsection{Коллаборативный вывод путём модификации \cegar{}}
В данном разделе предложен подход к коллаборативному выводу комбинированных инвариантов.
Подход основан на коллаборации двух алгоритмов вывода инвариантов и является асимметричным в следующем смысле.
Во-первых, требуется, чтобы один из алгоритмов был экземпляром \cegar{}, а другой может быть произвольным.
Во-вторых,  всем процессом управляет основной \cegar{}-цикл,  многократно вызывая второй алгоритм.

Предлагаемый подход назван \ourCEGAR{}, поскольку процесс <<коллаборации>> можно рассматривать как алгоритм \cegar{}, дополненный вызовами некоторого оракула $\oracle$. Пусть доменами верификаторов являются классы $\abstrDomain$ и $\blackBoxDomain$ соответственно. \ourCEGAR{} позволяет строить инварианты в классе, являющемся теоретико-множественным объединением этих классов.

\begin{define}\label{def:combined-class}
Для классов состояний $\abstrDomain\subseteq\states$ и $\blackBoxDomain\subseteq\states$ \emph{комбинированный класс} состояний определяется следующим образом:
$$\abstrDomain \uplus \blackBoxDomain \eqdef \{ A \ccup B \mid A\in\abstrDomain, B\in\blackBoxDomain \}.$$

\emph{Комбинированным (индуктивным) инвариантом} над $\abstrDomain$ и $\blackBoxDomain$ называется индуктивный инвариант в классе $\abstrDomain \uplus \blackBoxDomain$.
\end{define}

При помощи комбинированных инвариантов можно верифицировать больше программ, чем верификаторами для комбинируемых абстрактных доменов по отдельности.
% Инвариант может существовать в $\abstrDomain \uplus \blackBoxDomain$, даже если он не существует в $\abstrDomain$ и $\blackBoxDomain$. 
Коллаборативный подход объединяет сильные стороны верификаторов для отдельных классов и способен сходиться на многих задачах, где отдельные верификаторы не будут завершаться.

\begin{example}[$ForkJoin$]\label{ex:fork-join-chcs}
Рассмотрим трансформацию, которая преобразует параллельные программы следующим образом. 
На любом шаге трансформация может недетерминированно устранить все потоковые операции, объединив все потоки с главным ($Join$), и перейти к последовательному исполнению ($Seq$).
Если программа заканчивается последовательным кодом, то трансформация вставляет порождение новых потоков ($Fork$), за которым следуют произвольные преобразования потоков.
Если в каком-то фрагменте заданной программы встречается объединение потоков после порождения новых, то этот фрагмент не изменяется.

Эта трансформация может быть представлена в виде функциональной программы. Для её описания не нужно рассматривать базовые конструкции языка программирования кроме тех, которые связаны с потоками, поэтому для представления программ может быть использован следующий алгебраический тип данных:
$$ Prog ::= Seq\,|\,Fork(Prog)|\,Join(Prog). $$
Например, терм \texttt{Fork(Join(Seq))} представляет программу, которая порождает новые потоки, затем в какой-то момент их объединяет, а затем работает только последовательно.

Одним из свойств описанной трансформации является следующее: если исходная программа состоит из последовательности подряд идущих разветвлений и объединений потоков, то она никогда не может быть преобразована сама в себя.
Это свойство вместе с самой трансформацией представлено функциональной программой на листинге~\ref{fig:ex-fork-join}.
{\renewcommand{\figurename}{Листинг}
\begin{figure}
    \centering
\begin{minted}[breaklines,escapeinside=@@,linenos,frame=single,framesep=10pt]{fsharp}
type Prog = Seq | Fork of Prog | Join of Prog
fun randomTransform() : Prog
fun nondet() : bool

fun tr(p : Prog) : Prog =
    match nondet(), p with
    | false, Seq -> Fork(randomTransform())
    | false, Fork(Join(p')) -> Fork(Join(tr(p')))
    | _ -> Join(Seq)

fun ok(p : Prog) : bool =
    match p with
    | Seq -> true
    | Fork(Join(p')) -> ok(p')
    | _ -> false

(* для произвольной программы p : Prog *)
assert (not ok(p) or tr(p) <> p)
\end{minted}
    \caption{Пример функциональной программы с алгебраическими типами данных}
    \label{fig:ex-fork-join}
\end{figure}}

Функция \texttt{tr} выполняет описанную выше трансформацию над представлением программы алгебраическим типом данных $Prog$, в частности, вводит произвольные преобразования потоков, вызывая функцию \texttt{randomTransform}. Функция \texttt{ok} проверяет, что программа является последовательностью подряд идущих операций разветвления (\texttt{Fork}) и объединения (\texttt{Join}) потоков. Утверждение в конце кодирует свойство, которое необходимо проверить.

Эта программа безопасна относительно заданного свойства, однако не имеет индуктивных инвариантов, выразимых в классах \elemclass{} или \regclass{} для функций  \texttt{ok}  и  \texttt{tr}.
Вместе с тем у неё есть комбинированные инварианты в $\elemclass{}\uplus\regclass{}$.
Так, для любых программ $p, t : Prog$ функция $\texttt{ok}(p)$ возвращает \texttt{true}, если справедливо следующее утверждение:
\begin{align}\label{eq:for-join-ok-inv}
    p \in \evenInv{}.
\end{align}
Если $\texttt{tr}(p) = t$, тогда следующая формула представляет собой индуктивный инвариант для функции \texttt{tr}:
\begin{align}\label{eq:fork-join-tr-inv}
    \neg(p = t) \lor t \not\in \evenInv{},
\end{align}
где $t \not\in \evenInv{}$ означает, что АТД-терм $t$ \emph{не} содержится в языке $\evenInv{}$ следующего автомата над деревьями:
% \begin{figure}[h]
\vspace*{-2mm}\forkJoinExample{}\vspace*{-3mm}
% \end{figure}
\end{example}

\begin{algorithm2e}[t!]
	\KwData{верификатор $\oracle$ над доменом $\blackBoxDomain$.}
	\KwIn{программа $TS$ и свойство $\property$.}
	\KwOut{$SAFE$ и комбинированный инвариант в $\abstrDomain\uplus\blackBoxDomain$\\или $UNSAFE$ и контрпример $cex$.}
	\BlankLine
    $\tuple{\alpha, \gamma}\leftarrow \textsc{Initial}()$\;
    $A\leftarrow\cbot$\;
	\While{true}{
	    \textbf{асинхронно вызвать} $\RunBlackBox\left(TS, \property, \tuple{\alpha, \gamma}, A\right)$\label{line:call-bb}\;
	    $cex, A \leftarrow \textsc{ModelCheck}(TS, \property, \tuple{\alpha, \gamma})$\;
	    \If{$cex$ пуст}{
	        \Return{$SAFE(A)$}
	    } \If{\textsc{IsFeasible}(cex)}{
	        \Return{$UNSAFE(cex)$}
	    }
        $\tuple{\alpha, \gamma} \leftarrow \textsc{Refine}(\tuple{\alpha, \gamma}, cex)$\;
	}
\caption{\ourCEGAR{}}
\label{code:cegar}
\end{algorithm2e}

\paragraph{Описание алгоритма \textbf{\ourCEGAR{}}.}
Предлагаемый коллаборативный подход представлен на листинге~\ref{code:cegar}. Алгоритм работает аналогично классическому \cegar{}, представленному в разделе~\ref{sec:cici/origCEGAR}, но дополнительно в начале каждой итерации он асинхронно опрашивает коллаборирующий верификатор $\oracle$ путём вызова процедуры \RunBlackBox{} (строка~\ref{line:call-bb}). Вызовы делаются асинхронно, чтобы предотвратить зацикливание алгоритма.

\begin{algorithm2e}[t!]
	\KwData{Верификатор $\oracle$ над доменом $\blackBoxDomain$.}
	\KwIn{Программа $TS = \tuple{\states, \init, \thepost}$, свойство $\property$, абстракция $\tuple{\alpha, \gamma}$, множество состояний $A$, такое что $A=\cbot$ или $\init \subseteq A \subseteq \property$.}
% \KwOut{$SAFE$ с комбинированным инвариантом}
	\BlankLine
	$TS' \leftarrow \tuple{\states, \post{A}\csetminus A, \lambda B. \left(\post{B}\csetminus A\right)}$\;
	$B, cex \leftarrow \oracle\left( TS', \property \right)$\label{line:oracle-call}\;
	\If{$cex$ пуст}{
	    \Halt{$SAFE(A\ccup B)$}\label{line:halt-combined-safe}\;
	}
	$\widehat{cex} \leftarrow \textsc{RecoverCex}\left(TS, \property, \tuple{\alpha, \gamma}, A, cex\right)$\label{line:recover-cex}\;
	$\tuple{\alpha, \gamma}\leftarrow \textsc{Refine}(\tuple{\alpha, \gamma}, \widehat{cex})$\label{line:overwrite-abstraction}\;
\caption{Процедура \RunBlackBox{}.}
\label{code:runblackbox}
\end{algorithm2e}

Процедура {\textbf{\RunBlackBox}} представлена на листинге~\ref{code:runblackbox}. По исходной проблеме безопасности, текущей абстракции и множеству состояний $A = \gamma(a)$ для некоторого $a\in\abstrDomain$ она строит новую \emph{остаточную} систему переходов:
\[TS' = \tuple{\states,\init',\thepost'} = \tuple{\states, \post{A}\csetminus A,\; \lambda B. \left(\post{B}\csetminus A\right)}.\]
Затем безопасность остаточной системы проверяется коллаборирующим верификатором $\oracle$.
На листинге $A\csetminus B$ является сокращением для $A \ccap\cneg B$.
Процедура \RunBlackBox{} перезаписывает абстракцию, используемую в \cegar{} (стр.~\ref{line:overwrite-abstraction}): абстракция $\tuple{\alpha, \gamma}$ является глобальной и разделяется между двумя процедурами.

Остаточная система устроена следующим образом.
Её состояния в исходной системе достижимы из состояний, нарушающих индуктивность $A$.
В частности, её начальные состояния $\init'$~--- это $\post{A} \setminus A$, т.\:е. образ неиндуктивных состояний. Состояния, достижимые за один шаг в остаточной системе  $\thepost'(\init') = \post{\post{A} \setminus A} \setminus A$~--- это $\thepost$-образ неиндуктивных состояний.

Основная идея алгоритма \ourCEGAR{} заключается в том, чтобы использовать дополнительный верификатор $\oracle$ для \emph{ослабления} неиндуктивного множества состояний $A$ до некоторой неподвижной точки в комбинированном классе.
Если второй верификатор находит индуктивный инвариант остаточной системы, т.\:е. некую индуктивную аппроксимацию $B$ неиндуктивных состояний, то $A \ccup B$ будет индуктивным инвариантом исходной системы.
Иными словами, путём построения остаточной системы алгоритм берёт безопасную, но неиндуктивную часть текущего кандидата в инварианты и передаёт её сотрудничающему верификатору, чтобы он дополнил её до неподвижной точки, т.\:е. индуктивного инварианта.

Современные подходы к выводу индуктивных инвариантов программ, такие как \pdr{} (который можно рассматривать как усложнённый вариант \cegar{}), монотонно \emph{усиливают} кандидат в инварианты $A$ до тех пор, пока он не станет индуктивным.
В силу этого проблемой таких подходов является выбор стратегии усиления~\cite{krishnan2020global}: из-за слишком резкого усиления могут быть пропущены нужные неподвижные точки, в то время как из-за медленного усиления алгоритм может сходиться к неподвижной точке слишком долго или вовсе расходиться.

Так как предлагаемый подход не является монотонным, он позволяет строить инварианты, невыводимые верификаторами по отдельности. Кроме того, он (эвристически) может ускорить построение инварианта даже в том случае, если один из верификаторов может вывести его самостоятельно (эта гипотеза проверена в разделе~\ref{sec:evaluation/performance}). Вероятность пропуска неподвижных точек из-за слишком резкого усиления уменьшается: даже если первый верификатор чрезмерно усиливает кандидата в инварианты, второй верификатор всё равно может обнаружить более слабую неподвижную точку.

Таким образом, если второй верификатор $\oracle$ останавливается и возвращает индуктивный инвариант $B$, то \RunBlackBox{} возвращает комбинированный инвариант $A\ccup B$. Если $\oracle$ возвращает конкретный контрпример $cex$ к остаточной системе, то \RunBlackBox{} строит по нему абстрактный контрпример $\widehat{cex}$ к исходной системе и далее действует как обычный \cegar{}, уточняя домен с помощью $\widehat{cex}$.

Заметим, что множеств состояний $A$ и $B$ самих по себе недостаточно для доказательства безопасности исходной системы переходов. Другими словами, коллаборация осуществляется путём делегирования более \emph{простых} проблем верификатору $\oracle$, решение которых даёт только \emph{часть} ответа на исходную задачу.

\begin{lemma}\label{thm:runblackbox-safe}
Если процедура $\RunBlackBox\left(TS, \property, \tuple{\alpha, \gamma}, a\right)$ останавливается с результатом $SAFE(A \ccup B)$ (стр.~\ref{line:halt-combined-safe}), то $A \ccup B$ является комбинированным инвариантом проблемы $\tuple{TS, \property}$.
\end{lemma}
\begin{proof}
Докажем, что $A \ccup B$ является индуктивным инвариантом, показав, что для него выполняются все три признака индуктивных инвариантов из определения~\ref{defn:ind-invariant}: в этом множестве содержатся все начальные состояния, оно сохраняет отношение перехода и оно является подмножеством  свойства.

\textit{Начальные состояния.}
Из инварианта~\eqref{prop:cegar-inv-A} алгоритма \cegar{} получаем следующие случаи. Либо $\init\csubseteq A\csubseteq A\ccup B$, что и требовалось доказать. Либо $A=\cbot$, и тогда по определению $\post{\cbot}$, $\init=\post{\cbot}\csetminus \cbot=\post{A}\csetminus A \csubseteq B\csubseteq A \ccup B$.

\textit{Сохранение отношения перехода.}
В силу корректности верификатора $\oracle$ имеем, что множество $B$ является индуктивным инвариантом $\left( \tuple{\states, \init', \thepost'}, \property \right)$, т.\:е. $\post{A}\csetminus A\csubseteq B$ (определение $\init'$) и $\post{B}\csetminus A \csubseteq B$ (определение $\thepost'$).
Поэтому $\left(\post{A}\ccup \post{B}\right)\csetminus A\csubseteq B$, из чего следует $\post{A}\ccup \post{B}\csubseteq A\ccup B$, поэтому, так как функция $\thepost$ аддитивна, имеем $\post{A\ccup B} \csubseteq A\ccup B$.

\textit{Подмножество свойства.}
Справедливость $A \csubseteq \property$ следует из инварианта~\eqref{prop:cegar-inv-A} алгоритма \cegar{} и $B \subseteq \property$ по предположению корректности алгоритма $\oracle$. Следовательно, имеем $A\ccup B\csubseteq \property$.
\end{proof}

Контрпримерами для остаточной системы являются трассы, которые нарушают индуктивность текущего кандидата в инварианты $A$. \emph{Конкретный} контрпример к безопасности остаточной системы ($cex$ на стр.~\ref{line:oracle-call} с листинга~\ref{code:runblackbox}) соответствует некоторому \emph{абстрактному} контрпримеру исходной системы.
Поэтому \ourCEGAR{} параметризован процедурой \textsc{RecoverCex}, которая восстанавливает абстрактный контрпример к исходной системе по контрпримеру к остаточной системе (стр.~\ref{line:recover-cex}).
В следующем разделе~\ref{sec:cici/inference} предложена такая процедура для программ, представленных системами дизъюнктов Хорна, и контрпримеров, представленных деревьями опровержений.

Процедура \textsc{RecoverCex} должна удовлетворять следующему ограничению.

\begin{restrict}\label{thm:recovercex-sound}
$\textsc{RecoverCex}\left(TS, \property, \tuple{\alpha, \gamma}, a, cex\right)$
возвращает абстрактный контрпример к системе переходов $\tuple{TS, \property}$ относительно абстракции $\tuple{\alpha, \gamma}$.
\end{restrict}

\begin{theorem}
Если верификатор $\oracle$ корректен, то верификатор \ourCEGAR{} тоже корректен.
\end{theorem}
\begin{proof}
Справедливость этой теоремы непосредственно следует из корректности исходного \cegar{}~\cite{cegar}, леммы~\ref{thm:runblackbox-safe} и ограничения~\ref{thm:recovercex-sound}.
\end{proof}

\begin{theorem}
Если верно, что либо \cegar{}, либо верификатор $\oracle$ завершаются на системе $\tuple{TS,\property}$, то \ourCEGAR{} также завершается на этой же системе.
\end{theorem}
\begin{proof}
Если верификатор $\oracle$ завершается, то завершается и первый вызов процедуры $\RunBlackBox\left(TS, \property, \tuple{\alpha, \gamma}, \cbot\right)$, так как $\init' = \post{\cbot}\csetminus \cbot = \init$ и $\thepost' = \lambda B. (\post{B}\csetminus \cbot) = \thepost$.
Если \cegar{} завершается, то завершается и \ourCEGAR{}, так как вызов \RunBlackBox{} является асинхронным.
\end{proof}

\section{Коллаборативный вывод инвариантов}\label{sec:cici/inference}

В данном разделе весь подход в целом представлен как инстанциация алгоритма \ourCEGAR{} из прошлого раздела для систем дизъюнктов Хорна над АТД. 

Итоговый подход обладает следующими двумя свойствами.
Во-первых, он позволяет выводить индуктивные инварианты, выраженные в языке первого порядка над АТД, обогащённом ограничениями на принадлежность термов языкам деревьев $\overline{x}\in L$.
Во-вторых, подход позволяет расширить Хорн-решатели запросами к решателям для логики первого порядка, например, основанным на насыщении~\cite{kovacs2013first}, а также инструментам поиска конечных моделей~\cite{claessen2003new,reynolds2013finite}.

Прежде всего, определим, как будет выражаться класс комбинированных инвариантов.

\subsection{Комбинированные инварианты}\label{sec:cici/combinedInvs}
\begin{define}
Для каждого языка деревьев $\formallang\subseteq\huniv{\sigma_1}\times\dots\times\huniv{\sigma_m}$ определим предикатный символ принадлежности языку ``$\in\!\formallang$'' с арностью $\sigma_1\times\dots\times\sigma_m$. \emph{Ограничение принадлежности}~--- это атомарная формула с предикатным символом принадлежности языку.
Его семантика определяется расширением семантики Эрбрана $\hs$ следующим образом:
$\hs(\in\formallang)= \formallang.$
Язык ограничений АТД, расширенный такими атомами, называется \emph{языком первого порядка с ограничениями принадлежности}. Этот язык задаёт класс инвариантов обозначаемый $\regelemclass{}$ и абстрактный домен функций из предикатов $\relations$ в формулы языка с поэлементными операциями.
\end{define}

\begin{example}\label{exmp:running-example}
Функциональная программа из примера~\ref{ex:fork-join-chcs} соответствует следующей системе дизъюнктов Хорна:
\begin{align*}
p = Seq &\rightarrow ok(p)\\
p' = Fork(Join(p)) \land ok(p) &\rightarrow ok(p')\\
p = Seq \land t = Fork(p') &\rightarrow tr(p, t)\\
t = Join(Seq) &\rightarrow tr(p, t)\\
p' = Fork(Join(p)) \land t' = Fork(Join(t)) \land tr(p, t) &\rightarrow tr(p', t')\\
ok(p) \land tr(p, p) &\rightarrow \bot
\end{align*}

Она безопасна, но не имеет ни \regclass{}-, ни \elemclass{}-инвариантов.
Однако, у неё есть следующий \regelemclass{}-инвариант:
$$ok(p) \Leftrightarrow p \in \evenInv{},\quad tr(p, t) \Leftrightarrow \neg(p = t) \lor t \in \overline{\evenInv{}},$$
где $\evenInv{}$~--- это язык деревьев, задаваемый автоматом из примера~\ref{ex:fork-join-chcs},
а $\overline{\evenInv{}}$ является его дополнением.
\end{example}

\subsection{Система дизъюнктов Хорна как система переходов}

Зададим полную булеву решетку конкретных состояний $\tuple{\states, \csubseteq, \cbot, \ctop, \ccap, \ccup, \cneg}$. Определим $\states$ как множество всех отображений из символов $ P\in\relations$ в подмножества $\huniv{P}$. Определим также следующие операции:
\begin{align*}
  s_1 \csubseteq s_2 &\Leftrightarrow \forall P\in\relations\quad s_1(P)\subseteq s_2(P) &     s_1 \ccap s_2 &\eqdef \{P\mapsto s_1(P)\cap s_2(P) \mid P\in \relations \}\\
  \cbot&\eqdef \{P\mapsto\emptyset \mid P\in \relations \}& s_1 \ccup s_2 &\eqdef \{P\mapsto s_1(P)\cup s_2(P) \mid P\in \relations \}\\
  \ctop&\eqdef \{P\mapsto \huniv{P} \mid P\in \relations \}&  \cneg s &\eqdef \{P\mapsto \huniv{P} \setminus s(P) \mid P\in \relations \}
\end{align*}

Система дизъюнктов Хорна $\prog$ задаёт систему переходов $\tuple{\states, \init, \thepost}$:
\begin{align*}
    \init&\eqdef\post{\cbot}\\
    \post{s}(P)&\eqdef\left\{\overline{t} \mid
\left(B \rightarrow P(\overline{t})\right) \text{--- замкнутый экземпляр некоторого } C\in\prog,
s \models B \right\}
\end{align*}

Без потери общности предположим, что система дизъюнктов Хорна $\prog$ имеет единственный предикат для запроса $Q$, т.\:е. далее будем рассматривать только системы, полученные следующим образом:
$$ \prog' \eqdef \rules{\prog}\cup\left\{ \body{C}(\overline{x})\rightarrow Q(\overline{x})\mid C \text{ является запросом }\prog\right\}\cup\{Q(\overline{x})\rightarrow\bot\}.$$
Система дизъюнктов определяет свойство для системы переходов так: $\property(Q)\eqdef\bot$ и для каждого $P\in\relations$, $\property(P)\eqdef\top$.

\begin{proposition}
Система дизъюнктов Хорна $\prog$ выполнима, если соответствующая система переходов
$\tuple{\states, \init, \thepost}$ безопасна относительно $\property$.
\end{proposition}

\subsection{Порождение остаточной системы}\label{sec:subst_lemmas}

Процедура \RunBlackBox{} начинает с построения остаточной системы $\tuple{\post{A}\ccap \cneg A, \lambda B. \left(\post{B}\ccap \cneg A\right)}$. Эта система передаётся коллаборирующему верификатору.
Процедура $\substituteLemmas$, представленная на  листинге~\ref{code:residual-chc}, строит систему, эквивалентную такой остаточной системе, преобразуя исходную систему дизъюнктов Хорна $\prog$ в два шага.
Она принимает на вход исходную систему дизъюнктов $\prog$, элемент абстрактного домена $a$ и функцию из предикатных символов в формулы языка \regelemclass{}.

\begin{algorithm2e}[h]
	\KwIn{Cистема дизъюнктов Хорна $\prog$, функция из предикатных символов в формулы $a$.}
	\KwOut{Остаточная Хорн-система $\prog'$.}
	\BlankLine

    $\Phi \gets \prog\text{ с атомами }P(\overline{t})\text{ заменёнными на }a(P)(\overline{t})\lor P(\overline{t})$\label{line:subst-residual}\;
    \Return $КНФ(\Phi)$\label{line:cnf}\;

\caption{Алгоритм построения остаточной Хорн-системы \substituteLemmas{}.}
\label{code:residual-chc}
\end{algorithm2e}


Процедура заменяет каждый атом $P(t_1,\ldots,t_m)$ в голове и теле каждого дизъюнкта Хорн-системы дизъюнкцией $a(P)(t_1,\ldots,t_m)\lor P(t_1,\ldots,t_m)$ (стр.~\ref{line:subst-residual}), и затем приводит его в КНФ. Например, дизъюнкт 
$$P(x)\land \phi(x, x')\rightarrow P(x')$$ 
сначала превратится в формулу
$$ (a(P)(x) \lor P(x)) \land \phi(x, x') \rightarrow (a(P)(x') \lor P(x')), $$
которая после преобразования в КНФ (стр.~\ref{line:cnf}) будет разбита на следующие дизъюнкты:
\begin{align*}
  a(P)(x) \land \phi(x, x') \land \neg a(P)(x') &\rightarrow P(x')\\%\label{clause:tr-init}\\
  P(x) \land \phi(x, x') \land \neg a(P)(x') &\rightarrow P(x')%\label{clause:tr-trans}
\end{align*}
Таким образом, в результате преобразования в КНФ мы получаем систему дизъюнктов, которая семантически соответствует остаточной системе из прошлого раздела.

% \begin{figure}[t]
% \begin{align*}
% \forall p. \big( p = Seq &\rightarrow ok(p) \big) \land\\
% \forall p, p'. \big( p' = Fork(Join(p)) \land ok(p) &\rightarrow ok(p') \big) \land\\
% \forall p, t. \big( p = Seq \land t = Fork(p) &\rightarrow \left(\neg(p = t) \lor \neg(t = Seq) \lor L^{tr}(p, t)\right) \big) \land\\
% \forall p, t. \big( t = Join(Seq) &\rightarrow \left(\neg(p = t) \lor \neg(t = Seq) \lor L^{tr}(p, t)\right) \big) \land\\
% \forall p, p', t, t'. \big( p' = Fork(Join(p)) \land t' = Fork(Join(t)) \land \left(\neg(p = t) \lor \neg(t = Seq) \lor L^{tr}(p, t)\right) &\rightarrow \left(\neg(p' = t') \lor \neg(t' = Seq) \lor L^{tr}(p', t')\right) \big) \land\\
% \forall p. \big( ok(p) \land \left(\neg(p = p)  \lor \neg(p = Seq) \lor L^{tr}(p, p)\right) &\rightarrow \bot \big)
% \end{align*}
% \begin{align*}
% p = Seq &\rightarrow\left(p = Seq \lor ok(p)\right)&\\
% p' = Fork(Join(p)) \land \left(p = Seq \lor ok(p)\right) &\rightarrow\left(p' = Seq \lor ok(p')\right)&\\
% p = Seq \land t = Fork(p') &\rightarrow\left(\neg(p = t) \lor t = Join(Seq) \lor tr(p, t)\right)&\\
% t = Join(Seq) &\rightarrow\left(\neg(p = t) \lor t = Join(Seq) \lor tr(p, t)\right)&\\
% p' = Fork(Join(p)) \land t' = Fork(Join(t)) &\land \left(\neg(p = t) \lor t = Join(Seq) \lor tr(p, t)\right) \rightarrow&\\
% &\rightarrow\left(\neg(p' = t') \lor t' = Join(Seq) \lor tr(p', t')\right)&\\
% \left(p = Seq \lor ok(p)\right) &\land \left(\neg(p = p)  \lor p = Join(Seq) \lor tr(p, p)\right) \rightarrow \bot
% \end{align*}
% \caption{Example of applying transformation \substituteLemmas{}.}
%     \label{fig:exmp-transformed}
% \end{figure}

\begin{example}
Возьмём абстрактное состояние $a(tr)(p, t) \equiv \neg(p = t) \lor t = Join(Seq)$, $a(ok)(p) \equiv p = Seq$ и систему из примера~\ref{exmp:running-example}. Процедура \substituteLemmas{} сначала даст следующую  формулу:
\begin{align*}
  p = Seq &\rightarrow\big(p = Seq \lor ok(p)\big)&\\
  p' = Fork(Join(p)) \land \big(p = Seq \lor ok(p)\big) &\rightarrow\big(p' = Seq \lor ok(p')\big)&\\
  p = Seq \land t = Fork(p') &\rightarrow\big(\neg(p = t) \lor t = Join(Seq) \lor tr(p, t)\big)&\\
  t = Join(Seq) &\rightarrow\big(\neg(p = t) \lor t = Join(Seq) \lor tr(p, t)\big)&\\
  p' = Fork(Join(p)) \land t' = Fork(Join(t)) &\land&\\
  \land\big(\neg(p = t) \lor t = Join(Seq) \lor tr(p, t)\big)
  &\rightarrow\big(\neg(p' = t') \lor t' = Join(Seq) \lor tr(p', t')\big)&\\
  \big(p = Seq \lor ok(p)\big) \land \big(\neg(p = p)  &\lor p = Join(Seq) \lor tr(p, p)\big) \rightarrow \bot
  \end{align*}
% в \autoref{fig:exmp-transformed}.
Эта формула может быть упрощена до следующей системы дизъюнктов:
\begin{align*}
  p = Fork(Join(Seq)) &\rightarrow ok(p)\\
  p' = Fork(Join(p)) \land ok(p) &\rightarrow ok(p')\\
  t = Fork(Join(Join(Seq))) &\rightarrow tr(p, t)\\
  p' = Fork(Join(p)) \land t' = Fork(Join(t)) \land p' = t'\land tr(p, t) &\rightarrow tr(p', t')\\
  \big(p = Seq \lor ok(p)\big) \land \big(p = Join(Seq) \lor tr(p, p)\big) &\rightarrow \bot
  \end{align*}
\end{example}

\subsection{\ourCEGAR{} для дизъюнктов: восстановление контрпримеров}\label{sec:recover-cex}

В данном разделе представлена процедура построения абстрактного контрпримера исходной системы по конкретному контрпримеру для остаточной системы, получаемой как $\prog'=\substituteLemmas(\prog, a)$. Иными словами, представлена инстанциация процедуры \textsc{RecoverCex} из листинга~\ref{code:runblackbox}.

\paragraph{Абстрактные контрпримеры.}
Определим абстрактный контрпример к системе дизъюнктов Хорна как дерево опровержений, некоторые листья которого могут быть абстрактными состояниями.
Для формально определения введём трансформацию дизъюнктов $\simplifiedSubstituteLemmas{\prog, a}$, которая для каждого предиката $P \in \relations$ добавляет к системе $\prog$ новые дизъюнкты $ a(P)(\overline{x}) \rightarrow P(\overline{x}) $.

\begin{define}
\emph{Абстрактный контрпример} для Хорн-системы $\prog$ относительно абстрактного состояния $a$ является  деревом опровержений Хорн-системы $\simplifiedSubstituteLemmas{\prog, a}$.
\end{define}

Пусть $T$~--- это дерево опровержений для системы $\prog' = \substituteLemmas(\prog, a)$.
Далее приведена рекурсивная процедура построения дерева опровержений $T'$ для системы $\prog'' = \simplifiedSubstituteLemmas{\prog, a}$ по дереву $T$.

\textbf{База рекурсии}. Пусть $T$ состоит из единственной вершины $\tuple{C, \Phi}$, где $C \in \prog'$. Поскольку $\Phi=\body{C}$~--- формула без предикатов, то $C$ имеет следующий вид:
$$ \phi\land a(P_1)(\overline{x}_1)\land\ldots\land a(P_n)(\overline{x}_n)\land \neg a(P)(\overline{x})\rightarrow P(\overline{x}).$$
Построим $T'$ как вершину $\tuple{C', \Phi'}$, где $C'$ определим как
$
      \phi\land P_1(\overline{x}_1)\land\ldots\land P_n(\overline{x}_n) \rightarrow P(\overline{x})\quad\text{ и }\quad \Phi' \equiv \phi \land a(P_1)(\overline{x}_1) \land 
\ldots \land a(P_n)(\overline{x}_n)
$
с $n$ дочерними листьями $\tuple{C_i', a(P_i)(\overline{x}_i)}$, где $C_i'\equiv a(P_i)(\overline{x}_i)\rightarrow P_i(\overline{x}_i)$. Определение дерева опровержения для $T'$ тривиально выполняется. Заметим, что $\hs\models\Phi\rightarrow\Phi'$.

\textbf{Итерация рекурсии}. Пусть $T$~--- это узел $\tuple{C,\Phi}$ с детьми $\tuple{C_1, \Phi_1},\ldots,\tuple{C_n, \Phi_n}$, все $C_i \in \rules{P_i}$ из $\prog'$ и
\begin{align*}
  C &\equiv \phi \land a(R_1)(\overline{y}_1) \land \ldots \land a(R_m)(\overline{y}_m) \land \neg a(R)(\overline{y}) \land P_1(\overline{x}_1) \land \ldots \land P_n(\overline{x}_n) \rightarrow R(\overline{y})\\
  \Phi &\equiv \phi \land a(R_1)(\overline{y}_1) \land \ldots \land a(R_m)(\overline{y}_m) \land \neg a(R)(\overline{y}) \land \Phi_1(\overline{x}_1) \land \ldots \land \Phi_n(\overline{x}_n).
  \end{align*}
Благодаря итерации рекурсии у нас уже есть соответствующие им узлы $\tuple{C_1', \Phi_1'},\ldots,\tuple{C_n', \Phi_n'}$. Поэтому определим следующее:
\begin{align*}
C' &\equiv \phi\land R_1(\overline{y}_1) \land \ldots \land R_m(\overline{y}_m) \land P_1(\overline{x}_1) \land \ldots \land P_n(\overline{x}_n) \rightarrow R(\overline{y})\\
\Phi' &\equiv \phi \land a(R_1)(\overline{y}_1) \land \ldots \land a(R_m)(\overline{y}_m) \land \Phi_1'(\overline{x}_1) \land \ldots \land \Phi_n'(\overline{x}_n).
\end{align*}

Для каждого предиката $R_j$ добавим следующих потомков~--- $\tuple{C_{n+j}', a(R_j)(\overline{y}_j)}$, где 
$C_{n+j}'\equiv a(R_i)(\overline{y}_i)\rightarrow R_j(\overline{y}_j)$.

Для каждого $i$ имеем $\hs\models\Phi_i\rightarrow\Phi_i'$ по индукции, поэтому для их конъюнкции имеем $\hs\models\Phi\rightarrow\Phi'$.

В конце концов рекурсия придёт к корню дерева $T$, некоторой вершине $\tuple{C, \Phi}$, где $C$~--- это запрос из системы $\prog'$.
Для него рекурсивно построено дерево $T'$ с корнем $\tuple{C', \Phi'}$. Также по индукции имеем: $\hs\models\Phi\rightarrow\Phi'$. Поскольку $\Phi$ является выполнимой $\signature$-формулой, то $\Phi'$ тоже выполнима.
Таким образом, $T'$ является деревом опровержения системы $\prog''$.

\begin{proposition}
Процедура \textsc{RecoverCex} имеет линейную сложность от числа узлов входного дерева опровержения.
\end{proposition}

\subsection{Инстанцирование подхода в рамках \pdr{}}\label{sec:beyond-cegar}
Предложенный подход может быть реализован в рамках алгоритма \pdr{}~\cite{10.1007/978-3-642-18275-4_7}, который успешно применяется вывода индуктивных инвариантов в разных теориях~\cite{komuravelli2016smt}, если рассмотреть его как сложную версию алгоритма \cegar{}.

Приведённый алгоритм позволяет выводить комбинированные инварианты в классе $\elemclass{}\uplus\regclass{}$, т.\:е. индуктивные инварианты, выразимые формулами вида
$\phi(\overline{x})\,\lor\,\overline{x}\!\in\!L$,
где $\phi$~--- формула первого порядка над АТД, а $L$~--- язык деревьев.
Реализация подхода в рамках \pdr{} как сложной инстанциации \cegar{} может быть обобщена для автоматического вывода инвариантов в полном бескванторном фрагменте $\regelemclass{}$, состоящем из формул следующего вида:
\begin{align}\label{eq:inv-general-form}
    \bigwedge_i(\phi_i(\overline{x})\,\lor\,\overline{x}\!\in\!L_i).
\end{align}

\pdr{} представляет абстрактное состояние в виде конъюнкции формул (называемых \emph{леммами}). Другими словами, в процедуре $\substituteLemmas(\prog, a)$ (см. раздел~\ref{sec:subst_lemmas}) функция $a$ отображает каждый неинтерпретированный символ $P$ в некоторую конъюнкцию $\bigwedge_i \phi_i$. Обобщение подхода получается путём замены в этой процедуре применений неинтерпретированного предикатного символа $P$ на \emph{конъюнкции дизъюнкций} $\bigwedge_i (\phi_i(\overline{t})\lor L_i(\overline{t}))$ с новыми предикатными символами $L_i$. Таким образом, будут выводиться индуктивные инварианты вида~\ref{eq:inv-general-form} выше.

\section{Выводы}
Предложенный класс комбинированных инвариантов, построенный на регулярных инвариантах, позволяет выражать как классические символьные инварианты, так и сложные рекурсивные отношения.
Тем самым, предложенный класс инвариантов является достаточно выразительным и может использовать на  практике.
Кроме того, для него предложен эффективный метод вывода инвариантов, позволяющий путём небольшой модификации повторно использовать существующие эффективные алгоритмы вывода инвариантов для комбинируемых классов.
В следующей главе на теоретическом уровне сопоставлены существующие и предложенные классов индуктивных инвариантов.
