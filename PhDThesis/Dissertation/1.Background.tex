\chapter{Обзор предметной области}\label{ch:background}

В данной главе представлены ключевые для данного диссертационного исследования понятия и теоремы, а также описано состояние предметной области на момент написания работы.
В разделе~\cref{sec:background/historyExpressivity} приведена краткая история проблемы выразимости индуктивных инвариантов, которая является базовой для данного диссертационного исследования.
В разделе~\cref{sec:background/assertionLang} формально определены язык ограничений, логика первого порядка и алгебраические типы данных; именно с этими объектами будут оперировать предложенные в данной работе  методы верификации.
В разделе~\cref{sec:background/Horn} представлены системы дизъюнктов Хорна и показана их связь с задачей  верификации программ.
Для описания множеств термов алгебраических типов данных в разделе~\cref{sec:background/treeLangs} приведены базовые понятия формальных языков деревьев.
Наконец, в разделе~\cref{sec:background/conclusion} представлены выводы по обзору.


\section{Краткая история верификации программ}\label{sec:background/historyVerification}

Историю верификации принято начинать с отрицательных результатов: проблемы останова Тьюринга (1936~г.)~\cite{turing1936computable} и теоремы Райса (1953~г.)~\cite{10.2307/1990888}, которые говорят о невозможности существования верификатора, останавливающегося на всех входах и дающего только корректные результаты.
Первые конструктивные попытки создать подходы для верификации программ были предприняты  Р.\,В.~Флойдом (1967~г.)~\cite{Floyd1993} и Ч.\,Э.\,Р.~Хоаром (1969~г.)~\cite{10.1145/363235.363259}. Эти исследователи развили методы, основывающиеся на  сведении верификации к проверке логических условий.
% Хотя их методы активно развивались, они мало были использованы на практике из-за проблем с масштабируемостью~\cite{Clarke2018}.
Первым практичным подходом к верификации считается \emph{проверка моделей} (model checking, 1981~г.)~\cite{10.1007/BFb0025774}, возникшая в контексте верификации конкурентных программ.
% Проверка моделей решала задачу автоматической проверки созданных вручную моделей программ.
Её существенным ограничением был т.\:н. <<взрыв пространства состояний>>~\cite{10.1007/978-3-540-69850-0_1}: пространство состояний растёт \emph{экспоненциально} с ростом размерности состояния.

Для решения  этой проблемы К.\,Макмилланом была предложена \emph{символьная проверка моделей} (1987~г.), реализованная в инструменте SMV (1993~г.)~\cite{10.1007/3-540-61474-5_93}.
% Изначально в качестве символьного представления использовались бинарные диаграммы решений (binary decision diagram, BDD)~\cite{1676819}, однако
С 1996 года произошёл переход к представлению множеств состояний программы SAT-формулами логики высказываний (SATisfiability)~\cite{10.5555/1864519.1864564}, что позволило верифицировать системы, содержащие до  $10^{120}$ состояний~\cite{10.1007/3-540-61474-5_93}.
Это стало возможным благодаря новому поколению SAT-решателей, таких как \chaff{}~\cite{10.1145/378239.379017}, основанных на алгоритме проверки выполнимости формул с нехронологическим возвратом~--- CDCL (conflict driven clause learning)~\cite{silva1996grasp}.
На основе CDCL в 2002~г. был предложен алгоритм CDCL(T) для проверки выполнимости формул логики первого порядка в разных теориях (satisfiability modulo theories, SMT)~\cite{10.1007/3-540-45757-7_26}, спроектированных специально для задач в области формальных методов.
В 2002 г. был реализован первый SMT-решатель \textsc{CVC}~\cite{10.1007/3-540-45657-0_40}, использующий SAT-решатель \chaff{}.
% \nomenclature{BDD}{binary decision diagram, бинарная диаграмма решений\nomrefpage}

Появление эффективных SAT и SMT-решателей позволило вынести из процесса верификации проверку логических условий.
В 1999 г. была предложена \emph{ограничиваемая проверка моделей} (bounded model checking, BMC)~\cite{10.1007/3-540-49059-0_14}, строящая логические формулы из раскруток отношения переходов программы и отдающая их в сторонний решатель.
Затем, в \numrange{1995}{2000}~гг., благодаря Р.\,П.~Куршану и Э.\,Кларку, появился метод \emph{направляемого контрпримерами уточнения абстракций} (counterexample-guided abstraction refinement, \cegar{})~\cite{Kurshan1995,cegar}, который позволил  верифицировать программы путём итеративного построения индуктивных инвариантов в виде абстракций и их уточнения при помощи контрпримеров к индуктивности инвариантов программ.
\nomenclature{BMC}{bounded model checking, ограничиваемая проверка моделей\nomrefpage}
В \numrange{2003}{2005}~гг. К.\,Макмилланом было предложено строить абстракции при помощи \emph{интерполянтов} невыполнимых формул, извлекаемых из логического решателя~\cite{10.1007/978-3-540-45069-6_1,10.1007/978-3-540-31980-1_1}.
Интерполянты при этом, по сути, являются локальными (частичными) доказательствами корректности программы.
\nomenclature{\cegar{}}{counterexample-guided abstraction refinement, направляемое контрпримерами уточнение абстракций\nomrefpage}

В 2012~г. было предложено внедрить в стек <<верификатор, SMT-решатель, SAT-решатель>> еще также \emph{Хорн-решатель},  отвечающего за автоматический вывод индуктивных инвариантов и контрпримеров к спецификации~\cite{10.1145/2254064.2254112}.
Тем самым роль верификатора свелась  к синтаксической редукции программы к системе дизъюнктов Хорна, а <<ядром>> процесса верификации становится Хорн-решатель.
Так, например, \cegar{} был реализован в Хорн-решателе \eldarica{}.
В 2014~г. П.\,Гаргом был предложен подход ICE, основанный на обучении с учителем~\cite{10.1007/978-3-319-08867-9_5}.
ICE реализован, например, в Хорн-решателях \hoice{} и \rchc{}.

В 2011~г. А.\,Р.~Брэдли был предложен подход \pdr{}~\cite{10.1007/978-3-642-18275-4_7} для верификации аппаратного обеспечения на основе SAT-решателей.
К 2014~г. подход был обобщён для верификации программного обеспечения на основе SMT-решателей~\cite{10.1007/978-3-642-54862-8_4,10.1007/978-3-642-31612-8_13}.
\pdr{} усиливает \cegar{}, создавая абстракции путём построения индуктивных усилений спецификации, при этом равномерно распределяя ресурсы между поиском индуктивного инварианта и контрпримера.
\pdr{} реализован в Хорн-решателях \spacer{}~\cite{komuravelli2016smt} и \racer{}~\cite{10.1145/3498722}.

Благодаря эффективным алгоритмам Хорн-решатели всё больше применяются при верификации реальных программ, например, самоисполняющихся контрактов.

\section{История проблемы выразимости индуктивных инвариантов}\label{sec:background/historyExpressivity}
После появления логики Флойда-Хоара в \numrange{1967}{1969}~гг.~\cite{Floyd1993,10.1145/363235.363259} остро встал вопрос о достаточности предложенного исчисления для доказательства корректности всех возможных программ.
Иными словами, сразу была доказана корректность исчисления, но на долгие годы оставалась нерешённой проблема его \emph{полноты}, т.\:e. что предложенного исчисления достаточно, чтобы доказать безопасность всех безопасных программ.
Занимаясь это проблемой, в~1978~г. С.\,А.~Кук доказал~\cite{doi:10.1137/0207005} \emph{относительную} полноту логики Хоара.
Ограничение относительной полноты в теореме состояло в том, что все возможные слабейшие предусловия должны быть выразимы в языке ограничений.
Примерно с этого времени стали накапливались примеры простых программ, чьи инварианты невыразимы в языке ограничений~\cite{10.1145/371282.371285}.
Поэтому в~1987~г. А.~Бласс и Ю.~Гуревич предложили отказаться от логики первого порядка в пользу \emph{экзистенциальной логики с неподвижной точкой} (existential fixed-point logic)~\cite{Blass1987,blass2000the}. Она существенно более выразительна, чем логика первого порядка, поэтому для неё была доказана классическая теорема о полноте.


Следует отметить, что формулы экзистенциальной логики с неподвижной точкой соответствуют (при взятии отрицания) системам дизъюнктов Хорна с ограничениями~\cite{Bjorner2015}.
А последние позволяют выразить все возможные индуктивные инварианты программ, однако они не являются \emph{эффективным} представлением: задача проверки выполнимости систем дизъюнктов Хорна в общем случае неразрешима.
Поэтому проблема выразимости инвариантов не исчезла, но трансформировалась в основную проблему данного диссертационного исследования:
\emph{как выражать и эффективно строить решения систем дизъюнктов Хорна с ограничениями?}


На данный момент предлагаются различные подходы к практическому решению этой проблемы: от трансформации систем дизъюнктов в системы, у которых существование выразимого инварианта более вероятно (см. работы \numrange{2015}{2022}~гг. E.\,De\,Angelis,
F.\,Fioravant, A.\,Pettorossi~\cite{angelis_fioravanti_pettorossi_proietti_2015,10.1007/978-3-662-53413-7_8,10.1093/logcom/exab090,pettorossi_proietti_2022,10.1007/978-3-030-51074-9_6,angelis_fioravanti_pettorossi_proietti_2018}), в т.\:ч. синтаксических синхронизаций дизъюнктов~\cite{10.1007/978-3-662-53413-7_8,LPAR-21:Synchronizing_Constrained_Horn_Clauses}, до вывода реляционных инвариантов (инвариантов для нескольких предикатов)~\cite{mordvinov2020,DBLP:journals/corr/abs-2304-12588}.

Фактически, решению того же вопроса посвящены исследования в области \emph{полноты абстрактной интерпретации}~--- возникшего в~1977~г. подхода~\cite{10.1145/512950.512973}, который позволяет строить статические анализаторы, корректные по построению.
Неполнота возникает из-за аппроксимации неразрешимых свойств в разрешимом абстрактном домене, например, в разрешимом фрагменте логики первого порядка.


В 2000~г. было показано, что абстрактный домен можно уточнять по мере работы анализатора~\cite{10.1145/333979.333989}. Однако, как в 2015~г. показали Р.~Джакобацци и др.~\cite{giacobazzi2015analyzing}, это может привести к слишком точному абстрактному домену, в результате чего анализатор не будет завершаться.
Поэтому важнейшей частью проектирования абстрактного интерпретатора является построение абстрактного домена под конкретный класс задач~\cite{10.1093/logcom/2.4.511}.
Последние работы в области~\cite{10.1145/3498721,9470608} направлены на исследование точности анализа и \emph{локальной полноты}: полноты относительно заданного набора трасс.

\section{Язык ограничений}\label{sec:background/assertionLang}

Для произвольного множества $X$ определим следующие множества:
$X^{n}\eqdef \{\tuple{x_1,\ldots,x_n} \mid x_i \in X\}$ и $X^{\leq n} \eqdef \bigcup_{i=1}^n X^i$.

\subsection{Синтаксис и семантика языка ограничений}
Многосортная сигнатура первого порядка с равенством является кортежем $\signature = \tuple{\sorts, \fsymbs, \psymbs} $, где $\sorts $~--- множество сортов, $\fsymbs $~--- множество функциональных символов, $\psymbs $~--- множество предикатных символов, среди которых есть выделенный символ равенства $= _{\sigma} $ для каждого сорта $\sigma $ (индекс сорта у равенства везде далее будет опущен). Каждый функциональный символ $f \in \fsymbs $ имеет арность $\sigma_1 \times \dots \times \sigma_n \rightarrow \sigma $, где $\sigma_1, \ldots, \sigma_n, \sigma \in \sorts $, а каждый предикатный символ $p \in \psymbs $ имеет арность $\sigma_1 \times \dots \times \sigma_n $. Термы, атомы, формулы, замкнутые формулы и предложения языка первого порядка (ЯПП) определяются также, как обычно.
Язык первого порядка, определённый над сигнатурой $\signature$, будет называться \emph{языком ограничений}, а формулы в нём~--- $\signature$-формулами.

Многосортная структура (модель) $\structure $ для сигнатуры $\signature $ состоит из непустых носителей $\domain{\sigma} $ для каждого сорта $\sigma \in \sorts $. Каждому функциональному символу $f $ с арностью $\sigma_1 \times \dots \times \sigma_n \rightarrow \sigma $ сопоставим интерпретацию $\interprets{\structure}{f}: \domain{\sigma_1} \times \dots \times \domain{ \sigma_n} \rightarrow \domain{\sigma} $, и каждому предикатному символу $p $ с арностью $\sigma_1 \times \dots \times\sigma_n$ сопоставим интерпретацию $\interprets{\structure}{p} \subseteq \domain{\sigma_1} \times\dots\times\domain{\sigma_n} $. Для каждого замкнутого терма $t $ с сортом $\sigma$ интерпретация $\interprets{\structure}{t} \in \domain{\sigma} $ определяется рекурсивно естественным образом. 

Структура называется конечной, если все её носители всех сортов конечны, в противном случае она называется бесконечной.

Выполнимость предложения $\phi$ в модели $\structure$ обозначается $\structure \models \phi $ и определяется, как обычно.
Употреблением $\phi (x_1, \ldots, x_n) $ вместо $\phi$ будет подчёркиваться, что все свободные переменные в $\phi $ находятся среди $\{x_1, \ldots, x_n \} $. Далее, $\structure \models \phi (a_1, \ldots, a_n) $ обозначает, что $\structure $ выполняет $\phi$ на оценке, сопоставляющей свободным переменным элементы соответствующих носителей $a_1, \ldots, a_n $ (переменные также связаны с сортами). Универсальное замыкание формулы $\phi (x_1, \ldots, x_n) $ обозначается $\clforall{\phi} $ и определяется как $\forall x_1 \dots \forall x_n. \phi $. Если $\phi $ имеет свободные переменные, то $\structure \models \phi $ означает $\structure \models \clforall{\phi} $.
Формула называется \emph{выполнимой в свободной теории}, если она выполнима в некоторой модели той же сигнатуры.

\subsection{Алгебраические типы данных}
\emph{Алгебраический тип данных} (АТД) является кортежем $\tuple{C, \sigma} $, где $\sigma $~--- это сорт данного АТД, $C $~--- множество функциональных символов-конструкторов.
В научной литературе АТД также называют \emph{абстрактными типами данных}, \emph{индуктивными типами данных} и \emph{рекурсивными типами данных}.
Они позволяют задавать такие структуры данных как списки, бинарные и красно-чёрные деревья и др.

Пусть дан набор АТД $\tuple{C_1, \sigma_1}, \ldots, \tuple{C_n, \sigma_n} $ такой, что $\sigma_i \neq \sigma_j $ и $C_i \cap C_j = \emptyset $ при $i \neq j $.
В связи с фокусом данной работы далее мы будет рассматривать только сигнатуры теории алгебраических типов данных $\signature = \tuple{\sorts, \fsymbs, \psymbs} $, где $\sorts = \{\sigma_1, \ldots, \sigma_n \} $, $\fsymbs = C_1 \cup \dots \cup C_n $ и $\psymbs = \{= _{\sigma_1}, \ldots,= _{\sigma_n} \} $.
Поскольку $\signature $ не имеет предикатных символов, отличных от символов равенства (которые имеют фиксированные интерпретации внутри каждой структуры), существует единственная эрбрановская модель $\hs $ для $\signature $.
Носитель эрбрановской модели $\hs$~--- это кортеж $\tuple{\huniv{\sigma_1},\ldots,\huniv{\sigma_n}}$, где каждое множество $\huniv{\sigma_i}$~--- это  все замкнутые термы сорта $\sigma_i$.
Эрбрановская модель интерпретирует все замкнутые термы ими самими, поэтому служит стандартной моделью для теории алгебраических типов данных.
Формула $\phi $ языка ограничений будет называться \emph{выполнимой по модулю теории} АТД, если имеем $\hs \models \phi $.

Выполнимость формул в свободной теории, а также в теории АТД, может быть проверена автоматически при помощи так называемых \emph{SMT-решателей}, таких как \zprover{}~\cite{de2008z3}, \cvc{}~\cite{cvc5} и \princess{}~\cite{princess}, и посредством автоматических инструментов доказательства теорем, таких как \vampire{}~\cite{reger2017instantiation}.
Эти инструменты позволяют отделить задачу поиска доказательства безопасности программы от проверки таких доказательств,  автоматизируя последнюю задачу.


\section{Системы дизъюнктов Хорна с ограничениями}\label{sec:background/Horn}
Системы дизъюнктов Хорна~--- логический способ представлять программы совместно с их спецификациями.
К задаче проверки выполнимости систем дизъюнктов Хорна сводится задача верификации программ на самых разных языках программирования, от функциональных до объектно-ориентированных~\cite{Bjorner2015}.
Поэтому задача вывода индуктивных инвариантов в данной работе ставится и исследуется в формулировке для систем дизъюнктов Хорна, а дизъюнкты Хорна являются ключевым понятием для всей дальнейшей работы.

\subsection{Синтаксис}
Пусть $\relations = \{P_1, \dots, P_n \} $ является конечным множеством предикатных символов с сортами из сигнатуры $\signature $.
Такие символы будут называться \emph{неинтерпретированными}.
Формула $C $ над сигнатурой $\signature \cup \relations $ называется \emph{дизъюнктом Хорна с ограничениями}, если эта формула имеет следующий вид:
\begin{align*}
	\phi \land R_1(\overline{t}_1) \land \ldots \land R_m(\overline{t}_m) \rightarrow H.
\end{align*}
Здесь $\phi $~--- это \emph{ограничение} (формула языка ограничений без кванторов),  $R_i \in \relations $, а $\overline{t}_i $~--- кортеж термов. $H $ называется \emph{головой} дизъюнкта и может быть либо ложью $\bot $ (тогда дизъюнкт называется \emph{запросом}), либо атомарной формулой $R (\overline{t}) $ (тогда дизъюнкт называется \emph{правилом} \emph{для $R $}). При этом $R \in \relations $ и $\overline{ t} $ является кортежем термов.
Множество всех правил для $R \in \relations$ обозначается $\rules{R}$.
Посылка импликации $\phi \land R_1 (\overline{t} _1) \land \ldots \land R_m (\overline{t} _m) $ называется \emph{телом} формулы $C $ и обозначается $\body{C}$.

Системой дизъюнктов Хорна $\prog $ называется конечное множество дизъюнктов Хорна с ограничениями.

\subsection{Выполнимость и безопасные индуктивные инварианты}
Пусть $\overline{X} = \tuple{X_1, \ldots, X_n} $ является кортежем таких отношений, что если предикат $P_i $ имеет сорт $\sigma_1 \times \ldots \times \sigma_m $, то справедливо $X_i \subseteq \huniv{\sigma_1} \times \ldots \times \huniv{\sigma_m} $.
Для упрощения обозначений расширение модели $\hs \{P_1 \mapsto X_1, \ldots, P_n \mapsto X_n \} $ будет записываться как $\tuple{\hs, X_1, \ldots, X_n} $ или просто $\tuple{\hs, \overline{X}} $.

Система дизъюнктов Хорна $\prog$ называется \emph{выполнимой по модулю теории} АТД (или \emph{безопасной}), если существует кортеж отношений $\overline{X} $ такой, что выполнено $\tuple{\hs, \overline{X}} \models C $ для всех формул $C \in \prog $. В таком случае кортеж $\overline{X} $ называется \emph{(безопасным индуктивным) инвариантом} системы $\prog$.
Таким образом, по определению система дизъюнктов Хорна выполнима тогда и только тогда, когда у неё существует безопасный индуктивный инвариант.

Поскольку индуктивный инвариант $\overline{X}$~--- это кортеж множеств, которые для большинства систем будут бесконечными, то для автоматического вывода индуктивных инвариантов выбирается некоторый фиксированный класс, элементы которого выразимы некоторым конечным образом. Именно такие классы рассматриваются  в данной работе.

Важно отметить, что у  системы дизъюнктов Хорна может не быть индуктивного инварианта (если она невыполнима), может быть один индуктивный инвариант, а также их может быть несколько, в том числе бесконечно много.
Также важно, что если некоторый алгоритм ищет индуктивные инварианты системы дизъюнктов Хорна только в заранее заданном классе, то может возникнуть ситуация, что данная система  выполнима, однако ни один из её индуктивных инвариантов невыразим в этом классе. Как правило, это приводит к тому, что на такой системе алгоритм не завершает свою работу.

Здесь и далее нотация $\prog \in \mathcal{C}$, где $\prog$~--- название примера с некоторой системой дизъюнктов Хорна \emph{с одним неинтерпретированным символом}, а $\mathcal{C}$~--- некоторый класс индуктивных инвариантов, означает, что система $\prog$ безопасна и \emph{некоторый} её безопасный индуктивный инвариант (отношение, интерпретирующее единственный предикат) лежит в классе $\mathcal{C}$.

\begin{define}[\elemclass{}]
Отношение $ X \subseteq \domain{\sigma_1} \times \dots \times \domain{\sigma_n} $ \emph{выразимо языком АТД первого порядка} (\emph{элементарно}), если существует $\signature$-формула $ \phi (x_1, \ldots, x_n) $ такая, что $ (a_1, \ldots, a_n) \in X $ тогда и только тогда, когда $ \hs \models \phi (a_1, \ldots, a_n) $.  Класс всех элементарных отношений будем обозначать $ \elemclass{} $. Инварианты, лежащие в этом классе, называются \emph{элементарными}, а также \emph{классическими символьными инвариантами}.
\end{define}

\subsubsection{Элементарные инварианты с ограничениями размера термов}
Инструмент \eldarica{}~\cite{8603013} выводит инварианты системы дизъюнктов Хорна над АТД в расширении языка ограничений ограничениями на размер термов. Определим класс инвариантов, выразимых формулами этого языка.

\begin{define}[\sizeelemclass{}]
Сигнатура \sizeelemclass{}  получается из языка \elemclass{} путем добавления в  сорта $ Int $, операций из арифметики Пресбургера и функциональных символов $ \sizename{} _ \sigma $ с арностью $ \sigma \rightarrow Int $. Для краткости мы будем опускать знак $ \sigma $ в символах $ \sizename{} $.

Выполнимость формул с ограничениями на размер термов проверяется в структуре $ \hs_{\sizename{}} $, полученной путём соединения стандартной модели арифметики Пресбургера с эрбрановской моделью $ \hs $ и следующей интерпретацией функции размера:
$$\interprets{\hs_{\sizename{}}}{\sizename{}(f(t_1,\ldots,t_n)}\eqdef 1 + \interprets{\hs_{\sizename{}}}{t_1} + \ldots + \interprets{\hs_{\sizename{}}}{t_n}.$$
\end{define}

Например, размер терма
$t\equiv cons\big(Z, cons(S(Z), nil)\big) $
% с сортом $List$, определяемым как $List := nil : List \mid cons : Nat \times List \rightarrow List$,
в объединённой структуре вычисляется следующим образом: $\interprets{\hs_{\sizename{}}}{\sizename{}(t)}=6$.

\subsection{Невыполнимость и резолютивные опровержения}
Хорошо известно, что невыполнимость системы дизъюнктов Хорна может быть засвидетельствована резолютивным опровержением.

\begin{define}
\emph{Резолютивное опровержение} (дерево опровержений) системы дизъюнктов Хорна $\prog$~--- это конечное дерево с вершинами $\tuple{C, \Phi}$, где
\begin{enumerate}[label=(\arabic*)]
\item $C \in \prog$ и $\Phi$~--- это $\signature \cup \relations$-формула;
\item в корне дерева находится запрос $C$ и выполнимая $\signature$-формула $\Phi$;
\item в каждом листе дерева содержится пара $\tuple{C, \body{C}}$, причём $\body{C}$ является $\signature$-формулой;
\item вершина дерева $\tuple{C, \Phi}$ имеет детей $\tuple{C_1, \Phi_1},\ldots,\tuple{C_n, \Phi_n}$, если верно  следующее:
\begin{itemize}
    \item $\body{C} \equiv \phi \land P_1(\overline{x}_1) \land 
\ldots \land P_n(\overline{x}_n)$;
    \item $C_i \in \rules{P_i}$;
    \item $\Phi \equiv \phi \land \Phi_1(\overline{x}_1) \land 
\ldots \land \Phi_n(\overline{x}_n)$.
\end{itemize}
\end{enumerate}
\end{define}

\begin{theorem}
  У системы дизъюнктов Хорна есть резолютивное опровержение тогда и только тогда, когда она невыполнима.
\end{theorem}

\subsection{От верификации к решению систем дизъюнктов Хорна}
Инструменты, позволяющие автоматически решать задачу проверки выполнимости системы дизъюнктов Хорна, называются \emph{Хорн-решателями}. Как правило, Хорн-решатель для некоторой системы дизъюнктов возвращает индуктивный инвариант или резолютивное опровержение, хотя также может вернуть результат <<неизвестно>> или не завершиться.

При помощи различных теоретических подходов задача верификации программ может быть сведена к задаче проверки выполнимости системы дизъюнктов Хорна~\cite{10.1145/2254064.2254112,Bjorner2015}.
Среди таких теоретических подходов значимыми являются логика Флойда-Хоара для императивных программ~\cite{Floyd1993,10.1145/363235.363259}, а также зависимые (dependent types)~\cite{10.1145/292540.292560} и уточняющие типы (refinement types)~\cite{713327} для функциональных программ.
Существует множество инструментов, в рамках которых может быть реализовано это сведение, например, \liquidHaskell{}~\cite{10.1145/2692915.2628161} для языка \haskell{}, \rcaml{}~\cite{10.1007/978-3-319-63390-9_30} для языка \ocaml{}, \flux{}~\cite{https://doi.org/10.48550/arxiv.2207.04034} для языка \rust{}, \leon{}~\cite{10.1007/978-3-642-23702-7_23} и \stainless{}~\cite{10.1145/3360592} для языка \scala{}.
На основе этих подходов построены такие инструменты, как \rustHorn{}~\cite{10.1145/3462205}~--- верификатор для языка \rust{}, и \solCMC{}~\cite{10.1007/978-3-031-13185-1_16}~--- верификатор самовыполняющихся контрактов на языке \solidity{}. Эти инструменты напрямую используют Хорн-решатели с поддержкой АТД, такие как \spacer{} и \eldarica{}.

\section{Языки деревьев}\label{sec:background/treeLangs}
Различные множества АТД-термов, называемые языками деревьями, исследуются в рамках формальных языков как обобщения языков строк.
В частности, исследуется обобщение (строковых) автоматов до автоматов над деревьями, а также различные их расширения, как правило, обладающие свойствами разрешимости и замкнутости базовых языковых операций (например, проверки на пустоту пересечения языков)~\cite{chabin2007visibly, gouranton2001synchronized, limet2001weakly, chabin2006synchronized, jacquemard2009rigid, engelfriet2017multiple}.
Для данной работы различные классы языков деревьев представляют интерес, поскольку они могут служить в качестве классов безопасных индуктивных инвариантов программ, использующих АТД.

\subsection{Свойства и операции}
Для построения эффективного алгоритма вывода инвариантов от класса инвариантов, как правило, требуются следующие свойства: замкнутость относительно булевых операций, разрешимость задачи принадлежности кортежа инварианту и разрешимость задачи проверки пустоты инварианта.

\begin{define}[Замкнутость]
Пусть операция $\booleanOp$~--- это или $\cap$ (пересечение множеств), или $\cup$ (объединение множеств), или $\setminus$ (вычитание множеств). Класс множеств называется \emph{замкнутым относительно бинарной операции $\booleanOp$}, если для каждой пары множеств $X$ и $Y$ из данного класса множество $X \booleanOp Y$ также лежит в этом классе.
\end{define}

\begin{define}[Разрешимость принадлежности]
    Задача по определению принадлежности кортежа замкнутых термов некоторому множеству термов разрешима в некотором классе множеств термов тогда и только тогда, когда разрешимо множество пар из кортежей замкнутых термов $\overline{t}$ и элементов этого класса $i$ таких, что $i$ выражает некоторое множество $I$ и выполняется $\overline{t}\in I$.
\end{define}

\begin{define}[Разрешимость пустоты]
    Задача определения пустоты множества разрешима в классе множеств термов  тогда и только тогда, когда разрешимо множество элементов класса, выражающих пустое множество.
\end{define}

\subsection{Автомат над деревьями}
Автоматы над деревьями являются обобщением классических строковых автоматов на языки деревьев (языки термов), сохраняющим свойства разрешимости и замкнутости базовых операций. 
Классические результаты для автоматов над деревьями и их расширений представлены в книге~\cite{tata}.

\begin{define}\label{sec:background/TA}
  \emph{(Конечный) $n$-автомат (над деревьями)} над алфавитом $ \fsymbs $ является кортежем $ \automatonDef $, где $ \autStates $~--- это (конечное) множество состояний, $ \autFinStates \subseteq \autStates ^ n $~--- множество конечных состояний, $ \autTrans $~--- отношение перехода с правилами следующего вида:
  \begin{align*}
    f (s_1, \ldots, s_m) \rightarrow s.
  \end{align*}
Здесь использованы следующие обозначения: функциональные символы~--- $ f \in \fsymbs $, их арность~--- $ ar (f) = m $ и состояния~--- $ s, s_1, \ldots, s_m \in \autStates $.

  Автомат называется \emph{детерминированным}, если в $ \autTrans $ нет правил с совпадающей левой частью.
  \end{define}
  
  \begin{define}
  Кортеж замкнутых термов $ \tuple{t_1, \ldots, t_n} $ \emph{принимается} (допускается) $ n $-автоматом
  $ A = \automatonDef $, если $ \tuple{A [t_1], \ldots, A [t_n]} \in S_F $, где
  \begin{align*}
    A \big[f(t_{1},\ldots,t_{m}) \big] \eqdef\left\{\begin{array}{ll}
    s, &\text{ если } \big(f (A[t_1],\ldots, A[t_m])\rightarrow s\big)\in\autTrans,\\
    \text{не определено}, &\text{ иначе}.
    \end{array}\right.
  \end{align*}

  \emph{Язык автомата} $A$, обозначаемый $\langOf{A}$,~--- это множество всех допустимых автоматом $A$ кортежей термов.
  \end{define}
  
  \begin{example}
  Пусть $\signature = \tuple{Prop, \{ (\_\land\_), (\_\rightarrow\_), \top, \bot \}, \emptyset}$ является сигнатурой логики высказываний. Рассмотрим автомат $ A = \Automaton{q_0, q_1}{q_1} $ с  набором отношений перехода $ \autTrans $, представленными ниже.
  \begin{equation*}
  \begin{aligned}[c]
  q_1 \land q_1 &\mapsto q_1\\
  q_1 \land q_0 &\mapsto q_0\\
  q_0 \land q_1 &\mapsto q_0\\
  q_0 \land q_0 &\mapsto q_0
  \end{aligned}
  \qquad\qquad
  \begin{aligned}[c]
  q_1 \rightarrow q_0 &\mapsto q_0\\
  q_1 \rightarrow q_1 &\mapsto q_1\\
  q_0 \rightarrow q_0 &\mapsto q_1\\
  q_0 \rightarrow q_1 &\mapsto q_1
  \end{aligned}
  \qquad\qquad
  \begin{aligned}[c]
  \bot &\mapsto q_0\\
  \top &\mapsto q_1
  \end{aligned}
  \end{equation*}
  Автомат $A$ допускает только истинные пропозициональные формулы без переменных.
  \end{example}

\subsection{Конечные модели}
Существует взаимно-однозначное соответствие между конечными моделями формул свободной теории и автоматами над деревьями~\cite{kozen2012automata}. На основе  этого соответствия  можно создать следующую процедуру построения автоматов над деревьями по конечным моделям. 
По конечной модели $ \structure $ для каждого предикатного символа $ P \in \psymbs $ строится автомат $ A_P = \automaton{\thedomain{}}{\structure (P)}{\autTrans} $; для всех автоматов определено общее отношение переходов $ \autTrans $~--- для каждого $ f \in \fsymbs $ с арностью $ \sigma_1 \times \ldots \times \sigma_n \mapsto \sigma $ и для каждого $ x_i \in \domain{ \sigma_i} $ положим $ \autTrans \big (f (x_1, \ldots, x_n) \big) = \structure(f) (x_1, \ldots, x_n) $.

\begin{theorem}\label{thm:finite-to-automaton}
Для любого построенного автомата $A_P$ справедливо следующее утверждение: $$\langOf{A_P} = \{\tuple{t_1,\ldots,t_n} \mid \tuple{\interprets{\structure}{t_1},\ldots,\interprets{\structure}{t_n}} \in \structure(P)\}).$$
\end{theorem}

Практическая ценность этого результата заключается в том, что построение автомата над деревьями по формуле эквивалентно поиску конечной модели для неё.
Таким образом, ряд инструментов, таких как \mace{}~\cite{https://doi.org/10.48550/arxiv.cs/0310055}, \kodkod{}~\cite{10.1007/978-3-540-71209-1_49}, \paradox{}~\cite{claessen2003new}, а также \cvc{}~\cite{reynolds2013finite} и \vampire{}~\cite{10.1007/978-3-319-40970-2_20} могут быть использованы для поиска конечных моделей формул свободной теории и, как следствие, для автоматического построения автоматов над деревьями.


Большинство из этих инструментов реализуют кодирование в SAT: конечный домен и функции над ним кодируются в битовое представление, после этого по нему  получают формулу логики высказываний, которую передают в SAT-решатель.
Данные инструменты применяются для задач верификации~\cite{lisitsa2012finite}, а также  для построения бесконечных моделей формул первого порядка~\cite{peltier2009constructing}.

\section{Выводы}\label{sec:background/conclusion}
Ключевую роль в формальных методах, в особенности, в статическом анализе, играет задача автоматического вывода индуктивных инвариантов программ.
Не смотря на то, что существует некоторое количество весьма проработанных методов вывода индуктивных инвариантов,
и каждый год по этой теме появляются публикации  на различных конференциях по информатике и языкам программирования ранга A* (POPL, PLDI и CAV и др.).
Также каждый год проводятся соревнования между соответствующими инструментами,
всё ещё остаётся открытой проблема следующая проблема: как лучше выражать индуктивные инварианты программ.
Проблема подбора наилучшего представления инвариантов состоит в том, чтобы, с одной стороны, были выразимы инварианты реальных программ, а, с другой стороны, существовала бы эффективная процедура вывода инвариантов.
Эта проблема стоит наиболее остро в контексте алгебраических типов данных, для которых классические способы представлять инварианты крайне малоэффективны, а если инвариант не представим, то алгоритм его вывода в этом представлении не будет завершаться.
Это делает данное диссертационное исследование востребованным и актуальным.
